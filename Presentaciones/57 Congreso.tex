% arguelles v2.3.0
% author: Michele Piazzai
% contact: michele.piazzai@uc3m.es
% license: MIT

% Copied from GitHub: https://github.com/piazzai/arguelles

\documentclass[compress,12pt]{beamer}

\usetheme{Arguelles}
\usepackage{graphicx}
\usepackage{caption}
%\usepackage[spanish,es-noshorthands]{babel}
\usepackage[spanish]{babel} 
\usepackage[pages=some]{background}
\usepackage{tikz-cd}
\usepackage{amsmath,amssymb,latexsym,amscd} 
\usepackage[all,cmtip]{xy}
\usepackage{fancyhdr}
\usepackage{mathalfa}
\usepackage{mathrsfs}
\usetikzlibrary{babel}
\usepackage{hyperref}
\usepackage{ragged2e}
\usepackage{wasysym}
%\hypersetup{colorlinks=true,linkcolor=blue,citecolor=brown,linktocpage=true,pagebackref=true,hyperindex=true}
\pagenumbering{arabic}

\DeclareMathOperator{\op}{op}
\DeclareMathOperator{\pt}{pt}
\DeclareMathOperator{\spec}{spec}
\DeclareMathOperator{\Fit}{Fit}
\DeclareMathOperator{\Pth}{P}
\DeclareMathOperator{\Frm}{Frm}
\DeclareMathOperator{\Obj}{Obj}
\DeclareMathOperator{\Hom}{Hom}

\title{Modificaciones de parches}
\subtitle{y su relación con los axiomas tipo Hausdorff}
\event{57° Congreso Nacional de la SMM}
\date{Herramientas reticulares y categóricas para el estudio de anillos y módulos}
\author{Juan Carlos Monter Cortés\\ Luis Ángel Zaldívar Corichi}
\institute{Universidad de Guadalajara}
\email{juan.monter2902@alumnos.udg.mx\\ $\quad$ luis.zaldivar@academicos.udg.mx}

%\homepage{www.mywebsite.com}
%\github{username}

\begin{document}

\frame[plain]{\titlepage}

\begin{frame}{Contenido}
\tableofcontents %Imprime la tabla de contenido
\end{frame}

\section{Antecedentes}

\begin{frame}{El parche original}
    \begin{theorem}
        Sea $S$ un espacio coherente con topología $\mathcal{O}S$. Consideremos 
        \[
        A=\{U\in \mathcal{O}S\mid U \mbox{ compacto}\}\quad\mbox{ y }
        \]
        \[
        A^*=\{V\in\mathcal{P}S\mid V \mbox{ es complemento de }U \mbox{ y }U\in A\}
        \]
        Sea $\mathcal{O}'S$ la topología en $S$ que tiene como base al conjunto
        \[
        C=\{U\cap V\mid U\in A, V\in A^*\}.
        \]
        Entonces $(S, \mathcal{O}'S)$ es un espacio de Stone.
    \end{theorem}
    
\end{frame}

\begin{frame}[fragile]{La adjunción entre $\mathbf{Frm}$ y $\mathbf{Top}$}
\begin{itemize}
    \item $\mathcal{O}(\_)\colon \mathbf{Top}\to \mathbf{Frm}$ 
    \item $\pt(\_)\colon \mathbf{Frm}\to \mathbf{Top}$
    \item El funtor $\pt$ es adjunto izquierdo de $\mathcal{O}$.
\end{itemize}
\[\begin{tikzcd}
	{\mathbf{Top}} \\
	{} & {} & {} \\
	{\mathbf{Frm}}
	\arrow["{\pt}"{name=0, swap}, from=3-1, to=1-1, shift right=5]
	\arrow["{\mathcal{O}}"{name=1, swap}, from=1-1, to=3-1, shift right=5]
	\arrow["\dashv"{rotate=0}, from=1, to=0, phantom]
\end{tikzcd}\]
    
\end{frame}

\begin{frame}[fragile]
      \frametitle{Axiomas de separación}
      \framesubtitle{(sensibles a puntos)}
      En la topología clásica (topología sensible a puntos), tenemos la siguiente relación entre los axiomas de separación 

    \[\begin{tikzcd}
	{T_4} & {T_{3\frac{1}{2}}} & {T_3} & {T_2} & {T_1} & {T_0} \\
	& {(norm)} & {R_3} & {R_2} & {R_1} & {R_0}
	\arrow[Rightarrow, from=1-2, to=1-3]
	\arrow[Rightarrow, from=1-3, to=1-4]
	\arrow[Rightarrow, from=1-4, to=1-5]
	\arrow[Rightarrow, from=1-5, to=1-6]
	\arrow[Rightarrow, from=1-1, to=1-2]
	\arrow[Rightarrow, from=1-1, to=2-2]
	\arrow[Rightarrow, from=1-2, to=2-3]
	\arrow[Rightarrow, from=1-3, to=2-4]
	\arrow[Rightarrow, from=1-4, to=2-5]
	\arrow[Rightarrow, from=1-5, to=2-6]
	\arrow[Rightarrow, from=2-3, to=2-4]
	\arrow[Rightarrow, from=2-4, to=2-5]
	\arrow[Rightarrow, from=2-5, to=2-6]
\end{tikzcd}\]
\[
    \mbox{ }
\]
    
donde $R_2=\mathbf{(reg)}$, $R_3=\mathbf{(creg)}$, $T_i=R_{i-1}+T_{i-1}$ y $T_{3\frac{1}{2}}=R_3+T_3$, para $i=1, 2, 3$.\\

$T_4=\mathbf{(norm)}+T_1$.
    
\end{frame}

\begin{frame}{Axiomas de separación}
\framesubtitle{(libres de puntos)}
    \begin{description}
    \item[$(T_1)$] Todo elemento primo es máximo.
    \item<2->[$(reg)$] $\quad\forall\, U\in \mathcal{O}S$
    \[
    U=\bigcup\{V\mid V\prec U\}
    \]
    \item<3->[$(T_3)$] $=(reg)+(T_{1})$ 
    \item<4->[$(creg)$] $\quad\forall\, U\in \mathcal{O}S$
    \[
    U=\bigcup\{V\mid V\prec\prec U\}
    \]
    \item<5->[$(T_{3\frac{1}{2}})$] $=(creg)+(T_{1})$ 
    \item<6->[$(norm)$] $\quad\forall\, X, Y \in \mathcal{O}S$ tales que $X\cup Y=S$, $\exists$ $U, V \in \mathcal{O}S$ tales que
    \[
    X\cup U=S, \quad Y\cup V=S, \quad U\cap V=\emptyset
    \]
    \item<7->[$(T_{4})$] $=(norm)+(T_{1})$ 
\end{description}
\end{frame}

\begin{frame}{Axiomas de separación en $\mathbf{Frm}$}
Qué pasa si $A$ es un marco arbitrario? 
\begin{tiny}
$$\mbox{ }$$
\end{tiny}

\uncover<2->{\begin{description}
    \item[$(reg)$] $\quad \forall \,a\in A$, $a=\bigvee\{x\in A\mid x\prec a\}$
    \item[$(creg)$] $\quad\forall \,a\in A$, $a=\bigvee\{x\in A\mid x\prec\prec a\}$
    \item[$(norm)$] $\quad\forall \,a, b\in A$ tales que $a\vee b=1$, $\exists\; u, v\in A$ tales que 
    \[
    a\vee u=b\vee v=1 \qquad \mbox{ y }\qquad u\wedge v=0
    \]
\end{description}}
    
\end{frame}

\section{Modificación de parches}
\begin{frame}{Espacio de parches}
    \[
\mbox{Si } S\mbox{ es }T_2\Rightarrow \mbox{ todo conjunto compacto (saturado) es cerrado.}
\]

\uncover<2->{\begin{definition}
    Un espacio topológico $S$ es \emph{empaquetado} si todo conjunto compacto saturado es cerrado
\end{definition}}

\uncover<3->{\begin{lemma}
    Un espacio topológico que es $T_0$ y empaquetado es $T_1$.
\end{lemma}}

\uncover<4->{\[
T_2\Rightarrow T_0+\mbox{ empaquetado }\Rightarrow T_1
\]}
\end{frame}

\begin{frame}[plain]
     Si $Q\in \mathcal{Q}S$ y $Q\notin \mathcal{C}S \Rightarrow S$ no es empaquetado.

    \onslide<2->{\[
    \mbox{pbase}=\{U\cap Q'\mid U\in \mathcal{O}S, \; Q\in \mathcal{Q}S\}
    \]}

    \uncover<3->{\begin{definition}
        Para un espacio topológico $S$, sea $^pS$ (\emph{espacio de parches}), el espacio con los mismos puntos que $S$ y la topología $\mathcal{O}^pS$ generada por la pbase.
    \end{definition}}

    \uncover<4->{\[
    S \mbox{ es empaquetado }\quad \Leftrightarrow\quad  ^pS=S
    \]}

    \uncover<5->{\begin{lemma}
        El espacio de parches de un espacio $T_2$ es él mismo.
    \end{lemma}}
\end{frame}

\begin{frame}{Según Rosemary}
    Basados en el \emph{Teorema de Hoffmann-Mislove}, se introduce un nuevo marco de parches.

\[\text{p-base}(A) =\{u_{a}\wedge v_{F}\mid F\in A^{\wedge}\}.\]

\uncover<2->{Sea $P(A)=\langle\text{p-base}(A)\rangle$, es decir, tomamos supremos arbitrarios de elementos en $\text{p-base}(A)$

 \[\xymatrix{ A\ar[r]^{p}\ar@/^/@<+1.5ex>[rr]^{\eta_{A}} & P(A)\ar[r]^{\iota} & NA }\]
}

\uncover<3->{
\[
¿\mbox{Cuándo } A\cong P(A)?
\]}
\end{frame}

\begin{frame}
\frametitle{Marcos $\alpha$-arreglados}

\begin{theorem}
    Supongamos que $A$ es un marco regular y sea $j\in NA$ tal que $\bigtriangledown (j)$ es abierto. Entonces $j=u_d$, donde $d=j(0)$
\end{theorem}
\uncover<2->{\[
  A \mbox{ es regular } \Rightarrow A\cong P(A)
\]}

\uncover<3->{\begin{definition}
    $A\in \mathbf{Frm}$ es parche trivial si el encaje $A\to P(A)$ es un isomorfismo.
\end{definition}}
\end{frame}

\begin{frame}
\begin{definition}
Sea $A\in \mathbf{Frm}$ y $\alpha$ un ordinal, un filtro abierto $F$ en $A$ es $\alpha-$\emph{arreglado} si \[x\in F\Rightarrow u_{d(\alpha)}(x)=d(\alpha)\vee x=1,\] donde $d(\alpha)=f^{\alpha}(0)$ y $f=\dot\bigvee\{v_a\mid a\in F\}$.
\end{definition}

\onslide<2->{Un marco es \emph{parche trivial} si y solo si es arreglado (donde arreglado quiere decir que es $\alpha$-arreglado para algún ordinal $\alpha$).}

\onslide<3->\begin{theorem}
Sea $S$ un espacio $T_{0}$, éste tiene \alert<4->{marco de abiertos 1-arreglado si y solo si $S$ es $T_2$}.
\end{theorem}
\end{frame}

\section{Axiomas tipo Hausdorff}
\begin{frame}{Propiedad conservativa, de 1° orden y de 2° orden}

\begin{definition}
    \begin{enumerate}
        \item Para un espacio $S$ decimos que una propiedad $P$ es \emph{conservativa} si y solo si $\mathcal{O}S$ tiene la propiedad $P_S$.

        \item<2-> Decimos que una propiedad en marcos $P$ es \emph{suficientemente Hausdorff} si y solo si $P$ implica la propiedad Hausdorff espacial.
        
        \item<3-> Decimos que una propiedad en marcos $P$ es de \emph{1° orden} si y solo si $P$ es enunciada como una fórmula para elementos del marco.

        \item<4-> Decimos que una propiedad en marcos $P$ es de \emph{2° orden} si y solo si $P$ es enunciada como una caracterización de sublocales.
    \end{enumerate}
\end{definition}

\end{frame}

\begin{frame}{Axiomas tipo Hausdorff}
    \begin{description}
        \item[$(\mathbf{dH})$] $\quad a\vee b=1$ y $a, b\neq 1$, $\exists$ $u, v$ tales que $u\nleq a$, $v\nleq b$ y $u\wedge v=0$. 
        \item<2->[$(\mathbf{H})$] $\quad 1\neq a\nleq b\in L$, $\exists$ $u, v\in L$ tales que $u\nleq a$, $v\nleq b$ y $u\wedge v=0$. 
        \item<3->[$(\mathbf{Hp})$] $\quad$Cada elemento semiprimo en $L$ es máximo.
        \item<4->{[$(\mathbf{fH})$] $\quad$El sublocal diagonal $\Delta[L]$ es cerrado en $L\oplus L$.
        \[
        \Leftrightarrow \Delta[L]=\uparrow d_L
        \]
        donde $d_L$ es el menor elemento de $\Delta[L]$, es decir,
\[
d_L=\Delta(0)=\{(x, y)\mid x\wedge y\leq 0\}=\downarrow\{(x, x^*)\mid x\in L\}.
\]}
    \end{description}
\end{frame}

\begin{frame}{Jerarquía y comportamiento}
    \[
    \mathbf{(fH)}\Rightarrow \mathbf{(H)}\Rightarrow \mathbf{(dH)} \quad\mbox{ y }  \quad
    \mathbf{(fH)}\Rightarrow \mathbf{(H)} \Rightarrow \mathbf{(Hp)}
    \]
\begin{tiny}
$$\mbox{ }$$
\end{tiny}

\begin{center}
\begin{tabular}{| c | c | c | c | c | c |}
\hline
 Axioma/Comportamiento & \textbf{C.} & 1° & 2° & \textbf{S. H.} & \textbf{C. S. E.}\\ \hline
$\mathbf{(dH)}$ & x & $\checkmark$ & x & x & x \\ \hline
$\mathbf{(H)}$ & $\checkmark$ & $\checkmark$ & x & $\checkmark$ & x \\ \hline
$\mathbf{(Hp)}$ & $\checkmark$ & $\checkmark$ & x & $\checkmark$ & ? \\ \hline
$\mathbf{(fH)}$ & x & x &  $\checkmark$ & $\checkmark$ & $\checkmark$ \\ \hline
\end{tabular}
\end{center}
\begin{tiny}
$$\mbox{ }$$
\end{tiny}

\textbf{C.}= Propiedad conservativa\\
\textbf{S. H.}= Suficientemente Huasdorff\\
\textbf{C. S. E.}= Comportamiento similar al espacial


%\begin{definition}
    
%\end{definition}

%\begin{corollary}
    
%\end{corollary}
\end{frame}

\begin{frame}[fragile, plain]
    \[\begin{tikzcd}
	&& {\mathbf{(reg)}} \\
	{\mathbf{(aju)}} && {\mathbf{(fH)}} && {\mathbf{(H)}+\mathbf{(saju)}} \\
	{\mathbf{(saju)}} && {T_U} && {\mathbf{(H)}} \\
	&& {T_1}
	\arrow[Rightarrow, from=1-3, to=2-1]
	\arrow[Rightarrow, from=1-3, to=2-3]
	\arrow[Rightarrow, from=1-3, to=2-5]
	\arrow[dotted, no head, from=2-1, to=2-3]
	\arrow[Rightarrow, from=2-1, to=3-1]
	\arrow[Rightarrow, from=2-1, to=3-3]
	\arrow[dotted, no head, from=2-1, to=3-5]
	\arrow[dotted, no head, from=2-3, to=2-5]
	\arrow[dotted, no head, from=2-3, to=3-1]
	\arrow[Rightarrow, from=2-3, to=3-3]
	\arrow[Rightarrow, from=2-3, to=3-5]
	\arrow[Rightarrow, from=2-5, to=3-1]
	\arrow[dotted, no head, from=2-5, to=3-3]
	\arrow[Rightarrow, from=2-5, to=3-5]
	\arrow[dotted, no head, from=3-1, to=3-3]
	\arrow[dotted, no head, from=3-1, to=4-3]
	\arrow[dotted, no head, from=3-3, to=3-5]
	\arrow[Rightarrow, from=3-3, to=4-3]
	\arrow[Rightarrow, from=3-5, to=4-3]
\end{tikzcd}\]

\[\begin{tikzcd}
	&&&& {\mathbf{(H)}} \\
	{\mathbf{(reg)}} && {\mathbf{(fH)}} && {(T_U)} && {(T_1)}
	\arrow[Rightarrow, from=2-1, to=2-3]
	\arrow[Rightarrow, from=2-3, to=1-5]
	\arrow[Rightarrow, from=2-3, to=2-5]
	\arrow[Rightarrow, from=2-5, to=2-7]
\end{tikzcd}\]
\end{frame}

\begin{frame}[plain, fragile]
\[\begin{tikzcd}
	&& {\mathbf{(saju)}} \\
	{\mathbf{(reg)}} & {\mathbf{(H)}+\mathbf{(saju)}} & {\mathbf{(H)}} & {T_1}
	\arrow[Rightarrow, from=2-1, to=2-2]
	\arrow[Rightarrow, from=2-2, to=1-3]
	\arrow[Rightarrow, from=2-2, to=2-3]
	\arrow[Rightarrow, from=2-3, to=2-4]
\end{tikzcd}\]

\[\begin{tikzcd}
	&&&& {\mathbf{(saju)}} \\
	{\mathbf{(reg)}} && {\mathbf{(aju)}} && {(T_1)} \\
	&&&& {(T_U)}
	\arrow[Rightarrow, from=2-1, to=2-3]
	\arrow[Rightarrow, from=2-3, to=1-5]
	\arrow[Rightarrow, from=2-3, to=2-5]
	\arrow[Rightarrow, from=2-3, to=3-5]
\end{tikzcd}\]

    
\end{frame}

\section{Marcos arreglados vs axiomas tipo Hausdorff}

\begin{frame}
\frametitle{Algunos resultados}
\begin{definition}
    Decimos que un marco $A$ es espacial si $A=\mathcal{O}S$, para $S$ un espacio topológico.
\end{definition}

\uncover<2->{\begin{theorem}
    Si $A$ es un marco espacial entonces 
    \[
    A \text{ es } 1\text{-arreglado si y solo si } S \text{ es } T_2
    \] 
\end{theorem}
 }
\end{frame}

\begin{frame}[plain]
    \begin{itemize}
        \item Si $S$ es $T_2$ $\Rightarrow$ $\mathcal{O}S$ es parche trivial
        \item<2-> Un marco $A$ es arreglado $\Leftrightarrow$ A parche trivial.
        \item<3-> $\mathbf{(H)}$ es conservativa
    \end{itemize}

    \uncover<4->{\[
\mathcal{O}S \mbox{ es }\mathbf{(H)}\Leftrightarrow S\mbox{ es }T_2\Rightarrow \mathcal{O}S \mbox{ parche trivial }\Leftrightarrow \mathcal{O}S \mbox{ arreglado}
\]}

\uncover<5->{\begin{block}{Conjetura}
    Todo marco Hausdorff es 1-arreglado.
\end{block}}
\end{frame}

\begin{frame}[plain, fragile]
    Buscamos tener nuevas caracterizaciones de arreglo basadas en las nociones tipo Hausdorff para marcos. Por ejemplo:

\begin{theorem}
        Para $A$ un marco espacial, $\mathcal{O}S$ es un marco Hausdorff si y solo si $A$ es $1-$arreglado.
    \end{theorem}

\begin{theorem}
    Todo marco fuertemente Hausdorff es arreglado.
\end{theorem}

\[\begin{tikzcd}
	{\mathbf{(reg)}} && {\mathbf{(fH)}} && {\mathbf{(H)}} \\
	\\
	&& {\mathbf{Arreglado}}
	\arrow[Rightarrow, from=1-1, to=1-3]
	\arrow[Rightarrow, from=1-1, to=3-3]
	\arrow[Rightarrow, from=1-3, to=1-5]
	\arrow[Rightarrow, from=1-3, to=3-3]
	\arrow[Rightarrow, dashed, from=1-5, to=3-3]
\end{tikzcd}\]
\end{frame}

%\begin{frame}[bg=demo-arguelles.png]
 %     \frametitle{A frame with background image}
  %    You can still add title and subtitle. \par
   %   You can also use a background in the title slide by setting: \\
    %  \texttt{\textbackslash frame[plain,bg=demo-background.jpg]\{\textbackslash titlepage\}}
%\end{frame}

%\begin{frame}[plain]
 %     \frametitle{A plain frame has no headline}
  %    \begin{table}
   %         \small
    %        \begin{tabular}{rl}
    %              \ttfamily\textbackslash Alegreya              & \Alegreya Lorem ipsum dolor sit amet              \\
     %             \ttfamily\textbackslash AlegreyaMedium        & \AlegreyaMedium Lorem ipsum dolor sit amet        \\
     %             \ttfamily\textbackslash AlegreyaExtraBold     & \AlegreyaExtraBold Lorem ipsum dolor sit amet     \\
     %             \ttfamily\textbackslash AlegreyaBlack         & \AlegreyaBlack Lorem ipsum dolor sit amet         \\
     %             \ttfamily\textbackslash AlegreyaSansThin      & \AlegreyaSansThin Lorem ipsum dolor sit amet      \\
 %                 \ttfamily\textbackslash AlegreyaSansLight     & \AlegreyaSansLight Lorem ipsum dolor sit amet     \\
  %                \ttfamily\textbackslash AlegreyaSans          & \AlegreyaSans Lorem ipsum dolor sit amet          \\
   %               \ttfamily\textbackslash AlegreyaSansMedium    & \AlegreyaSansMedium Lorem ipsum dolor sit amet    \\
%                  \ttfamily\textbackslash AlegreyaSansExtraBold & \AlegreyaSansExtraBold Lorem ipsum dolor sit amet \\
 %                 \ttfamily\textbackslash AlegreyaSansBlack     & \AlegreyaSansBlack Lorem ipsum dolor sit amet
  %          \end{tabular}
   %   \end{table}
    %  \vfill
     % \begin{alert}{Alert!}
%            A \textit{plain} frame does not show the progress bar but still appears in it unless the frame comes after \texttt{\textbackslash End}
 %     \end{alert}
%\end{frame}

\begin{frame}[standout]
Lo que viene...
\begin{itemize}
    \item<2-> Si $A$ es un marco arreglado $\Rightarrow$ $A_j$ es arreglado.
    \item<3-> Si $\{A_i\}_{i\in \mathcal{I}}$ marcos arreglados $\Rightarrow$ $\prod A_i$ es arreglado.
    \item<4-> Si $S$ es un sublocal compacto de un local Hausdorff $\Rightarrow$ $S$ es cerrado.
    \item<5-> Caracterizar marcos que no son arreglados.
    \item<6-> Que propiedades de separación cumple $PA$.
    \item<7-> $\vdots$
\end{itemize}
     % \centering\large
      %A \textbf{\itshape\scshape standout} frame can be used to focus attention
\end{frame}

\begin{frame}[plain,standout]
      \centering
      \Huge{\smiley Muchas gracias por su atención\smiley}
      %In combination with \textit{plain}, \\
      %it makes a nice thank-you slide!
      %\vfill
      %\scalebox{4}{\faGithub} \par\bigskip
      %\url{https://github.com/piazzai/arguelles} \\
      %\url{https://ctan.org/pkg/beamertheme-arguelles}
\end{frame}

\End

\section*{\textsc{Referencias}}
\begin{frame}[allowframebreaks]
\frametitle{Referencias}
\begin{thebibliography}{20}\markboth{Bibliografía}{Bibliografía}

\bibitem{P.T.} P. T. Johnstone, \textit{Stone spaces}, Cambridge Studies in Advanced Mathematics, vol. 3, Cambridge University Press, Cambridge, 1982. MR 698074


\bibitem{J.M.} J. Monter; A. Zaldívar, \textit{El enfoque locálico de las reflexiones booleanas: un análisis en la categoría de marcos} [tesis de maestría], 2022. Universidad de Guadalajara.

\bibitem{J.P.} J. Picado and A. Pultr, \textit{Frames and locales: Topology without points}, Frontiers in Mathematics, Springer Basel, 2012.

\bibitem{J.P.2} J. Picado and A. Pultr, \textit{Separation in point-free topology}, Springer, 2021.

\bibitem{R.S.} Rosemary A Sexton, \textit{A point free and point-sensitive analysis of the patch assembly}, The University of Manchester (United Kingdom), 2003.

\bibitem{H.S.3} Harold Simmons, \textit{The assembly of a frame}, University of Manchester (2006).

\bibitem{R.S.3} RA Sexton and H. Simmons, \textit{Point-sensitive and point-free patch constructions}, Journal of Pure and Applied Algebra \textbf{207} (2006), no. 2, 433-468.

\bibitem{A.Z.} A. Zaldívar, \textit{Introducción a la teoría de marcos} [notas curso], 2024. Universidad de Guadalajara.

\end{thebibliography}
\end{frame}

\end{document}
