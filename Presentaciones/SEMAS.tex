\documentclass[compress,12pt]{beamer}

\usetheme{Arguelles}
\usepackage{graphicx}
\usepackage{caption}
%\usepackage[spanish,es-noshorthands]{babel}
\usepackage[spanish]{babel} 
\usepackage[pages=some]{background}
\usepackage{tikz-cd}
\usepackage{amsmath,amssymb,latexsym,amscd} 
\usepackage[all,cmtip]{xy}
\usepackage{fancyhdr}
\usepackage{mathalfa}
\usepackage{mathrsfs}
\usetikzlibrary{babel,positioning,calc,arrows.meta}
\usepackage{hyperref}
\usepackage{ragged2e}
\usepackage{wasysym}
\usepackage{multicol}
%\hypersetup{colorlinks=true,linkcolor=blue,citecolor=brown,linktocpage=true,pagebackref=true,hyperindex=true}
\pagenumbering{arabic}
\usepackage{tabularx}
\usepackage{booktabs}
\usepackage{array}

\newcommand{\den}[1]{[\![#1]\!]} % denotación sin stmaryrd

% Columnas: L = texto que se ajusta (izquierda), C = columna de notación centrada
\newcolumntype{L}{>{\raggedright\arraybackslash}X}
\newcolumntype{C}{>{\centering\arraybackslash}m{2.4cm}}

% Encabezado reutilizable
\newcommand{\TablaNucleosHead}{%
  \toprule
  \multicolumn{1}{c}{\textbf{Noción locálica}\\\textbf{(geométrica)}} &
  \textbf{Notación} &
  \multicolumn{1}{c}{\textbf{Correspondencia en núcleos}\\\textbf{(algebraica)}}\\
  \midrule
}


% Si quieres que todas las imágenes tengan la misma altura:
\newcommand{\personimg}[2]{%
  \begin{minipage}[t]{0.32\textwidth}
    \centering
    \includegraphics[height=3.2cm,keepaspectratio]{#1}\\[-2pt]
    {\footnotesize #2}
  \end{minipage}
}

% Variante para 4 imágenes (2x2)
\newcommand{\personimgsmall}[2]{%
  \begin{minipage}[t]{0.48\textwidth}
    \centering
    \includegraphics[height=3.0cm,keepaspectratio]{#1}\\[-2pt]
    {\footnotesize #2}
  \end{minipage}
}

\DeclareMathOperator{\op}{op}
\DeclareMathOperator{\pt}{pt}
\DeclareMathOperator{\spec}{spec}
\DeclareMathOperator{\Fit}{Fit}
\DeclareMathOperator{\Pth}{P}
\DeclareMathOperator{\Frm}{Frm}
\DeclareMathOperator{\Top}{Top}
\DeclareMathOperator{\Obj}{Obj}
\DeclareMathOperator{\Hom}{Hom}

%%Se define el "environment" teorema
\newtheorem{thm}{Teorema}
\newtheorem{dfn}{Definición}
\newtheorem*{dfn*}{Definición}
\newtheorem{lem}{Lema}
\newtheorem{cor}{Corolario}
\newtheorem{prop}{Proposición}
\newtheorem{obs}{Observación}
\newtheorem{ej}{Ejemplo}

\title{¿Para qué estudiar topología sin puntos y cómo hacerlo?}
\date{\today}
\author{Juan Carlos Monter Cortés}
\institute{Universidad de Guadalajara}
\email{juan.monter2902@alumnos.udg.mx}
\event{Seminario de Estudiantes en Matemáticas (SEMAS), CUCEI}

%\homepage{www.mywebsite.com}
\github{JCmonter}

\begin{document}

\frame[plain]{\titlepage}

\begin{frame}{Contenido}
\tableofcontents %Imprime la tabla de contenido
\end{frame}

\section{¿Para qué estudiar topología sin puntos?}

\begin{frame}{Nacimiento de la topología clásica}
\centering

% --- Fila de 3 imágenes ---
\personimg{Imagenes/gauss.jpg}{Gauss}\hfill
\personimg{Imagenes/Poincare.png}{Poincar\'e}\hfill
\personimg{Imagenes/Hausdorff.jpg}{Hausdorff}

\vspace{0.6cm}

\begin{itemize}\justifying
  \item Surge del estudio geométrico y analítico de deformaciones continuas.
  \item Se formaliza la noción de \emph{espacio topológico}: puntos + abiertos.
\end{itemize}

\vspace{0.1cm}
\onslide<2->{\centering{\small\itshape La topología clásica estudia espacios como conjuntos de puntos organizados por abiertos.}}
\end{frame}

\begin{frame}{Nacimiento de la topología ``sin puntos''}
\centering

\personimg{Imagenes/Isbell.png}{Isbell}\hfill
\personimg{Imagenes/Tarski.jpg}{Tarski}\hfill
\personimg{Imagenes/Mstone.jpg}{Stone}

\vspace{0.6cm}

\begin{itemize}\justifying
  \item Se observa que los abiertos forman estructuras algebraicas ricas.
  \item Comienza el enfoque: estudiar “espacios” a través de la estructura de abiertos.
\end{itemize}

\vspace{0.1cm}
\onslide<2->{\centering{\small\itshape La topología puede desarrollarse sin mencionar puntos explícitamente.}}
\end{frame}

\begin{frame}{Actualidad de la topología sin puntos}
\centering

\personimg{Imagenes/Simmons.png}{Simmons}\hfill
\personimg{Imagenes/Johnstone.jpg}{Johnstone}\hfill
\personimg{Imagenes/Picado.jpg}{Picado}

\vspace{0.55cm}

\begin{itemize}\justifying
  \item Teoría consolidada: libros, cursos y herramientas modernas.
  \item Investigación activa y conexiones con lógica, computación y geometría.
\end{itemize}

\vspace{0.1cm}
\onslide<2->{\centering{\small\itshape La topología sin puntos es hoy un área viva de investigación.}}
\end{frame}

\begin{frame}{Topología sin puntos en México}
\centering

\personimg{Imagenes/MITAC.jpeg}{MITAC 2024}\hfill
\personimg{Imagenes/FandL.jpg}\hfill
\personimg{Imagenes/Congreso.jpg}{58 Congreso SMM}

\vspace{0.55cm}

\begin{itemize}\justifying
    \item En CUCEI se estudian axiomas de separación, construcciones de parches, aspectos categóricos de la teoría de marcos,
    entre otras cosas.
\end{itemize}

\vspace{0.1cm}
\onslide<2->{\centering{\normalsize\itshape \alert{¿Consideras que en algún momento has realizado topología sin puntos?}}}
\end{frame}

\begin{frame}{Topología, continuidad y axiomas de separación}
 \begin{block}{Definición:}
    Una \textbf{topología} sobre un conjunto $S$ es una familia $\tau \subseteq \mathcal{P}(S)$ tal que:
    \begin{itemize}
      \item $\varnothing, S \in \tau$.
      \item Si $U,V \in \tau$, entonces $U \cap V \in \tau$.
      \item Si $\{U_i\}_{i \in I} \subseteq \tau$, entonces $\bigcup_{i \in I} U_i \in \tau$.
    \end{itemize}
  \end{block}
  \onslide<2->{Al par $(S, \tau)$ se le llama \textbf{espacio topológico}.} \onslide<3->{De hecho,
  \[
  (\tau, \subseteq, \bigcup, \cap, \varnothing, X)
  \] 
  es un \alert<3->{marco}.}
\end{frame} 

\begin{frame}{Primer encuentro con la continuidad}
  \textbf{En cálculo:} $f:\mathbb{R}\to\mathbb{R}$.

  \begin{block}{Definición $\varepsilon$–$\delta$}
    $f$ es continua en $a$ si
    \[
      \forall \varepsilon>0 \;\; \exists \delta>0 \;\; 
      |x-a|<\delta \;\Rightarrow\; |f(x)-f(a)|<\varepsilon.
    \]
  \end{block}

  \uncover<2->{\centering
    \textcolor{red!70!black}{\Large ``Debemos exhibir los elementos $\varepsilon$-$\delta$ de la definición''}
  }
\end{frame}

\begin{frame}{Continuidad en espacios métricos}
  \textbf{En análisis matemático:} $f:(X,d_1)\to(Y,d_2)$.

  \begin{block}{Definición $\varepsilon$–$\delta$ (versión métrica)}
    $f$ es continua en $a$ si
    \[
      \forall \varepsilon>0 \;\; \exists \delta>0 \;\; 
      d_1(x,a)<\delta \;\Rightarrow\; d_2(f(x),f(a))<\varepsilon.
    \]
  \end{block}

  \uncover<2->{\centering
    \textcolor{red!70!black}{\Large ``La idea es la misma y sigue siendo técnica''}
  }
\end{frame}

\begin{frame}{Caracterización topológica}
  \begin{block}{Teorema}
    $f:(X,d_1)\to(Y,d_2)$ es continua si y solo si 
    \[
      \forall\; V \;\text{abierto en } Y, \quad f^{-1}(V)\;\text{es abierto en } X.
    \]
  \end{block}

  \uncover<2->{\centering
    \textcolor{blue!60!black}{\Large ``Una condición global que no menciona puntos''}
  }

  \vspace{1em}

  \uncover<3->{\centering
    \textbf{¿De qué manera prefieres demostrar continuidad?}}
\end{frame}

\begin{frame}{Axiomas de separación en Top}
  \begin{itemize}
    \item \textbf{T$_0$:} para $x\neq y$, existe abierto que contiene a uno y no al otro.
    \item \textbf{T$_1$:} para $x\neq y$, hay abierto que contiene a $x$ pero no a $y$ \emph{y} viceversa).
    \item \textbf{T$_2$:} para $x\neq y$, existen abiertos disjuntos $U,V$ con $x\in U$, $y\in V$.
    \item \textbf{Regular:} para todo $x\in X$ y todo cerrado $F\subseteq X$
        con $x\notin F$, existen abiertos disjuntos $U,V$ tales que
        $x\in U$ y $F\subseteq V$.
     \item \textbf{Normal:} para cualesquiera cerrados disjuntos
        $F,G\subseteq X$, existen abiertos disjuntos $U,V$ con
        $F\subseteq U$ y $G\subseteq V$.
  \end{itemize}
  \vspace{0.6em}
  \centering
  \onslide<2->{\textcolor{blue!60!black}{\large¿Cómo se leen estas ideas \emph{sin puntos}?}}
\end{frame}

\section{¿Qué es la topología sin puntos?}

\begin{frame}{¿Por qué topología sin puntos?}
  \centering
  \begin{columns}[c]
    % --- Punto ---
    \column{0.32\textwidth}
      \only<1->{
        \begin{tikzpicture}
          \fill (0,0) circle (3pt); % un solo punto
        \end{tikzpicture}\\[0.3em]
        \scriptsize Punto
      }

    % --- Recta ---
    \column{0.32\textwidth}
      \only<2->{
        \begin{tikzpicture}
          \draw[-] (-1,0) -- (1,0); % línea recta
        \end{tikzpicture}\\[0.3em]
        \scriptsize Recta
      }

    % --- Abierto (círculo más grande con puntos dentro) ---
    \column{0.32\textwidth}
      \only<3->{
        \begin{tikzpicture}
          % círculo más grande y relleno semitransparente
          \fill[blue!20, opacity=0.3] (0,0) circle (1.3);
          \draw[blue, thick] (0,0) circle (1.3);

          % puntos dentro (animados)
          \only<4->{\fill (0.5,0.4) circle (2.5pt);}
          \only<5->{\fill (-0.5,-0.4) circle (2.5pt);}
        \end{tikzpicture}\\[0.3em]
        \scriptsize Abierto
      }
  \end{columns}
\end{frame}

\begin{frame}{¿Qué es la topología sin puntos?}
  \begin{itemize}
    \item<1-> En topología clásica describimos un espacio como un conjunto de puntos $S$ junto con una familia de abiertos $\tau$.
    \item<2-> En la topología sin puntos nos olvidamos de $S$ y trabajamos únicamente con la estructura de abiertos.
    \item<3-> El enfoque anterior se trabaja por medio de la \textbf{teoría de marcos}.
  \end{itemize}
  \vspace{1em}
  \uncover<4->{\centering
    \textcolor{blue!60!black}{\Large ``Lo importante no son los puntos, sino cómo se relacionan los abiertos''}
  }
\end{frame}

\begin{frame}{¿Qué es un marco?}
  \begin{columns}[T,totalwidth=\textwidth]
    % --- Columna izquierda ---
    \begin{column}{0.45\textwidth}
      \begin{itemize}
        \item<1-> $A$
        \item<2-> $(A, \leq)$
      \end{itemize}
    \end{column}

    % --- Columna derecha ---
    \begin{column}{0.55\textwidth}
      \begin{itemize}
        \item<3-> $(A, \leq, \vee, 0)$ o $(A, \leq, \wedge, 1)$
        \item<4-> $(A, \leq, \bigvee, \bigwedge, 0, 1)$
      \end{itemize}
    \end{column}
  \end{columns}

  \vspace{1em}

  \onslide<5->\begin{block}{Definición:}
    Un \textbf{marco} es una retícula completa, $(A, \leq, \bigvee, \wedge, 0, 1)$,
    en la que los ínfimos finitos se distribuyen sobre supremos arbitrarios:
    \begin{equation}\label{LDM}
      a \wedge \bigvee X =\bigvee\{a\wedge x\mid x\in X\}\quad \mbox{donde }a\in A\text{ y } X\subseteq A.
    \end{equation}
  \end{block}
\onslide<6->{A (\ref{LDM}) se le conoce como \emph{Ley Distributiva de marcos (LDM)}.}
\end{frame}

\begin{frame}{Algunos aspectos adicionales}
  \begin{itemize}
    \item<1-> \textbf{Morfismos de marcos:} funciones que preservan la estructura de la retícula.
    \item<2-> \textbf{Categoría Frm:} marcos como objetos y morfismos  de marcos como flechas.
    \item<3-> \textbf{Funtorialidad:} podemos relacionar objetos y flechas de $\Frm$ con otras categorías 
    o subcategorías (y viceversa).
    \item<4-> \textbf{Adjunciones:} existen correspondencias biyectivas y equivalencias.
    \item<5-> \textbf{Generalización:} una clase de marcos pueden verse como generalizaciones de los 
    subespacios.
  \end{itemize}

    \vspace{1em}
  \uncover<6->{\centering
    \textcolor{blue!60!black}{\large ``Detrás de la idea intuitiva hay una teoría categórica robusta''}
  }
\end{frame}

\begin{frame}{Conexión entre Top y Frm}

\small

%--- Definición de categorías ---
\[
\begin{array}{rcl}
\Top & = &
\left\{
\begin{array}{l}
\text{Objetos: espacios topológicos} \\
\text{Flechas: funciones continuas}
\end{array}
\right. \\[0.6cm]
\Frm & = &
\left\{
\begin{array}{l}
\text{Objetos: marcos} \\
\text{Flechas: morfismos de marcos}
\end{array}
\right.
\end{array}
\]

\vspace{0.2cm}

%--- Diagrama ---
\begin{center}
\begin{tikzpicture}[node distance=4cm, auto]
\node (Top) {\Large $\Top$};
\node (Frm) [right of=Top] {\Large $\Frm$};

\draw[->] (Top) to[bend left] node[above] {$\mathcal{O}(\_)$} (Frm);
\draw[->] (Frm) to[bend left] node[below] {$\pt(\_)$} (Top);
\end{tikzpicture}
\end{center}
\end{frame}

\begin{frame}{Los funtores $\mathcal{O}(\_)$ y $\pt(\_)$}
\small

\begin{columns}[T,onlytextwidth]

% -------------------------
% Columna: O(_)
% -------------------------
\column{0.49\textwidth}
\begin{center}
\textbf{El funtor $\mathcal{O}(\_):\Top \to \Frm$}
\end{center}

\vspace{-0.2cm}

\[
\begin{array}{ccl}
S & \longmapsto & \mathcal{O}(S)
\end{array}
\]

\vspace{-0.1cm}

\begin{center}
\begin{tikzpicture}[node distance=2.7cm]
\node (S) {$S$};
\node (T) [right=of S] {$T$};
\draw[->] (S) -- node[above] {$\varphi$} (T);

\node (M) [below=0.55cm of $(S)!0.5!(T)$, rotate=-90] {$\Large\mapsto$};

\node (OS) [below=0.6cm of S] {$\mathcal{O}(S)$};
\node (OT) [below=0.6cm of T] {$\mathcal{O}(T)$};
\draw[->] (OT) -- node[below] {$\mathcal{O}(\varphi)=\varphi^{-1}$} (OS);
\end{tikzpicture}
\end{center}

% -------------------------
% Columna: pt(_)
% -------------------------
\column{0.49\textwidth}
\begin{center}
\textbf{El funtor $\pt(\_):\Frm \to \Top$}
\end{center}

\vspace{-0.2cm}

\[
\begin{array}{ccl}
A & \longmapsto & \pt(A)
\end{array}
\]

\vspace{-0.1cm}

\begin{center}
\begin{tikzpicture}[node distance=2.7cm]
\node (A) {$A$};
\node (B) [right=of A] {$B$};
\draw[->] (A) -- node[above] {$f$} (B);

\node (M2) [below=0.55cm of $(A)!0.5!(B)$, rotate=-90] {$\Large\mapsto$};

\node (ptA) [below=0.6cm of A] {$\pt(A)$};
\node (ptB) [below=0.6cm of B] {$\pt(B)$};
\draw[->] (ptB) -- node[below] {$\pt(f)=f_*$} (ptA);
\end{tikzpicture}
\end{center}

\end{columns}

\vspace{-0.25cm}

\begin{center}
\[
\Top(S,\pt(A)) \;\cong\; \Frm(A,\mathcal{O}(S)).
\]
\end{center}

\vspace{-0.1cm}
\centering{\normalsize\itshape \alert{La topología clásica y la topología sin puntos se conectan por medio de esta adjunción.}}

\end{frame}

\begin{frame}{Las ventajas de trabajar en Frm}
    \begin{itemize}
        \item Estructuras simples.
        \item<2-> Existen herramientas que facilitan el estudio de los marcos.
        \item<3-> Correspondencias biyectivas.
        \item<4-> Buen comportamiento categórico.
        \item<5-> A veces es más sencillo hacer teoría de marcos.
    \end{itemize}        
\end{frame}

\begin{frame}{Topología clásica vs. topología sin puntos}
  \begin{columns}[T,totalwidth=\textwidth]
    \begin{column}{0.48\textwidth}
      \textbf{Topología clásica}
      \begin{itemize}
        \item<1-> Requiere un conjunto de puntos $S$.
        \item<2-> Los abiertos son subconjuntos de $S$.
        \item<3-> Argumentos basados en puntos.
      \end{itemize}
    \end{column}
    \begin{column}{0.48\textwidth}
      \textbf{Topología sin puntos (marcos)}
      \begin{itemize}
        \item<4-> No requiere conjunto de puntos.
        \item<5-> Se trabaja con la estructura de abiertos.
        \item<6-> Resultados topológicos pueden ser trasladados a marcos.
      \end{itemize}
    \end{column}
  \end{columns}
\end{frame}


\section{¿Cómo estudiar topología sin puntos?}
\begin{frame}{Recorrido: de $\varepsilon$--$\delta$ a la topología sin puntos}
  \centering
  \resizebox{0.95\linewidth}{!}{%
  \begin{tikzpicture}[>=latex, node distance=3.0cm, every node/.style={align=center}]
    \node[draw, rounded corners, fill=red!10] (calc) {\textbf{Cálculo}\\$\varepsilon$--$\delta$\\($\mathbb{R}\to\mathbb{R}$)};
    \node[draw, rounded corners, fill=orange!10, right=of calc] (metric) {\textbf{Espacios métricos}\\Def. con distancias};
    \node[draw, rounded corners, fill=blue!10, right=of metric] (top) {\textbf{Topología}\\Preimágenes de abiertos};

    \coordinate (mid) at ($(metric)!0.5!(top)$);
    \node[draw, rounded corners, fill=green!6, below=of mid] (frm) {\textbf{Topología sin puntos}\\(marcos)};

    \draw[->, thick] (calc) -- node[above]{generalizar} (metric);
    \draw[->, thick] (metric) -- node[above]{abstraer} (top);
    \draw[->, thick] (top) -- node[right]{reinterpretar} (frm);
  \end{tikzpicture}%
  }
\end{frame}

\begin{frame}{De Top a Frm}
\centering
\resizebox{0.95\linewidth}{!}{%
\begin{tikzpicture}[>=latex, every node/.style={align=center}]
  % estilos
  \tikzset{
    box/.style={draw=black!30, rounded corners, fill=white,
                text=orange!70!black, font=\Large, inner sep=4pt,
                minimum width=44mm, minimum height=10mm},
    lab/.style={font=\bfseries\LARGE, text=black!60},
    arr/.style={->, line width=0.9pt, draw=black!55, -{Latex[length=2.2mm]}}
  }

  % columnas
  \coordinate (L) at (0,0);
  \coordinate (R) at (8,0);

  % encabezados
  \node[lab] at ($(L)+(0,2.5)$) {Top};
  \node[lab] at ($(R)+(0,2.5)$) {Frm};

  % cajas
  \node[box] (t1) at ($(L)+(0,1.2)$) {Noción \\ topológica};
  \node[box] (t2) at ($(L)+(0,-1.0)$) {Reescribir la noción \\ para el marco $\mathcal{O}S$};

  \node[box] (f1) at ($(R)+(0,1.2)$) {Noción \\en marcos};
  \node[box] (f2) at ($(R)+(0,-1.0)$) {Generalizar a un\\ marco arbitrario};

  % flechas (rectas)
  %\draw[arr] (t1) -- (f1);                 % horizontal
  \draw[arr] (f2.north) -- (f1.south);     % vertical hacia abajo
  \draw[arr] (t1.south) -- (t2.north);     % vertical hacia abajo
  \draw[arr] (t2) -- (f2);                 % horizontal de derecha a izquierda
\end{tikzpicture}%
}
\end{frame}

\begin{frame}{De Frm a Top}
\centering
\resizebox{0.95\linewidth}{!}{%
\begin{tikzpicture}[>=latex, every node/.style={align=center}]
  % estilos
  \tikzset{
    box/.style={draw=black!30, rounded corners, fill=white,
                text=orange!70!black, font=\Large, inner sep=4pt,
                minimum width=44mm, minimum height=10mm},
    lab/.style={font=\bfseries\LARGE, text=black!60},
    arr/.style={->, line width=0.9pt, draw=black!55, -{Latex[length=2.2mm]}}
  }

  % columnas base
  \coordinate (L) at (0,0);
  \coordinate (R) at (8,0);

  % encabezados
  \node[lab] at ($(L)+(0,2.5)$) {Top};
  \node[lab] at ($(R)+(0,2.5)$) {Frm};

  % cajas
  \node[box] (t1) at ($(L)+(0,1.2)$) {Definir propiedad\\ topológicas};
  \node[box] (t2) at ($(L)+(0,-1.0)$) {Obtener caracterización\\ para el marco $\mathcal{O}S$};

  \node[box] (f1) at ($(R)+(0,1.2)$) {Noción \\ en marcos};
  \node[box] (f2) at ($(R)+(0,-1.0)$) {Desarrollar teoría\\ para la noción en marcos};

  % flechas (todas rectas/verticales)
  %\draw[arr] (t1) -- (f1);                 % horizontal
  \draw[arr] (f1.south) -- (f2.north);     % vertical hacia abajo
  \draw[arr] (t2.north) -- (t1.south);     % vertical hacia arriba
  \draw[arr] (f2) -- (t2);       % regreso horizontal limpio
\end{tikzpicture}%
}
\end{frame}

\begin{frame}{Normalidad sin puntos}
Sea $S\in \Top$. Si $A\in \mathcal{O}S$, entonces $A'\in \mathcal{C}S$.
\begin{itemize}
\item<2-> \textbf{Normalidad:} para cualesquiera $A, B\in \mathcal{C}S$ con $A\cap B=\varnothing$, 
existen $U,V\in \mathcal{O}S$ con $A\subseteq U$, $B\subseteq V$ y $U\cap V=\varnothing$.

 \uncover<3->{\centering
  \textcolor{blue!60!black}{\normalsize¿Qué necesitamos para hacer la traducción?}}
  \item<4-> Si $A,B\in \mathcal{C}S \iff X=A', Y=B'\in \mathcal{O}S$.
    \item<5-> $A\cap B=\varnothing \iff X\cup Y=S$.
    \item<6-> $A\subseteq U \iff U'\subseteq X\iff X\cup U=S$.
    \item<7-> \alert{Traducción:} para cualesquiera $X, Y\in \mathcal{O}S$ con $X\cup Y=S$, 
existen $U,V\in \mathcal{O}S$ con $X\cup U=Y\cup V=S$ y $U\cap V=\varnothing$. 
\end{itemize}
\end{frame}

\begin{frame}{Ingredientes para traducir $T_1$}

\begin{block}{Elementos en un marco $A$}
  \begin{itemize}
    \item<2-> \textbf{máximo:} un elemento $m\in A$ es \emph{máximo} si $m<1$ y,
    para todo $b\in A$, \; $m\le b<1 \;\Rightarrow\; b=m$.
    % (equiv.: no existe propio entre $m$ y $1$)

    \item<3-> \textbf{primo:} un elemento $a\in A$ con $a\ne 1$ es
    \emph{primo} si para todos $u,v\in A$,
    \[
      u\wedge v \le a \;\Rightarrow\; (u\le a)\ \text{o}\ (v\le a).
    \]
  \end{itemize}
\end{block}

\onslide<4->\begin{alertblock}{Observación}
En un \emph{marco} todo elemento \textbf{máximo} es
\textbf{primo}. El recíproco \textbf{no} vale en general.
\end{alertblock}

\end{frame}

\begin{frame}{T$_1$ sin puntos}
\textbf{T$_1$:} Para todo $x\neq y\in S$, existe $U\in \mathcal{O}S$ con $x\in U$ y $y\notin U$.

\vspace{0.5em}
\begin{itemize}
  \item<2-> $S$ es $T_1$ $\iff$ $\forall\,x\in S$, $\overline{\{x\}}=\{x\}$ (los unipuntuales son cerrados).
  \item<3-> Entonces $S\setminus\{x\}\in \mathcal{O}S$.
  \item<4-> $S\setminus\overline{\{x\}}$ es un elemento máximo.
  \item<5-> $S\setminus\{x\}$ es un elemento primo. 
  \item<6-> \alert{Traducción:} todo elemento primo es máximo.
\end{itemize}
\end{frame}

\begin{frame}{Ingredientes para traducir regularidad}

  Si $a,b\in A$, podemos hacer algebra a través de sus operaciones.
  \begin{itemize}
    \item<2-> Operaciones del marco: $\bigvee,\;\wedge,\;\succ,\;\prec$\; y \;$\neg$.
    \item<3-> Implicación:
      \[
        (a \succ b)=\bigvee\{x\in A\mid x\wedge a\leq b\}.
      \]
    \item<4-> Negación:
      \[
        \neg a=\bigvee\{x\in A\mid x\wedge a=0\}=(a \succ 0).
      \]
    \item<5-> Relación bastante por debajo: 
      \[
      \begin{split}
        a \prec b &\iff\;\mbox{ existe } c\in A \mbox{ tal que } c\wedge a=0, c\vee b=1.\\
        &\iff\; \neg a \vee b=1.
      \end{split}
      \]
  \end{itemize}

\end{frame}


\begin{frame}{Regularidad sin puntos}
\textbf{Regularidad:} Para todo $x\in S$ y $X\in \mathcal{C}S$ con $x\notin X$, existen $U,V\in \mathcal{O}S$ tales que 
\[
x\in U, \quad X\subseteq V, \quad U\cap V=\varnothing.
\]
\begin{itemize}
  \item<2-> $S$ es regular $\iff$ para todo $x\in X$ y $U\in \mathcal{O}S$ con  $x\in U$, existe $V\in\mathcal{O}S$ tal que
    \[
      x\in V\subseteq \overline{V}\subseteq U.
    \]
  \item<3-> Si $V\in \mathcal{O}S$, entonces $\neg V=S\setminus \overline{V}\in \mathcal{O}S$.
  \item<4-> $\overline{V}\subseteq U$ $\iff$ $\neg V\vee U=S$ $\iff$ $V\prec U$.
  \item<5-> \alert{Traducción:} para todo $U\in \mathcal{O}S$
  \[
    U=\bigcup\{V\in \mathcal{O}S \mid V\prec U\}.
  \]
\end{itemize}
\end{frame}

\begin{frame}{Los axiomas en Frm}
Si $A$ es un marco arbitrario, entonces:
\begin{itemize}
  \item<1-> $A$ es \textbf{T$_1$} si todo elemento primo es máximo.
  \item<2-> $A$ es \textbf{regular} si para todo $a\in A$
  \[
    a=\bigvee\{x\in A \mid x\prec a\}.
  \]
  o equivalentemente $\forall \; a\nleq b\in A\;$ $\exists\; x, y\in A$ tales que 
  \[
    a\vee x=1, \quad y\nleq b\quad x\wedge y=0.
  \]
  \item<3-> $A$ es \textbf{normal} si para todos $a,b\in A$ con $a\vee b=1$, existen $u,v\in A$ tales que
  \[
    a\vee u=1, \quad b\vee v=1, \quad u\wedge v=0.
  \]
  %\uncover<4->{\centering
  %\textcolor{blue!60!black}{\large Obtuvimos algunos axiomas de separación que podemos usar cuando trabajemos con marcos :D}}
\end{itemize}
\end{frame}

\begin{frame}[fragile]{Axiomas tipo Hausdorff}
En espacios
\[
T_3\Rightarrow T_2 \Rightarrow T_1
\]

$T_2$ es más débil que $T_3$ y más fuerte que $T_1$. Además, 
\[
S\text{ es }T_2\iff \{(s, s)\in S\times S\}\subseteq S\times S\text{ es cerrado}.
\]

En marcos
\[\begin{tikzcd}
	{\mathbf{(reg)}} & {\mathbf{(fH)}} & {\mathbf{(H)}} & {\mathbf{(dH)}} & {T_1} \\
	&&& {\mathbf{(Hp)}}
	\arrow[Rightarrow, from=1-1, to=1-2]
	\arrow[Rightarrow, from=1-2, to=1-3]
	\arrow[Rightarrow, from=1-3, to=1-4]
	\arrow[Rightarrow, from=1-3, to=2-4]
	\arrow[Rightarrow, from=1-4, to=1-5]
\end{tikzcd}\]

{\centering
\textcolor{blue!60!black}{\large La teoría de separación en $\Frm$ es más rica que en $\Top$.}}
\end{frame}

\begin{frame}{Topología sin puntos y otras áreas de las matemáticas}

\begin{itemize}
\item Lógica matemática.
\item Teoría de categorías.
\item Teoría de haces y topos.
\item Ciencias de la computación.
\item Topología clásica.
\end{itemize}

\vspace{0.3cm}

\centering{\Large\itshape La topología sin puntos es un lenguaje transversal.}

\end{frame}

\begin{frame}{Mensaje final}

\begin{itemize}
\item La topología puede estudiarse desde los puntos\dots
\item pero también desde la estructura de los abiertos.

\item La topología sin puntos:
\begin{itemize}
\item no reemplaza a la topología clásica,
\item la extiende y la profundiza.
\end{itemize}

\item Muchas nociones clásicas admiten formulaciones puramente algebraicas.
\end{itemize}

\vspace{0.3cm}

\centering{\Large\itshape Cambiar el punto de vista cambia las preguntas que podemos hacer.}
\end{frame}


%\begin{frame}{Algunas traducciones más tecnicas}
%\scriptsize
%\setlength{\tabcolsep}{2pt}
%\renewcommand{\arraystretch}{1.15}
%\begin{tabular}{|p{.43\linewidth}|p{.14\linewidth}|p{.43\linewidth}|}
%\hline
%\textbf{Noción locálica (geométrica)} & \textbf{Notación} & \textbf{Correspondencia en núcleos (algebraica)}\\ \hline
%Sublocal denso más pequeño de $A$ & $d(A)$ & Corresponde al núcleo doble negación ($w_0$).\\ \hline
%Sublocal denso en $A$ &  & Corresponde al núcleo $j$ que está por debajo del núcleo doble negación ($j\leq w_0$).\\ \hline
%Sublocal abierto &  & Corresponde al núcleo $v_a$.\\ \hline
%Sublocal cerrado &  & Corresponde al núcleo $u_a$.\\ \hline
%Sublocal booleano &  & Corresponde al núcleo $w_a$.\\ \hline
%Sublocal $A_k$ denso en $A_j$ &  & Corresponde al núcleo $k$ tal que $k\leq w_a$.\\ \hline
%Sublocal $A_k$ denso en $NA_j$ &  & Corresponde al núcleo $k$ tal que $j\lessdot k$.\\ \hline
%Sublocal $A_k$ denso en ninguna parte de $A_j$ & $A_k\leq_{nd} A_j$ & Corresponde al núcleo $k$ tal que $j(0)\lessdot k(0)$.\\ \hline
%\end{tabular}

%\large La tabla completa puede verse en \cite{J.M.}
%\end{frame}


\section*{\textsc{Referencias}}
\begin{frame}[allowframebreaks]
\frametitle{Bibliografía}
\begin{thebibliography}{20}\markboth{Bibliografía}{Bibliografía}

\bibitem{P.T.} P. T. Johnstone, \textit{Stone spaces}, Cambridge Studies in Advanced Mathematics, vol. 3, Cambridge University Press, Cambridge, 1982. MR 698074

%\bibitem{J.M.} J. Monter; A. Zaldívar, \textit{El enfoque locálico de las reflexiones booleanas: un análisis en la categoría de marcos} [tesis de maestría], 2022. Universidad de Guadalajara.

%\bibitem{P.S.} J. Paseka and B. Smarda, \textit{$ T_2 $-frames and almost compact frames.} Czechoslovak Mathematical Journal (1992), 42(3), 385-402.

\bibitem{J.P.} J. Picado and A. Pultr, \textit{Frames and locales: Topology without points}, Frontiers in Mathematics, Springer Basel, 2012.

%\bibitem{J.P.2} J. Picado and A. Pultr, \textit{Separation in point-free topology}, Springer, 2021.

%\bibitem{R.S.} RA Sexton, \textit{A point free and point-sensitive analysis of the patch assembly}, The University of Manchester (United Kingdom), 2003.

%\bibitem{R.S.2} RA Sexton, \textit{Frame theoretic assembly as a unifying construct}, The University of Manchester (United Kingdom), 2000.

%\bibitem{R.S.3} RA Sexton and H. Simmons, \textit{Point-sensitive and point-free patch constructions}, Journal of Pure and Applied Algebra \textbf{207} (2006), no. 2, 433-468.

\bibitem{H.S.} H. Simmons, \textit{An Introduction to Frame Theory}, lecture notes, University of Manchester. Disponible en línea en \url{https://web.archive.org/web/20190714073511/http://staff.cs.manchester.ac.uk/~hsimmons}.

%\bibitem{H.S.R} H. Simmons, \textit{Regularity, fitness, and the block structure of frames.} Applied Categorical Structures 14 (2006): 1-34.

%\bibitem{H.S.4} H. Simmons, \textit{The lattice theoretic part of topological separation properties}, Proceedings of the Edinburgh Mathematical Society, vol.~21, pp.~41--48, 1978.

%\bibitem{H.S.V} H. Simmons, \textit{The Vietoris modifications of a frame}. Unpublished manuscript (2004), 79pp., available online at http://www. cs. man. ac. uk/hsimmons.

%\bibitem{A.W.} A. Wilansky, \textit{Between T1 and T2}, MONTHLY (1967): 261-266.

\bibitem{A.Z.} A. Zaldívar, \textit{Introducción a la teoría de marcos} [notas curso], 2025. Universidad de Guadalajara.
\end{thebibliography}
\end{frame}

\begin{frame}[plain,standout]
      \centering
      \Huge{\smiley Gracias por su atención\smiley}
      %In combination with \textit{plain}, \\
      %it makes a nice thank-you slide!
      \vfill
      \scalebox{4}{\faGithub} \par\bigskip
      \url{https://github.com/JCmonter/Apuntes/tree/main/Presentaciones} \\
      %\url{https://ctan.org/pkg/beamertheme-arguelles}
\end{frame}

\appendix
\section*{Apéndice: Definiciones de respaldo}

% --- Orden parcial ---
\begin{frame}{Orden parcial}
  \begin{block}{Definición}
    Una relación $\leq$ en un conjunto $A$ es un \textbf{orden parcial} si cumple:
    \begin{itemize}
      \item Reflexividad: $a \leq a$.
      \item Antisimetría: $a \leq b$ y $b \leq a \;\Rightarrow\; a=b$.
      \item Transitividad: $a \leq b$ y $b \leq c \;\Rightarrow\; a \leq c$.
    \end{itemize}
  \end{block}
  \vspace{0.5em}
  \centering
  \begin{tikzpicture}[scale=0.9, every node/.style={circle,draw,inner sep=1.5pt}]
    \node (a) at (0,0) {a};
    \node (b) at (1.2,1) {b};
    \node (c) at (-1.2,1) {c};
    \node (d) at (0,2) {d};
    \draw (a) -- (b) -- (d) -- (c) -- (a);
  \end{tikzpicture}
  \\
  \scriptsize Ejemplo de un conjunto parcialmente ordenado (Hasse).
\end{frame}

% --- Supremo / Ínfimo ---
\begin{frame}{Supremo e ínfimo}
  \begin{itemize}
    \item El \textbf{supremo} de un subconjunto es la menor cota superior.
    \item El \textbf{ínfimo} es la mayor cota inferior.
  \end{itemize}
  \vspace{0.8em}
  \centering
  \begin{tikzpicture}[scale=0.9, every node/.style={circle,draw,inner sep=1.5pt}]
    \node (0) at (0,0) {0};
    \node (a) at (-1,1) {a};
    \node (b) at (1,1) {b};
    \node (1) at (0,2) {1};
    \draw (0) -- (a) -- (1);
    \draw (0) -- (b) -- (1);
  \end{tikzpicture}
  \\
  \scriptsize Aquí $\sup\{a,b\} = 1$, \; $\inf\{a,b\} = 0$.
\end{frame}

% --- Semirretícula y retícula ---
\begin{frame}{De semirretícula a retícula}
  \begin{itemize}
    \item Una \textbf{semirretícula} es un poset con supremos finitos (o ínfimos finitos).
    \item Una \textbf{retícula} tiene ambos: supremos e ínfimos finitos.
  \end{itemize}
  \vspace{0.8em}
  \centering
  \begin{tikzpicture}[scale=0.9, every node/.style={circle,draw,inner sep=1.5pt}]
    \node (0) at (0,0) {0};
    \node (a) at (-1,1) {a};
    \node (b) at (1,1) {b};
    \node (c) at (0,1) {c};
    \node (1) at (0,2) {1};
    \draw (0) -- (a) -- (1);
    \draw (0) -- (b) -- (1);
    \draw (0) -- (c) -- (1);
  \end{tikzpicture}
  \\
  \scriptsize Retícula: todos los pares tienen sup e inf.
\end{frame}

% --- Marco ---
\begin{frame}{Marco}
  \begin{block}{Definición}
    Un \textbf{marco} es una retícula completa $(A, \leq, \bigvee, \wedge, 0, 1)$
    donde vale la \textbf{Ley Distributiva de Marcos} (LDM):
    \[
      a \wedge \bigvee_{i \in I} b_i \;=\; \bigvee_{i \in I}(a \wedge b_i).
    \]
  \end{block}
  \vspace{0.8em}
  \centering
  \scriptsize Esta es la definición formal que conecta con la topología sin puntos.
\end{frame}

\begin{frame}[fragile]{Un ejemplo espacial}
Sea $S=\mathbb{R}$ y consideremos las topologías generadas por:
\[
\mathcal{O}_lS=\{(-\infty, a)\},\quad  \mathcal{O}_mS=\{(a,b)\}, \quad \mathcal{O}_nS=\{[a,b)\},
\]
donde $a,b\in S$. Entonces
\[
\mathcal{O}_lS \hookrightarrow \mathcal{O}_mS \hookrightarrow \mathcal{O}_nS
\]

Se puede verificar que 
\[
\mathcal{O}_l^pS=\mathcal{O}_mS\simeq P\mathcal{O}_lS\quad\mbox{ y }\quad\mathcal{O}_l^fS=\mathcal{O}_nS\simeq N\mathcal{O}_lS,
\]
es decir, 
\[
\mathcal{O}_lS=A\rightarrow PA\hookrightarrow NA
\]
\end{frame}

\begin{frame}{Axiomas tipo Hausdorff}
    \begin{description}
        \item[$\mathbf{(dH):}$]$\quad$ Si $a\vee b=1$, con $a,b\neq 1$, $\exists\, u,v$ tales que $u\nleq a$, $v\nleq b$ y $u\wedge v=0$.
        \item[$\mathbf{(H):}$]$\quad$ Si $1\neq a\nleq b$ $\exists\, u, v$ tales que $u\nleq a$, $v\nleq b$ y $u\wedge v=0$.
        \item[$\mathbf{(Hp):}$]$\quad$ Todo elemento semiprimo es máximo.
        \item[$\mathbf{(fH):}$]$\quad$ El sublocal diagonal es cerrado.
    \end{description}
\end{frame}

\end{document}