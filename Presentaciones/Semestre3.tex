\documentclass[compress,12pt]{beamer}

\usetheme{Arguelles}
\usepackage{graphicx}
\usepackage{caption}
%\usepackage[spanish,es-noshorthands]{babel}
\usepackage[spanish]{babel} 
\usepackage[pages=some]{background}
\usepackage{tikz-cd}
\usepackage{amsmath,amssymb,latexsym,amscd} 
\usepackage[all,cmtip]{xy}
\usepackage{fancyhdr}
\usepackage{mathalfa}
\usepackage{mathrsfs}
\usetikzlibrary{babel}
\usepackage{hyperref}
\usepackage{ragged2e}
\usepackage{wasysym}
\usepackage{multicol}
%\hypersetup{colorlinks=true,linkcolor=blue,citecolor=brown,linktocpage=true,pagebackref=true,hyperindex=true}
\pagenumbering{arabic}

\DeclareMathOperator{\op}{op}
\DeclareMathOperator{\pt}{pt}
\DeclareMathOperator{\spec}{spec}
\DeclareMathOperator{\Fit}{Fit}
\DeclareMathOperator{\Pth}{P}
\DeclareMathOperator{\Frm}{Frm}
\DeclareMathOperator{\Obj}{Obj}
\DeclareMathOperator{\Hom}{Hom}

\title{¿Por qué son más las nociones Hausdorff en álgebra que en topología?}
\event{Seminario de Álgebra, CUCEI}
\date{\today}
\author{Juan Carlos Monter Cortés}
\institute{Universidad de Guadalajara}
\email{juan.monter2902@alumnos.udg.mx}

%\homepage{www.mywebsite.com}
%\github{username}

\begin{document}

\frame[plain]{\titlepage}

\begin{frame}{Lo que veremos hoy \smiley}
\tableofcontents %Imprime la tabla de contenido
\end{frame}

\section{Preliminares}

\begin{frame}{¿Qué es un marco?}
    \begin{itemize}
    \begin{multicols}{2}
        \item $A$
        \item $(A, \leq)$
        \item $(A, \leq, \vee, 0)$ o $(A, \leq, \wedge, 1)$
        \item $(A, \leq, \bigvee, \bigwedge, 0, 1)$
    \end{multicols} 
    \end{itemize}

    \onslide<2->{Un marco es una retícula completa que cumple cierta ley distributiva (ley distributiva de marcos), es decir,}
    \onslide<3->{\[
    (A, \leq, \bigvee, \wedge, 0, 1), \quad a\wedge\bigvee X=\bigvee \{a\wedge x\mid x\in X\}
    \]}

    \onslide<4->\[
    \mathbf{Frm}=\left\{ \begin{array}{ll} A, & \mbox{ marcos}\\ \\  f, & \mbox{ morfismo de marcos} \end{array} \right.
    \]
\end{frame}

\begin{frame}{¿Por qué estudiamos los marcos?}
    \begin{itemize}
        \item Estructuras simples.
        \item<2-> Existen herramientas que facilitan el estudio de los marcos.
        \item<3-> Correspondencias biyectivas.
        \item<4-> Buen comportamiento categórico.
        \item<5-> \alert<7->{La topología de un espacio ($\mathcal{O}S)$ es un marco.}
        \item<6-> $\mathbf{Loc}=\mathbf{Frm}^{\op}$ está en relación con $\mathbf{Top}$.
    \end{itemize}
\end{frame}

\begin{frame}[fragile]{¿Qué relación existe entre los marcos y los espacios topológicos?}
\[
\left(\mathcal{O}S, \subseteq, \cap, \bigcup, S, \emptyset\right)
\]

\begin{itemize}
    \item $\mathcal{O}(\_)\colon \mathbf{Top}\to \mathbf{Frm}$ 
    \item $\pt(\_)\colon \mathbf{Frm}\to \mathbf{Top}$
    \item El funtor $\pt$ es adjunto izquierdo de $\mathcal{O}$.
\end{itemize}
\[\begin{tikzcd}
	{\mathbf{Top}} \\
	{} & {} & {} \\
	{\mathbf{Frm}}
	\arrow["{\pt}"{name=0, swap}, from=3-1, to=1-1, shift right=5]
	\arrow["{\mathcal{O}}"{name=1, swap}, from=1-1, to=3-1, shift right=5]
	\arrow["\dashv"{rotate=0}, from=1, to=0, phantom]
\end{tikzcd}\]
    
\end{frame}

\begin{frame}[fragile]
\frametitle{Axiomas de separación}
    En la topología clásica (topología sensible a puntos), tenemos la siguiente relación entre los axiomas de separación 

    \[\begin{tikzcd}
	{T_4} & {T_{3\frac{1}{2}}} & {T_3} & {T_2} & {T_1} & {T_0} \\
	& {(norm)} & {R_3} & {R_2} & {R_1} & {R_0}
	\arrow[Rightarrow, from=1-2, to=1-3]
	\arrow[Rightarrow, from=1-3, to=1-4]
	\arrow[Rightarrow, from=1-4, to=1-5]
	\arrow[Rightarrow, from=1-5, to=1-6]
	\arrow[Rightarrow, from=1-1, to=1-2]
	\arrow[Rightarrow, from=1-1, to=2-2]
	\arrow[Rightarrow, from=1-2, to=2-3]
	\arrow[Rightarrow, from=1-3, to=2-4]
	\arrow[Rightarrow, from=1-4, to=2-5]
	\arrow[Rightarrow, from=1-5, to=2-6]
	\arrow[Rightarrow, from=2-3, to=2-4]
	\arrow[Rightarrow, from=2-4, to=2-5]
	\arrow[Rightarrow, from=2-5, to=2-6]
\end{tikzcd}\]
\[
    \mbox{ }
\]
    
donde $R_2=\mathbf{(reg)}$, $R_3=\mathbf{(creg)}$, $T_i=R_{i-1}+T_{i-1}$ y $T_{3\frac{1}{2}}=R_3+T_3$, para $i=1, 2, 3$.\\

$T_4=\mathbf{(norm)}+T_1$.
    
\end{frame}

\begin{frame}{El axioma $T_2$}
    \begin{description}
        \item[$T_2$:] Sean $x\neq y\in S$, entonces existen $U, V\in \mathcal{O}S$ tales que
        \[
        x\in U,\quad y\in V, \quad U\cap V=\emptyset.
        \]
    \end{description}
\end{frame}

\begin{frame}{¿Qué sabemos sobre $T_2$?}
    \begin{itemize}
        \item Es más fuerte que $T_1$.
        \item<2-> Es más débil que $T_3$.
        \item<3-> Un espacio $S$ es $T_2$ si y solo si la diagonal es cerrada.
        \item<4-> Si $S$ es $T_2$, todo $Q\in \mathcal{Q}S$ es cerrado.
        \item<5-> Si $S$ es $T_2$ y compacto, $S$ es regular.
        \item<6-> El producto de espacios $T_2$ es $T_2$.
        \item<7-> Los subespacios de un espacio $T_2$ son $T_2$.
        \item<8-> $\vdots$
    \end{itemize}
\end{frame}

\section{Aspectos libres de puntos}

\begin{frame}{Un poco más sobre los marcos}
    Sean $a, b\in A\in \mathbf{Frm}$.
    \begin{itemize}
        \item<2-> $\neg a=$ negación (pseudocomplemento).
        \item<3-> $a^*=$ complemento.
        \item<4-> $(a\succ b)=\bigvee\{x\in A\mid x\wedge a\leq b\}$.
        \item<5-> $\neg a=(a\succ 0)$.
        \item<6-> $a$ es regular si $a=\neg\neg a$.
        \item<7-> $A\in \mathbf{Frm}$ es espacial si $A=\mathcal{O}S$.
    \end{itemize}
\end{frame}

\begin{frame}{Un poco sobre sublocales}
    \begin{block}{Definición:}
        Sea $L\in \mathbf{Loc}$. $S\subseteq L$ es un sublocal si:
        \begin{enumerate}
            \item $a, b\in S \Rightarrow a\wedge b\in S$.
            \item $l\in L$ y $s\in S$ $\Rightarrow (l\succ s)\in S$.
        \end{enumerate}
    \end{block}

    \onslide<2->\[
    S\mbox{ es cerrado}\Leftrightarrow S=\overline{S}
    \]
    \onslide<3->{¿Quién es $\overline{S}$?}
    \onslide<4->{\[
    \overline{S}=\uparrow \bigwedge S
    \]
    para cada sublocal $S$ y $\bigwedge S\in S$.}
\end{frame}

\begin{frame}{Ajustado, subajustado y el axioma $T_D$}
\begin{description}
    \item[$\mathbf{(aju)}$] $\quad\forall\; U, V\in \mathcal{O}S$, con $U\nsubseteq V$, $\exists\; W\in \mathcal{O}S$ tal que 
    \[
    U\cup W=S\quad \mbox{ y }\quad (W\cup V)^\circ\neq V.
    \]
    \item<2->[$\mathbf{(saju)}$] $\quad\forall\; U, V\in \mathcal{O}S$, con $U\nsubseteq V$, $\exists\; W\in \mathcal{O}S$ tal que 
    \[
    U\cup W=S\neq V\cup W.
    \]
    \item<3->[$T_D$] $\quad\forall\; x\in S$, $\exists\; U\in \mathcal{O}S$ tal que 
    \[
    x\in U\quad \mbox{ y } \quad U\setminus \{x\}\in \mathcal{O}S.
    \]
\end{description}
\onslide<4->\[
    \mathbf{(aju)}\Rightarrow \mathbf{(saju)},\quad \mathbf{(saju)}\nRightarrow T_D,\quad T_D\nRightarrow \mathbf{(saju)}
    \]
\end{frame}

\begin{frame}{Otra alternativa de ver los axiomas $T_n$}
    \[
    \begin{split}
        T_1&= T_D+\mathbf{(saju)}\\
        T_2&= N_2+\mathbf{(saju)}\\
        T_3&= N_3+\mathbf{(saju)}\\
        T_4&= N_4+\mathbf{(saju)}\\
    \end{split}
    \]
    \onslide<2->{donde $N_i$, para $i=2, 3, 4$, son conocidas como \emph{propiedades de normalidad} (ver \cite{H.S.4})}.\\
    \begin{tiny}
$$\mbox{ }$$
\end{tiny}

    \onslide<3->\alert{Existen otras maneras de definir los axiomas de separación en $\mathbf{Frm}$}
\end{frame}

\begin{frame}{Elemento máximo, primo y semiprimo}
    $A\in \mathbf{Frm}$ y $S=\pt A$, entonces 
    \begin{itemize}
        \item<2-> $p\in S$ si y solo si $p\neq 1$ y $\forall\; a, b\in A$ si 
        \[
        a\wedge b\leq p\Rightarrow a\leq p\mbox{ o } b\leq p.
        \]
        Si $p\in S$ entonces $p$ es \emph{primo}.
        \item<3-> $p\in A$ es \emph{semiprimo} si $\forall\; a, b\in A$ con $a\wedge b=0$, entonces $a\leq p$ o $b\leq p$.
        \item<4-> $p\in A$ es \emph{máximo} si $\forall\; m\in A$ con $p\leq m$, entonces $p=m$ o $m=1$.
    \end{itemize}

\onslide<5->\[
    \mbox{máximo }\Rightarrow \mbox{ primo }\Rightarrow \mbox{ semiprimo}
    \]
\end{frame}

\begin{frame}{Axiomas de separación}
\framesubtitle{(libres de puntos)}
    \begin{description}
    \item[$(T_1)$] Todo elemento primo es máximo.
    \item<2->[$(reg)$] $\quad\forall\, U\in \mathcal{O}S$, con ``$\prec$''= \emph{bastante por debajo}
    \[
    U=\bigcup\{V\mid V\prec U\}
    \]
    \item<3->[$(T_3)$] $=(reg)+(T_{1})$ 
    \item<4->[$(creg)$] $\quad\forall\, U\in \mathcal{O}S$, con ``$\prec\prec$''= \emph{completamente por debajo}
    \[
    U=\bigcup\{V\mid V\prec\prec U\}
    \]
    \item<5->[$(T_{3\frac{1}{2}})$] $=(creg)+(T_{1})$ 
    \item<6->[$(norm)$] $\quad\forall\, X, Y \in \mathcal{O}S$ tales que $X\cup Y=S$, $\exists$ $U, V \in \mathcal{O}S$ tales que
    \[
    X\cup U=S, \quad Y\cup V=S, \quad U\cap V=\emptyset
    \]
    \item<7->[$(T_{4})$] $=(norm)+(T_{1})$ 
\end{description}
\end{frame}

\section{Hausdorff en $\mathbf{Frm}$}

\begin{frame}{Propiedad conservativa, de 1° orden y de 2° orden}

\begin{definition}
    \begin{enumerate}
        \item Para un espacio $S$ decimos que una propiedad $P$ es \emph{conservativa} si y solo si $\mathcal{O}S$ tiene la propiedad $P_S$.

        \item<2-> Decimos que una propiedad en marcos $P$ es \emph{suficientemente Hausdorff} si y solo si $P$ implica la propiedad Hausdorff espacial.
        
        \item<3-> Decimos que una propiedad en marcos $P$ es de \emph{1° orden} si y solo si $P$ es enunciada como una fórmula para elementos del marco.

        \item<4-> Decimos que una propiedad en marcos $P$ es de \emph{2° orden} si y solo si $P$ es enunciada como una caracterización de sublocales.
    \end{enumerate}
\end{definition}

\end{frame}

\subsection{Algo más débil que $T_2$}
\begin{frame}{Débilmente Hausdorff}
\framesubtitle{[Dowker y Strauss (1972)]}
    Esta noción sugiere ver $T_2$ como algo más fuerte que $T_1$.\\

    \onslide<2->\begin{block}{Definición}
        Sea $A\in \mathbf{Frm}$. Decimos que $A$ es \emph{débilmente Hausdorff} si se cumple lo siguiente:
        \begin{description}
            \item[$(\mathbf{dH})$] $\quad a\vee b=1$ y $a, b\neq 1$, $\exists$ $u, v$ tales que $u\nleq a$, $v\nleq b$ y $u\wedge v=0$.  
        \end{description}
        con $a, b, u, v\in A$.
    \end{block}

    \onslide<3->{Recordemos que 
    \[
    T_1=T_D+\mathbf{(saju)}.
    \]}
\end{frame}

\begin{frame}
        La propuesta de Dowker y Strauss fue
    \[
    T_2=\mathbf{(dH)}+\mathbf{(saju)}.
    \]

\onslide<2->\begin{block}{Observaciones}
    \begin{itemize}
        \item<3-> $\mathbf{(dH)}=N_2$.
        \item<4-> Un espacio $S$ cumple $T_2$ (espacial) $\Leftrightarrow$ $\mathcal{O}S$ cumple $T_2$ (en marcos).
        \item<5-> $\mathbf{(dH)}$ no es suficientemente Hausdorff.
    \end{itemize}
\end{block}
    
\end{frame}

\subsection{Algo más fuerte que $T_2$}
\begin{frame}{Hausdorff}
\framesubtitle{[Johnstone y Shu-Hau (1987)]}
 Esta noción sugiere ver $T_2$ como algo más débil que $T_3$.\\
 
    \onslide<2->\begin{block}{Definición}
    Sea $A\in \mathbf{Frm}$ y consideremos $a,b\in A$. Decimos que $a$ está \emph{bastante por debajo} de $b$ (denotado por ``$a\prec b$'') si $\exists\; c\in A$ tal que 
    \[
    a\wedge c=0 \quad\mbox{ y }\quad c\vee b=1
    \]
    \end{block}

\onslide<3->\begin{description}
  \item<2->[$\mathbf{(reg)}$] $\quad\forall\, b\in A$,
    \[
    b=\bigvee\{a\in A\mid a\prec b\}
    \]
\end{description}
    
\end{frame}

\begin{frame}
    Haciendo las respectivas modificaciones obtenemos

    \onslide<2->\begin{block}{Definición [Paseka y Smarda (1987)]:}
        Sea $A\in \mathbf{Frm}$ y sean $a, b\in A$ con $b\neq 1$. Decimos que ``$a\sqsubset b$'' si 
        \[
        a\leq b\quad \mbox{ y }\quad \neg a\leq b
        \]
    \end{block}

    \onslide<3->\begin{description}
        \item[$T_2$] $\forall\;b\in A$ con $b\neq 1$,
        \[
        b=\bigvee\{a\in A\mid a\sqsubset b\}.
        \]
    \end{description}
    \onslide<4->{Equivalentemente

    \begin{description}
        \item[$T_2$] $\forall\;1\neq a \nleq b$, $\exists\; u, v\in A$ tales que  
        \[
        u\nleq a,\quad v\nleq b, \quad v\leq a,\quad u\wedge v=0.
        \]
    \end{description}}
    
\end{frame}

\begin{frame}
    Johnstone y Shu-Hau modificaron la noción dada por Isbell y obtuvieron 

    \begin{description}
        \item[$\mathbf{(H)}$] $\forall\;1\neq a\nleq b$, $\exists\; u, v\in A$ tales que
        \[
        u\nleq a, \quad v\nleq b,\quad u\wedge v=0.
        \]
    \end{description}

    \onslide<2->\begin{block}{Observaciones:}
    \begin{itemize}
        \item<3-> $T_2 \mbox{ (P. y S.) } \Leftrightarrow \mathbf{(H)}$.
        \item<4-> $\mathbf{(H)}\Leftrightarrow T_2$ (D. y S.).
        \item<5-> Un espacio cumple $T_2$ (espacial) $\Leftrightarrow \mathcal{O}S$ cumple $\mathbf{(H)}$.
        \item<6-> $\mathbf{(H)}$ es hereditaria.
        \item<7-> $\mathbf{(H)}$ es cerrada bajo coproductos.
        \item<8-> $\mathbf{(reg)}\Rightarrow \mathbf{(H)}$
    \end{itemize}
        
    \end{block}
\end{frame}

\begin{frame}{Ejemplo}
    $A\in \mathbf{Frm}$ y sea 
    \[
    \tilde{A}=\{(x, y)\mid x\leq y\}\subseteq L\times \mathcal{B}(L),
    \]
    donde
    \[
    \mathcal{B}(L)=\{x\in A\mid x=\neg\neg x\}=\{x^*\mid x\in L\}
    \]

    \onslide<2->\begin{block}{Observaciones:}
        \begin{itemize}
            \item<3-> $A$ es Hausdorff $\Rightarrow$ $\tilde{A}$ es Hausdorff.
            \item<4-> $A$ compacto $\Rightarrow$ $\tilde{A}$ compacto.
            \item<5-> $\tilde{A}$ no es subajustado.
            \item<6-> $\tilde{A}$ es Hausdorff y compacto, pero no regular.
        \end{itemize}
    \end{block}
\end{frame}

\subsection{Algo parecido a $T_2$}
\begin{frame}{Fuertemente Hausdorff}
\framesubtitle{[Isbell (1972)]}
Esta noción sugiere ver $T_2$ en marcos como algo similar a $T_2$ espacial.\\  

\onslide<2->\begin{block}{Proposición:}
    $S$ es $T_2$ $\Leftrightarrow \Delta=\{(x, x)\in (S\times S)\mid x\in S\}$ es cerrada en $S\times S$.
\end{block}

\onslide<3->{La caracterización anterior es traducida para el producto de locales (coproducto de marcos).} 
\end{frame}

\begin{frame}
    \begin{block}{Definición:}
    Sea $L$ un local. Consideramos el coproducto binario $L\oplus L$. Decimos que un marco es \emph{fuertemente Hausdorff} si y solo si el sublocal diagonal $\Delta[L]$ es cerrado en $L\oplus L$.
\end{block}

\uncover<2->{La propiedad enunciada en la definición anterior puede ser reescrita de la siguiente manera.

\begin{description}
    \item[$\mathbf{(fH)}$] $\quad\Delta[L]=\uparrow d_L$, 
\end{description}
donde $d_L$ es el menor elemento de $\Delta[L]$, es decir,
\[
d_L=\Delta(0)=\{(x, y)\mid x\wedge y\leq 0\}=\downarrow\{(x, x^*)\mid x\in L\}.
\]}
\end{frame}

\begin{frame}
\begin{block}{Observaciones:}
    \begin{itemize}
    \item Cada sublocal de un marco fuertemente Hausdorff es fuertemente Hausdorff.
    \item<2-> $\mathbf{(fH)}$ $\Rightarrow$ $\mathbf{(H)}$.
    \item<3-> $\mathbf{(fH)} + \mbox{compacto}$ $\Rightarrow$ $\mathbf{(reg)}$.
    \item<4-> Sean $S$ un espacio $T_0$ y $\mathcal{O}S$ un marco fuertemente Hausdorff. Entonces $S$ es $T_2$.
    \item<5-> $\mathbf{(H)} + \mathbf{(saju)}$ $\nRightarrow$ $\mathbf{(fH)}$.
    \item<6-> $\mathbf{(reg)} \Rightarrow \mathbf{(fH)}$.
    \end{itemize}
\end{block}
    
\end{frame}

\subsection{Una manera diferente de ver $T_2$}
\begin{frame}{Hausdorff punteados}
\framesubtitle{[Rosicky y Smarda (1985)]}

Esta noción sugiere ver $T_2$ como una propiedad relaciona con $T_1$ en marcos.\\

\onslide<2->{Recordemos que 
    \begin{description}
    \item[$(T_1)$] Todo elemento primo en $A$   es máximo.
    \end{description}}

\uncover<3->{\begin{block}{Definición:}
    Decimos que un marco $A$ es \emph{Hausdorff punteado} si cumple la siguiente propiedad
    \begin{description}
        \item[$\mathbf{(Hp)}$] $\quad$Todo elemento semiprimo en $A$ es máximo. 
    \end{description}
\end{block}}
\end{frame}

\begin{frame}
    \begin{block}{Observaciones:}
        \begin{itemize}
            \item $S$ es $T_2$ (espacial) $\Leftrightarrow$ $\mathcal{O}S$ es $\mathbf{(Hp)}$.
            \item<2-> $\mathbf{(H)}\Rightarrow \mathbf{(Hp)}$.
            \item<3-> ¿Qué pasaría si tenemos $\mathbf{(Hp)}+\mathbf{(saju)}$?
            \item<4-> $\mathbf{(Hp)}$ caracteriza a algunos marcos espaciales.
            \item<5-> No existe mas información sobre $\mathbf{(Hp)}$.
        \end{itemize}
    \end{block}
\end{frame}

\begin{frame}[fragile, plain, standout]
    \[\begin{tikzcd}
	&& {\mathbf{(reg)}} \\
	{\mathbf{(aju)}} && {\mathbf{(fH)}} && {\mathbf{(H)}+\mathbf{(saju)}} \\
	{\mathbf{(saju)}} && {T_U} && {\mathbf{(H)}} \\
	&& {T_1}
	\arrow[Rightarrow, from=1-3, to=2-1]
	\arrow[Rightarrow, from=1-3, to=2-3]
	\arrow[Rightarrow, from=1-3, to=2-5]
	\arrow[dotted, no head, from=2-1, to=2-3]
	\arrow[Rightarrow, from=2-1, to=3-1]
	\arrow[Rightarrow, from=2-1, to=3-3]
	\arrow[dotted, no head, from=2-1, to=3-5]
	\arrow[dotted, no head, from=2-3, to=2-5]
	\arrow[dotted, no head, from=2-3, to=3-1]
	\arrow[Rightarrow, from=2-3, to=3-3]
	\arrow[Rightarrow, from=2-3, to=3-5]
	\arrow[Rightarrow, from=2-5, to=3-1]
	\arrow[dotted, no head, from=2-5, to=3-3]
	\arrow[Rightarrow, from=2-5, to=3-5]
	\arrow[dotted, no head, from=3-1, to=3-3]
	\arrow[dotted, no head, from=3-1, to=4-3]
	\arrow[dotted, no head, from=3-3, to=3-5]
	\arrow[Rightarrow, from=3-3, to=4-3]
	\arrow[Rightarrow, from=3-5, to=4-3]
\end{tikzcd}\]

\[\begin{tikzcd}
	&& {(H)} \\
	{\mathbf{(reg)}} & {\mathbf{(fH)}} & {T_U} & {T_1}
	\arrow[Rightarrow, from=2-1, to=2-2]
	\arrow[Rightarrow, from=2-2, to=1-3]
	\arrow[Rightarrow, from=2-2, to=2-3]
	\arrow[Rightarrow, from=2-3, to=2-4]
\end{tikzcd}\]
\end{frame}

\begin{frame}[plain, fragile, standout]
\[\begin{tikzcd}
	&& {\mathbf{(saju)}} \\
	{\mathbf{(reg)}} & {\mathbf{(H)}+\mathbf{(saju)}} & {\mathbf{(H)}} & {T_1}
	\arrow[Rightarrow, from=2-1, to=2-2]
	\arrow[Rightarrow, from=2-2, to=1-3]
	\arrow[Rightarrow, from=2-2, to=2-3]
	\arrow[Rightarrow, from=2-3, to=2-4]
\end{tikzcd}\]

\[\begin{tikzcd}
	&&&& {\mathbf{(saju)}} \\
	{\mathbf{(reg)}} && {\mathbf{(aju)}} && {(T_1)} \\
	&&&& {(T_U)}
	\arrow[Rightarrow, from=2-1, to=2-3]
	\arrow[Rightarrow, from=2-3, to=1-5]
	\arrow[Rightarrow, from=2-3, to=2-5]
	\arrow[Rightarrow, from=2-3, to=3-5]
\end{tikzcd}\]

    
\end{frame}

\section{¿Cuál noción Hausdorff es mejor?}

\begin{frame}[standout]{En resumen}

\begin{center}
\begin{tabular}{| c | c | c | c | c | c |}
\hline
 Axioma/Comportamiento & \textbf{C.} & 1° & 2° & \textbf{S. H.} & \textbf{C. S. E.}\\ \hline
$\mathbf{(dH)}$ & x & $\checkmark$ & x & x & x \\ \hline
$\mathbf{(H)}$ & $\checkmark$ & $\checkmark$ & x & $\checkmark$ & x \\ \hline
$\mathbf{(Hp)}$ & $\checkmark$ & $\checkmark$ & x & $\checkmark$ & ? \\ \hline
$\mathbf{(fH)}$ & x & x &  $\checkmark$ & $\checkmark$ & $\checkmark$ \\ \hline
\end{tabular}
\end{center}
\begin{tiny}
$$\mbox{ }$$
\end{tiny}

\textbf{C.}= Propiedad conservativa\\
\textbf{S. H.}= Suficientemente Huasdorff\\
\textbf{C. S. E.}= Comportamiento similar al espacial


%\begin{definition}
    
%\end{definition}

%\begin{corollary}
    
%\end{corollary}
\end{frame}

\begin{frame}[standout]{Conclusiones}
    \begin{itemize}
        \item $T_2$ (D. y S.) no es suficientemente Hausdorff.
        \item<2-> $\mathbf{(H)}$ podría ser un muy buen acercamiento a $T_2$.
        \item<3-> $\mathbf{(fH)}$ reúne los aspectos espaciales, pero es complicado trabajar con ella.
        \item<4-> $\mathbf{(Hp)}$ necesita ser aun explorada.
        \item<5-> No podemos concluir con certeza cual noción Hausdorff en marcos es ``la mejor''.
    \end{itemize}
\end{frame}

\begin{frame}[plain,standout]
      \centering
      \Huge{\smiley Gracias por su atención\smiley}
      %In combination with \textit{plain}, \\
      %it makes a nice thank-you slide!
      %\vfill
      %\scalebox{4}{\faGithub} \par\bigskip
      %\url{https://github.com/piazzai/arguelles} \\
      %\url{https://ctan.org/pkg/beamertheme-arguelles}
\end{frame}

\End

\section*{\textsc{Referencias}}
\begin{frame}[allowframebreaks]
\frametitle{References}
\begin{thebibliography}{20}\markboth{Bibliografía}{Bibliografía}
\bibitem{D.S.} H. Dowker and D. Strauss, \textit{Separation axioms for frames}. In topics in topology, pp. 223-240. Proc. Colloq., Keszthely, 1992.

\bibitem{R. I.} J. R. Isbell, \textit{Atomless parts of spaces}. Math. Scand. 31 (1972) 5-32. 

\bibitem{P.T.} P. T. Johnstone, \textit{Stone spaces}, Cambridge Studies in Advanced Mathematics, vol. 3, Cambridge University Press, Cambridge, 1982. MR 698074

\bibitem{J.S.} P.T. Johnstone, S.-H. Sun, \textit{Weak products and Hausdorff locales}. In: Categorical algebra and its applications, pp. 173-193. Lecture notes in mathematics, vol. 1348. Springer-Verlag, Berlin, 1988.

\bibitem{J.M.} J. Monter; A. Zaldívar, \textit{El enfoque locálico de las reflexiones booleanas: un análisis en la categoría de marcos} [tesis de maestría], 2022. Universidad de Guadalajara.

\bibitem{P.S.} J. Paseka, B. Smarda, \textit{$T_2$-frames and almost compact frames}. Czechoslovak Math. J. 42 (1992) 297-313.

\bibitem{J.P.} J. Picado and A. Pultr, \textit{Frames and locales: Topology without points}, Frontiers in Mathematics, Springer Basel, 2012.

\bibitem{J.P.2} J. Picado and A. Pultr, \textit{Separation in point-free topology}, Springer, 2021.

\bibitem{R.S.} J. Rosicky, B. Smarda, \textit{$T_1$-locales.} Math. Proc. Cambridge Philos. Soc. 98 (1985) 81-86.

%\bibitem{R.S.} Rosemary A Sexton, \textit{A point free and point-sensitive analysis of the patch assembly}, The University of Manchester (United Kingdom), 2003.

%\bibitem{R.S.3} RA Sexton and H. Simmons, \textit{Point-sensitive and point-free patch constructions}, Journal of Pure and Applied Algebra \textbf{207} (2006), no. 2, 433-468.

\bibitem{H.S.3} Harold Simmons, \textit{The assembly of a frame}, University of Manchester (2006).

\bibitem{H.S.4} Harold Simmons, \textit{The lattice theoretic part of topological separation properties}, Proceedings of the Edinburgh Mathematical Society (1978), vol. 21, no 1, p. 41-48.

\bibitem{A.Z.} A. Zaldívar, \textit{Introducción a la teoría de marcos} [notas curso], 2024. Universidad de Guadalajara.

\end{thebibliography}
\end{frame}

\begin{frame}[plain, standout]
    \begin{center}
    \includegraphics[height=9.5cm]{tacos.jpg}
    \end{center}
\end{frame}


\end{document}