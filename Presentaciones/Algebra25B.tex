\documentclass[compress,12pt]{beamer}

\usetheme{Arguelles}
\usepackage{graphicx}
\usepackage{caption}
\usepackage[spanish,es-noshorthands]{babel}
%\usepackage{babel} 
\usepackage[pages=some]{background}
\usepackage{tikz}
\usepackage{tikz-cd}
\usepackage{amsmath,amssymb,latexsym,amscd} 
\usepackage[all,cmtip]{xy}
\usepackage{fancyhdr}
\usepackage{mathalfa}
\usepackage{mathrsfs}
\usetikzlibrary{babel}
\usepackage{hyperref}
\usepackage{ragged2e}
\usepackage{wasysym}
\usepackage{tikz}
\usetikzlibrary{arrows.meta,calc,positioning,overlay-beamer-styles,shadows.blur}
%\hypersetup{colorlinks=true,linkcolor=blue,citecolor=brown,linktocpage=true,pagebackref=true,hyperindex=true}
\pagenumbering{arabic}

\DeclareMathOperator{\op}{op}
\DeclareMathOperator{\pt}{pt}
\DeclareMathOperator{\spec}{spec}
\DeclareMathOperator{\Fit}{Fit}
\DeclareMathOperator{\Pth}{P}
\DeclareMathOperator{\Frm}{Frm}
\DeclareMathOperator{\Top}{Top}
\DeclareMathOperator{\Ord}{Ord}
\DeclareMathOperator{\Obj}{Obj}
\DeclareMathOperator{\Hom}{Hom}

\title{¿Para qué son eficientes los marcos eficientes?}
%\subtitle{y algunos axiomas de separación en la topología sin puntos}
\event{Seminario de Álgebra, CUCEI}
\date{\today}
\author{Juan Carlos Monter Cortés}
\institute{Universidad de Guadalajara}
\email{juan.monter2902@alumnos.udg.mx}

%\homepage{www.mywebsite.com}
\github{JCmonter}

\begin{document}

%\frame[plain]{\titlepage}

\begin{frame}[plain]
\titlepage

\vfill
\begin{center}
    \textcolor{white}{\scriptsize Investigación apoyada por SECIHTI-PROYECTO CBF2023-2024-2630}
\end{center}

\end{frame}

\begin{frame}{Contenido}
\tableofcontents
\end{frame}

\section{Construcciones de parches}
\begin{frame}{Un poco de historía}
\centering
\resizebox{\textwidth}{!}{%
\begin{tikzpicture}[
  x=1cm, y=1cm, % <-- un poquito más grande en x
  >=Stealth,
  year/.style={font=\small\bfseries},
  tick/.style={circle, fill, minimum size=3pt, inner sep=0pt},
  axis/.style={line width=0.6pt, gray!60}
]

% --- Eje base (mismo espacio entre nodos)
\draw[axis] (0,0) -- (17,0);

% --- Coordenadas EQUIESPACiADAS
\coordinate (p68) at (1,0);
\coordinate (p82) at (4,0);
\coordinate (p96) at (7,0);
\coordinate (p00) at (10,0);
\coordinate (p13) at (13,0);
\coordinate (pMi) at (16,0); % MITAC

% --- Años (1968, 1996, 2013 abajo; 1982, 2000, 2024 arriba)
\node[tick, visible on=<1->] at (p68) {};
\node[year, visible on=<1->] at ($(p68)+(0,-0.6)$) {1968};

\node[tick, visible on=<2->] at (p82) {};
\node[year, visible on=<2->] at ($(p82)+(0,0.7)$) {1982};

\node[tick, visible on=<3->] at (p96) {};
\node[year, visible on=<3->] at ($(p96)+(0,-0.6)$) {1996};

\node[tick, visible on=<4->] at (p00) {};
\node[year, visible on=<4->] at ($(p00)+(0,0.7)$) {2000};

\node[tick, visible on=<5->] at (p13) {};
\node[year, visible on=<5->] at ($(p13)+(0,-0.6)$) {2013};

\node[tick, visible on=<6->] at (pMi) {};
\node[year, visible on=<6->] at ($(pMi)+(0,0.7)$) {2025}; % cambia el año si quieres

% --- Imágenes alternadas (más GRANDES)

% 1968 - Hochster (arriba)
\node[visible on=<1->, anchor=south] (e1968) at ($(p68)+(0,1.5)$)
  {\includegraphics[width=4cm]{Imagenes/Hochster.jpg}}; % o .png / .jpeg

% 1982 - Stone (abajo)
\node[visible on=<2->, anchor=north] (e1982) at ($(p82)+(-0,-1.5)$)
  {\includegraphics[width=3cm]{Imagenes/Stone.png}};

% 1996 - Escardó (arriba)
\node[visible on=<3->, anchor=south] (e1996) at ($(p96)+(0,1.5)$)
  {\includegraphics[width=4cm]{Imagenes/Escardo.jpg}};

% 2000 - Rosy (abajo)
\node[visible on=<4->, anchor=north] (e2000) at ($(p00)+(0,-1.5)$)
  {\includegraphics[width=3cm]{Imagenes/Rosy.jpg}};
% 2013 - Klinke (arriba, silueta)
\node[visible on=<5->, anchor=south] (e2013) at ($(p13)+(0,1.5)$)
  {\includegraphics[width=6cm]{Imagenes/Portada.png}};

% 2024 - MITAC (abajo, la última un poco más abajo para lucir)
\node[visible on=<6->, anchor=north] (eMITAC) at ($(pMi)+(0,-1.6)$)
  {\includegraphics[width=7cm]{Imagenes/MITAC.jpeg}};

% --- Flechas verticales
\draw[->, thick, visible on=<1->] (e1968.south) -- ($(p68)+(0,0.06)$);
\draw[->, thick, visible on=<2->] (e1982.north) -- ($(p82)+(0,-0.06)$);
\draw[->, thick, visible on=<3->] (e1996.south) -- ($(p96)+(0,0.06)$);
\draw[->, thick, visible on=<4->] (e2000.north) -- ($(p00)+(0,-0.06)$);
\draw[->, thick, visible on=<5->] (e2013.south) -- ($(p13)+(0,0.06)$);
\draw[->, thick, visible on=<6->] (eMITAC.north) -- ($(pMi)+(0,-0.06)$);

\end{tikzpicture}%
}
\end{frame}

\begin{frame}[fragile]{Espacio de parches}
    Consideremos $S\in \Top$. Denotamos por $^pS=(S, \mathcal{O}^pS)$ al \textbf{espacio de parches} de $S$, donde $\mathcal{O}^pS$ está generado por
    \[
    \mbox{pbase}=\{U\cap Q'\mid U\in \mathcal{O}S, Q\in \mathcal{Q}S\mbox{\footnote{Decimos que $Q\subseteq S$ es saturado si $Q=\bigcap\{V\in \mathcal{O}S\mid Q\subseteq V\}$.\\ 
    $\mathcal{Q}S=$ Conjuntos compactos saturados de $S$.}}\}
    \]
    \[\begin{tikzcd}
\mathcal{O}S \arrow[r, hook] & \mathcal{O}^pS
\end{tikzcd}\]

\onslide<2->{\begin{block}{Definición}
$S$ es \textbf{empaquetado} si todo subconjunto compacto (saturado) es cerrado
\[
S \mbox{ es empaquetado}\iff\, ^pS=S\quad \mbox{ y }\quad T_2\Rightarrow \mbox{empaquetado}\Rightarrow T_1
\]
\end{block}}
\end{frame}

\subsection{Teoría de marcos}
\begin{frame}[fragile]{Teoría de marcos}
\[
\Frm= \left\{ \begin{array}{lc} \Obj: & (A, \leq, \wedge, \bigvee, 1, 0) \\ \\ \mbox{Flechas:} & f\colon A\to B  \end{array} \right.
\]

Para $S\in \Top$,
\[
(\mathcal{O}S, \subseteq, \cap, \bigcup, S, \emptyset)\in \Frm
\]
Además,
\[
\begin{tikzcd}
\Top \arrow[rr, "\mathcal{O}( \_ )", bend left] & \perp & \Frm \arrow[ll, "\pt( \_ )", bend left]
\end{tikzcd}
\]
es una adjunción, es decir, $\Hom_{\Top}(\pt A, S)\cong \Hom_{\Frm}(A, \mathcal{O}S)$.
\end{frame}

\begin{frame}{Núcleos y cocientes}
\begin{block}{Definición}
Sea $j\colon A\to A$. Decimos que $j$ es un \textbf{núcleo} si para todo $a, b\in A$ se cumplen:
\scriptsize{
\[
  i)\, a\leq j(a)\quad ii)\, a\leq b\Rightarrow j(a)\leq j(b)\quad iii)\, j^2(a)=j(a)\quad iv)\; j(a\wedge b)=j(a)\wedge j(b)
\]}
\end{block}
\onslide<2->{$NA=\{\mbox{núcleos en }A\}\in \Frm$.\\}
\onslide<3->{Si $a\in A$, definimos 
\[
u_a(x)=a\vee x\quad v_a(x)=(a\succ x)\quad w_a(x)=((x\succ a)\succ a)
\]
y $u_a, v_a, w_a\in NA$}. \onslide<4->{Además, $\eta_A\colon A\to NA$ dada por $\eta_A(a)=u_a$}.
\end{frame}

\begin{frame}
\begin{block}{Definición}
Sea $A\in \Frm$. Un \textbf{cociente} de $A$ es un marco $B$ y un morfismo 
\[
f\colon A\to B
\]
suprayectivo.
\end{block}
\onslide<2->{Sea $A_j=\{a\in A\mid j(a)=a\}$. Si $j\in NA$,
\[
j\colon A\to A_j \mbox{ es suprayectivo}\quad \mbox{ y }\quad A_j\in \Frm
\]
$A_j$ es el \emph{marco cociente}}. \onslide<3->{En particular,
\[
A_{u_a}=\mbox{c. cerrado}, \quad A_{v_a}=\mbox{c. abierto},\quad A_{w_a}=\mbox{c. regular}.
\]}
\end{frame}

\begin{frame}{Filtros}
  \begin{block}{Definición}
    Sea $A\in \Frm$. Decimos que $F\subseteq A$ es un \textbf{filtro} si:
    \begin{enumerate}
      \item $1\in F$.
      \item Si $a\leq b$ y $a\in F$, entonces $b\in F$.
      \item Si $a, b\in F$, entonces $a\wedge b\in F$.
    \end{enumerate}
  \end{block}
  \onslide<2->{En particular, decimos que $F$ es un \textbf{filtro abierto} si: 
  \[
    X\subseteq A \mbox{ dirigido tal que }\bigvee X\in F\Rightarrow F\cap X\neq \emptyset.
  \]
  $A^\wedge=$Filtros abiertos en $A$}.
\end{frame}

\begin{frame}{Filtros de admisibilidad}
\begin{block}{Definición}
Sean $A\in \Frm$ y $j\in NA$. 
\begin{enumerate}
  \item Un filtro es \textbf{admisible} si tiene la forma
  \[
    \nabla(j)=\{a\in A\mid j(a)=1\}.
  \]
  \item<2-> Para $j, k\in NA$ definimos 
  \[
    j\sim k\iff \nabla(j)=\nabla(k)
  \]
  \item<3-> Decimos que $f\in [\; j \;]$ es un \textbf{núcleo ajustado} si para todo $k\in [\; j \;]$, $f\leq k$. \onslide<4->{Equivalentemente
  \[
    f \mbox{ es ajustado} \iff f=\bigvee\{v_a\mid a\in F=\nabla(j)\}.
  \]}
\end{enumerate}
\end{block}
\end{frame}

\begin{frame}[fragile]{Axiomas de separación en Frm\footnotemark[2]}
\[\begin{tikzcd}
	&& {\mathbf{(reg)}} \\
	{\mathbf{(aju)}} && {\mathbf{(fH)}} && {\mathbf{(H)}+\mathbf{(saju)}} \\
	{\mathbf{(saju)}} && {T_1} && {\mathbf{(H)}}
	\arrow[Rightarrow, from=1-3, to=2-1]
	\arrow[Rightarrow, from=1-3, to=2-3]
	\arrow[Rightarrow, from=1-3, to=2-5]
	\arrow[Rightarrow, from=2-1, to=3-1]
	\arrow[Rightarrow, from=2-1, to=3-3]
	\arrow[dotted, no head, from=2-3, to=2-1]
	\arrow[dotted, no head, from=2-3, to=2-5]
	\arrow[dotted, no head, from=2-3, to=3-1]
	\arrow[Rightarrow, from=2-3, to=3-3]
	\arrow[Rightarrow, from=2-3, to=3-5]
	\arrow[Rightarrow, from=2-5, to=3-5]
	\arrow[dotted, no head, from=3-1, to=3-3]
	\arrow[Rightarrow, from=3-5, to=3-3]
\end{tikzcd}\]
\footnotetext[2]{Para cualesquiera $a\nleq b\in A$ tenemos que $A$ es:\\ 
\textbf{(reg)} si $\exists \, x,y\in A$ tales que $a\vee x=1, y\nleq b$ y $x\wedge y=0$.\\
\textbf{(H)} si $\exists\, c\in A$ tal que $c\nleq a$ y $\neg c\leq b$.\\
\textbf{(aju)} si $\exists\, x,y\in A$ tales que $x\vee a=1, y\nleq b$ y $x\wedge y\leq b$.\\
\textbf{(saju)} si $\exists\, c\in A$ tal que $c\vee a=1\neq c\vee b$.\\
\textbf{(fH)} y $T_1$ son nociones algo diferentes. Todas estas pueden verse en \cite{J.P.2}.}
\end{frame}

\section{Marco de parches}

\begin{frame}{El Teorema de Hoffman-Mislove}
\begin{block}{Proposición}
\begin{enumerate}
  \item Si $F\in A^\wedge$, entonces $F=\nabla(j)$.
  \item $F=\nabla(j)\in A^\wedge$ si y solo si $A_j$ es compacto\footnotemark[3]. 
  \item $F=\nabla(j)\in A^\wedge$ si y solo si para todo $k\in [\; j \;]$ se tiene que $v_F\leq k$, donde $v_F=f_F^\infty$.
\end{enumerate}
\end{block}

  \only<2>{
    \begin{block}{Teorema [Hoffman-Mislove]}
      Sea $A\in \Frm$ y $S=\pt A\in \Top$. Existe una correspondencia biyectiva entre $Q\in \mathcal{Q}S$ y $F\in A^\wedge$.
    \end{block}
    \[
      Q\longleftrightarrow F
    \]
  }

  % Segunda versión (extendida)
  \only<3>{
    \begin{block}{Teorema [Hoffman-Mislove (extendido)]}
      Sea $A\in \Frm$ y $S=\pt A\in \Top$. Existe una correspondencia biyectiva entre $Q\in \mathcal{Q}S$ y núcleos ajustados.
    \end{block}
    \[
      Q\longleftrightarrow F\longleftrightarrow v_F
    \]}
  
  \footnotetext[3]{$A\in \Frm$ es compacto si y solo si para $X\subseteq A$, $1=\bigvee X$.}
\end{frame}

\subsection{Parche trivial}
\begin{frame}[fragile]{Marco de parches}
Consideremos $A\in \Frm$. Denotamos por $PA$ al \textbf{marco de parches} de $A$, donde $PA$ está generado por
\[
\mbox{Pbase}=\{u_a\wedge v_F\mid a\in A, F\in A^\wedge\}
\]
\[
\begin{tikzcd}
A \arrow[r] \arrow[rr, "\eta_A", bend left] & PA \arrow[r] & NA
\end{tikzcd}
\]

\onslide<2->{\begin{block}{Definición}
    $A$ es \textbf{parche trivial} si y solo si $A\simeq PA$. \only<3->{\alert<3->{¿Cuándo sucede esto?}}
      \[
        ?\Rightarrow \mbox{ parche trivial }\Rightarrow T_1
      \]
\end{block}}
\end{frame}

\begin{frame}{El diccionario}

\begin{columns}[T,onlytextwidth]
  \column{0.48\textwidth}
  \begin{block}{\centering $\Top$}
    \begin{itemize}
      \item<2-> Espacio de parches ($^pS$)
      \item<4-> pbase$=\{U\cap Q'\}$ 
      \item<6-> Empaquetado ($^pS=S$)
      \item<8-> $Q\in \mathcal{Q}S\Rightarrow Q\in \mathcal{C}S$
      \item<10-> $T_2\Rightarrow $Empaquetado
      \item<12-> Subespacios compactos cerrados
    \end{itemize}
  \end{block}

  \column{0.48\textwidth}
  \begin{block}{\centering $\Frm$}
    \begin{itemize}
      \item<3-> Marco de parches ($PA$)
      \item<5-> Pbase$=\{u_a\wedge v_F\}$
      \item<7-> Parche trivial ($PA\cong A$)
      \item<9-> $u_a=v_F$
      \item<11-> ¿$\mathbf{(H)}\Rightarrow $Parche trivial?
      \item<13-> \alert<16->{Cocientes compactos cerrados.}
    \end{itemize}
  \end{block}
\end{columns}
\end{frame}

\begin{frame}{Parche trivial vs otras propiedades}
  \begin{block}{Proposición [Caso espacial ($A=\mathcal{O}S$)]}
    Sea $S\in \Top$. Las siguientes afirmaciones se cumplen:
    \begin{enumerate}
      \item Si $S$ es $T_2$, entonces $\mathcal{O}S$ es parche trivial.
      \item Si $S$ es $T_1$, sobrio y empaquetado, entonces $\mathcal{O}S$ es parche trivial.
    \end{enumerate}
    \end{block}

    \onslide<2->{
    \begin{block}{Proposición [Caso general]}
      Sea $A\in \Frm$. Las siguientes afirmaciones se cumplen:
      \begin{enumerate}
        \item Si $A$ cumple $\mathbf{(reg)}$ y $j\in NA$. Si $\nabla(j)\in A^\wedge$, entonces $j=u_a$, donde $a=j(0)$.
        \item Si $A$ cumple $\mathbf{(aju)}$, entonces $j\in NA$ es ajustado.
      \end{enumerate}
    \end{block}}
    \end{frame}

\begin{frame}{Comparando $u_a$ y $v_F$}
\begin{block}{Proposición}
Si $A\in \Frm$ y $j\in NA$, las siguientes se cumplen:
\begin{enumerate}
  \item<2-> $u_a\leq j \iff a\leq j(0)$
  \item<3-> $v_a\leq j \iff j(a)=1\iff a\in \nabla(j)$.
\end{enumerate}
\end{block}

\onslide<4->{Para $f_F=\dot{\bigvee}\{v_a\mid a\in F=\nabla(j)\}$, tenemos
\[
f^0=id,\quad f^{\alpha+1}=f\circ f^\alpha,\quad f^\lambda=\bigvee\{f^\alpha\mid\alpha<\lambda\}.
\]
Así $v_F=f^\infty$}. \onslide<5->{De manera similar, 
\[
d(0)=0,\quad d(\alpha+1)=f(d(\alpha)),\quad d(\lambda)=\bigvee\{d(\alpha)\mid \alpha <\lambda\}.
\]}
\onslide<6->{Si $d=f^\infty(0)=v_F(0)$, se cumple que 
$u_d\leq v_F$.} \onslide<7->{\alert<7->{¿Cuándo $v_F\leq u_d$?}} 
\end{frame}


\section{Marcos eficientes}

\begin{frame}{Marcos eficientes}
\begin{block}{Definición [\cite{R.S.3}, Def. 8.2.1]}
Sean $A\in \Frm$, $F\in A^\wedge$ y $\alpha\in \Ord$. Decimos que:
\begin{enumerate}
    \item<2-> $F$ es $\alpha$\textbf{-eficiente} si para $x\in F$, $d(\alpha)\vee x=1$, donde
    \[
    d(\alpha)=f^\alpha(0).
    \]
    \item<3-> $A$ es $\alpha$\textbf{-eficiente} si cada $F\in A^\wedge$ es $\alpha$-eficiente.
    \item<4-> $A$ es \textbf{eficiente} si es $\alpha$-eficiente para algún $\alpha\in \Ord$.
\end{enumerate}
\end{block}

\onslide<5->{\begin{block}{Proposición [\cite{R.S.3}, Lema 8.2.2]}
\[
    A \mbox{ es eficiente}\quad \iff\quad A \mbox{ es parche trivial}.
\]
\end{block}}
\end{frame}


\begin{frame}{Propiedades de los marcos eficientes}
Este es un resumen de las propiedades que Sexton menciona en \cite{R.S.3}
\begin{itemize}
\item<2-> En el caso espacial ($A=\mathcal{O}S$),
\[
\begin{split}
\mathcal{O}S \mbox{ es 0-eficiente } & \iff S=\emptyset \\
\mathcal{O}S \mbox{ es 1-eficiente } & \iff S \mbox{ es }T_2 \\
\mathcal{O}S \mbox{ es eficiente } & \iff S \mbox{ es empaquetado y apilado}.
\end{split}
\]

\item<3-> Para $A\in \Frm$ arbitrario
\[
\begin{split}
A\mbox{ es }\mathbf{(reg)} & \Rightarrow A \mbox{ es eficiente}\\
A\mbox{ es }\mathbf{(aju)} & \Rightarrow A \mbox{ es eficiente}\\
A\mbox{ es eficiente } & \Rightarrow A\mbox{ es } T_1
\end{split}
\]
\end{itemize}
\end{frame}

\begin{frame}{¿Qué significa ser $\alpha$-eficiente?}
\[
1-\mbox{Eficiente}\Rightarrow 2-\mbox{Eficiente}\Rightarrow \cdots \Rightarrow \alpha-\mbox{Eficiente}\Rightarrow \cdots \Rightarrow T_1
\]

\onslide<2->{
\begin{block}{Proposición}
Sean $A\in \Frm$ y $j, k\in NA$. 
\begin{enumerate}
  \item<3-> Si $j\leq k$, entonces $\nabla(j)\subseteq \nabla(k)$.
  \item<4-> Si $j$ es ajustado, se cumple que 
\[
j\leq k\iff \nabla(j)\subseteq \nabla(k).
\]
\end{enumerate}
\end{block}}
\end{frame}

\begin{frame}
Sean $A\in \Frm$ y $F\in A^\wedge$.
\begin{itemize}
\item<2-> Si $A$ es 1-eficiente, para $x\in F$ se cumple que $x\vee d(1)=1$, es decir,
\onslide<3->{\[
\begin{split}
u_{d(1)}(x)=1& \Rightarrow x\in \nabla(u_{d(1)})\Rightarrow \nabla(v_F)=F\subseteq \nabla(u_{d(1)})\\
& \Rightarrow v_F\leq u_{d(1)}\\
& \Rightarrow v_F(0)=d(\infty)\leq u_{d(1)}=d(1).\\
& \Rightarrow d(\infty)= d(1).
\end{split}
\]}
\onslide<4->{Además, 
\[
u_{d(1)}\leq u_d\leq v_F\Rightarrow \nabla(u_{d(1)})\subseteq \nabla(u_d)\subseteq \nabla(v_F)=F.
\]}
\end{itemize}
\onslide<5->{Por lo tanto $\nabla(u_{d(1)})=F$}.
\end{frame}

\begin{frame}{La pregunta inicial}
  \alert{¿Para qué son eficientes los marcos eficientes?}
\begin{itemize}
\item<2-> Bajo eficiencia, la sucesión $d(\alpha)$ se estabiliza en el grado de eficiencia.
\item<3-> Proporciona el menor núcleo cerrado que admite a un filtro abierto.
\item<4-> Para cada filtro abierto $F$, proporciona un cociente compacto cerrado.  
\end{itemize}
\onslide<5->{Lo ``malo...:'': el grado de eficiencia puede ser arbitrariamente grande \Large{\frownie{}}}
\end{frame}

\begin{frame}{Más propiedades de los marcos eficientes}
\begin{block}{Proposición}
Si $A\in \Frm$ es eficiente y $j\in NA$, entonces $A_j$ es eficiente.
\end{block}

\begin{block}{Proposición}
Si $\{A_i\}_{i\in I}$ es una familia de marcos eficientes, entonces $\bigoplus_{i\in I}A_i$ es un marco eficiente.
\end{block}

\end{frame}

\begin{frame}[fragile]
\begin{block}{Corolario}
Si $A$ es $\mathbf{(fH)}$, $A$ es eficiente.
\end{block}

\[\begin{tikzcd}
	& {\mathbf{(H)}} \\
	{\mathbf{(reg)}} & {\mathbf{(fH)}} & {\text{Eficiente}} \\
	& {\mathbf{(aju)}}
	\arrow[Rightarrow, dotted, from=1-2, to=2-3]
	\arrow[Rightarrow, from=2-1, to=1-2]
	\arrow[Rightarrow, from=2-1, to=2-2]
	\arrow[Rightarrow, from=2-1, to=3-2]
	\arrow[Rightarrow, from=2-2, to=1-2]
	\arrow[Rightarrow, from=2-2, to=2-3]
	\arrow[Rightarrow, from=3-2, to=2-3]
\end{tikzcd}\]
\end{frame}

\subsection{Ejemplos}
\begin{frame}{Algunos ejemplos}
\begin{itemize}
\item Con la topología cofinita vemos que $PA=NA$.
\item<2-> Con la topología conumerable vemos que $\pt NA\subseteq \pt PA$.
\[
 \alert{\mbox{En ambos casos }A \mbox{ no es } 1\mbox{-eficiente y } PA \mbox{ es espacial.}}
\] 
\item<3-> La topología subregular de los reales proporciona un marco 1-eficiente que no es regular. 
\[
\alert{1\mbox{-eficiente}\nRightarrow \mathbf{(reg)}}
\]
\item<4-> Con la topología máximo compacta tenemos un marco 2-eficiente que no es 1-eficiente.
\[
\alert{2\mbox{-eficiente}\nRightarrow \mathbf{(H)}}
\]
\item<5-> La topología guía sobre un árbol muestra que existen marcos $\omega$-eficientes.
\[
\alert{\omega\mbox{-eficiente}\nRightarrow n\mbox{-eficiente para algún } n\in \mathbb{N}}
\]
\end{itemize}
\end{frame}

\End

\section*{\textsc{Referencias}}
\begin{frame}[allowframebreaks]
\frametitle{Bibliografía}
\begin{thebibliography}{20}\markboth{Bibliografía}{Bibliografía}

\bibitem{P.T.} P. T. Johnstone, \textit{Stone spaces}, Cambridge Studies in Advanced Mathematics, vol. 3, Cambridge University Press, Cambridge, 1982. MR 698074

%\bibitem{J.M.} J. Monter; A. Zaldívar, \textit{El enfoque locálico de las reflexiones booleanas: un análisis en la categoría de marcos} [tesis de maestría], 2022. Universidad de Guadalajara.

%\bibitem{P.S.} J. Paseka and B. Smarda, \textit{$ T_2 $-frames and almost compact frames.} Czechoslovak Mathematical Journal (1992), 42(3), 385-402.

\bibitem{J.P.} J. Picado and A. Pultr, \textit{Frames and locales: Topology without points}, Frontiers in Mathematics, Springer Basel, 2012.

\bibitem{J.P.2} J. Picado and A. Pultr, \textit{Separation in point-free topology}, Springer, 2021.

\bibitem{R.S.} RA Sexton, \textit{A point free and point-sensitive analysis of the patch assembly}, The University of Manchester (United Kingdom), 2003.

\bibitem{R.S.2} RA Sexton, \textit{Frame theoretic assembly as a unifying construct}, The University of Manchester (United Kingdom), 2000.

\bibitem{R.S.3} RA Sexton and H. Simmons, \textit{Point-sensitive and point-free patch constructions}, Journal of Pure and Applied Algebra \textbf{207} (2006), no. 2, 433-468.

\bibitem{H.S.} H. Simmons, \textit{An Introduction to Frame Theory}, lecture notes, University of Manchester. Disponible en línea en \url{https://web.archive.org/web/20190714073511/http://staff.cs.manchester.ac.uk/~hsimmons}.

\bibitem{H.S.R} H. Simmons, \textit{Regularity, fitness, and the block structure of frames.} Applied Categorical Structures 14 (2006): 1-34.

\bibitem{H.S.4} H. Simmons, \textit{The lattice theoretic part of topological separation properties}, Proceedings of the Edinburgh Mathematical Society, vol.~21, pp.~41--48, 1978.

\bibitem{H.S.V} H. Simmons, \textit{The Vietoris modifications of a frame}. Unpublished manuscript (2004), 79pp., available online at http://www. cs. man. ac. uk/hsimmons.

%\bibitem{A.W.} A. Wilansky, \textit{Between T1 and T2}, MONTHLY (1967): 261-266.

\bibitem{A.Z.} A. Zaldívar, \textit{Introducción a la teoría de marcos} [notas curso], 2025. Universidad de Guadalajara.
\end{thebibliography}
\end{frame}

%\begin{frame}[plain, standout]
%\centering

%{\Huge \bfseries ¡Muchas gracias!}\\[1cm]

%{\Large
%\textit{Preguntas, comentarios o sugerencias} \\[1cm]
%\texttt{juan.monter2902@alumnos.udg.mx}
%}
%\end{frame}

\begin{frame}[plain,standout]
      \centering
      \Huge{\bfseries \smiley ¡Muchas gracias! \smiley}\\[1cm]
      %In combination with \textit{plain}, \\
      %it makes a nice thank-you slide!
      \scalebox{4}{\faGithub} \par\bigskip
      \url{https://github.com/JCmonter/Apuntes/tree/main/Presentaciones} \\
      %\url{https://ctan.org/pkg/beamertheme-arguelles}
\end{frame}

\appendix
\section*{Apéndice: Definiciones de respaldo}

% --- Orden parcial ---
\begin{frame}{La topología subregular de $\mathbb{R}$}
  Consideramos $S=\mathbb{R}$ con la topología generada por la base
  \[
    \mathcal{O}_{sr}S=\{U\cup (\mathbb{Q}\cap V)\mid U, V\in \mathcal{O}S\}.
  \]
  y $\mathcal{O}S$ es la topología métrica.
  \begin{itemize}
    \item $\mathcal{O}S\subseteq \mathcal{O}_{sr}S$.
    \item $(S, \mathcal{O}_{sr}S)$ es $T_2$, entonces $\mathcal{O}_{sr}S$ es 1-eficiente.
    \item $\mathcal{O}S_{sr}$ no es ajustado.
  \end{itemize}
\end{frame}

\begin{frame}{Topología máximo compacta}
  Sea $S=\{x, y\} \cup \mathbb{N}^2$, donde $(x, y)\notin \mathbb{N}^2$ y $x\neq y$. Definimos 
  \[
    R_n=\{(m,n)\mid m\in \mathbb{N}\}.
  \]
  Tomamos 
  \[
    \mathcal{O}S=\mathcal{P}\mathbb{N}^2\cup \mathcal{U}_x\cup \mathcal{V}_y,
  \]
  donde 
  \[
    \mathcal{U}_x=\{U\subseteq S\mid x\in U \mbox{ y }U\cap R_n \mbox{ es cofinito }\forall n\in \mathbb{N}\}, 
  \]
  y
  \[
    \mathcal{V}_y=\{V\subseteq S\mid y\in V \mbox{ y }R_n\subseteq V \mbox{ para casi todo } n\in \mathbb{N}\}.
  \]
\end{frame}

\begin{frame}{Topología guía sobre un árbol}
 $\mathbb{T}$ es un árbol si para cada nodo $x$ el conjunto 
\[
  \mathcal{P}(x)=\{y\in \mathbb{T}\mid y\leq x\}
\]
es linealmente ordenado. Denotamos por 
\[
  I(x)=\{y\in \mathbb{T}\mid y \mbox{ es sucesor inmediato de } x\}.
\]
Para $*\notin \mathbb{T}$, definimos $S=\mathbb{T}\cup \{*\}$ y la familia $\mathcal{O}S$ de subconjuntos $U\subseteq S$ tales que:
\begin{itemize}
  \item $(\forall x\in \mathbb{T})[x\in U\Rightarrow I(x)\setminus U \mbox{ es numerable}]$.
  \item $*\in U\Rightarrow (\forall x\in \mathbb{T})[I(x)\setminus U \mbox{ es numerable}]$.
\end{itemize}
$\mathcal{O}S$ es una topología sobre $S$, llamada \textbf{topología guía} sobre el árbol $\mathbb{T}$.
\end{frame}

\end{document}