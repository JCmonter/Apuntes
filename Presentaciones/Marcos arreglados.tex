\documentclass[compress,12pt]{beamer}

\usetheme{Arguelles}
\usepackage{graphicx}
\usepackage{caption}
%\usepackage[spanish,es-noshorthands]{babel}
\usepackage[spanish]{babel} 
\usepackage[pages=some]{background}
\usepackage{tikz-cd}
\usepackage{amsmath,amssymb,latexsym,amscd} 
\usepackage[all,cmtip]{xy}
\usepackage{fancyhdr}
\usepackage{mathalfa}
\usepackage{mathrsfs}
\usetikzlibrary{babel}
\usepackage{hyperref}
\usepackage{ragged2e}
\usepackage{wasysym}
\usepackage{multicol}
%\hypersetup{colorlinks=true,linkcolor=blue,citecolor=brown,linktocpage=true,pagebackref=true,hyperindex=true}
\pagenumbering{arabic}

\DeclareMathOperator{\op}{op}
\DeclareMathOperator{\pt}{pt}
\DeclareMathOperator{\spec}{spec}
\DeclareMathOperator{\Fit}{Fit}
\DeclareMathOperator{\Pth}{P}
\DeclareMathOperator{\Frm}{Frm}
\DeclareMathOperator{\Obj}{Obj}
\DeclareMathOperator{\Hom}{Hom}

\title{Cocientes compactos vs marcos arreglados}
%\event{Seminario de Álgebra, CUCEI}
\date{\today}
\author{Juan Carlos Monter Cortés}
\institute{Universidad de Guadalajara}
\email{juan.monter2902@alumnos.udg.mx}

%\homepage{www.mywebsite.com}
%\github{username}

\begin{document}

\frame[plain]{\titlepage}

\begin{frame}{Contenido}
\tableofcontents %Imprime la tabla de contenido
\end{frame}

\begin{frame}{Por si acaso...}
    \begin{itemize}
        \begin{multicols}{2}
            \item $A$
            \item $(A, \leq)$
            \item $(A, \leq, \vee, 0)$ o $(A, \leq, \wedge, 1)$
            \item $(A, \leq, \bigvee, \bigwedge, 0, 1)$
        \end{multicols} 
        \end{itemize}
    
        \onslide<2->{Un marco es una retícula completa que cumple cierta ley distributiva (ley distributiva de marcos), es decir,}
        \onslide<3->{\[
        (A, \leq, \bigvee, \wedge, 0, 1), \quad a\wedge\bigvee X=\bigvee \{a\wedge x\mid x\in X\}
        \]}
    
        \onslide<4->\[
        \mathbf{Frm}=\left\{ \begin{array}{ll} A, & \mbox{ marcos}\\ \\  f, & \mbox{ morfismo de marcos} \end{array} \right.
        \]
\end{frame}

\begin{frame}{¿Por qué estudiamos los marcos?}
    \begin{itemize}
        \item Estructuras simples.
        \item<2-> Existen herramientas que facilitan el estudio de los marcos.
        \item<3-> Correspondencias biyectivas.
        \item<4-> Buen comportamiento categórico.
        \item<5-> \alert<7->{La topología de un espacio ($\mathcal{O}S)$ es un marco.}
        \item<6-> $\mathbf{Loc}=\mathbf{Frm}^{\op}$ está en relación con $\mathbf{Top}$.
    \end{itemize}
\end{frame}

\section{Cocientes compactos}
\begin{frame}{Cocientes en $\mathbf{Frm}$}
$\mathbf{Frm}$ proporciona correspondencias biyectivas interesantes
\onslide<2->{\[
\mbox{Congruencias}\leftrightarrow \mbox{Conjuntos implicativos}\leftrightarrow \mbox{\alert<3->{Núcleos}}
\]}

\onslide<4->{\begin{block}{Definición:}
Sea $A\in \mathbf{Frm}$ y $j\colon A\to A$, decimos que $j$ es un \emph{núcleo} si:
\begin{enumerate}
    \item $j$ infla.
    \item $j$ es monótona.
    \item $j$ es idempotente.
    \item $j$ respeta ínfimos finitos.
\end{enumerate}
\end{block}
\[
    NA= \mbox{núcleos de } A.
\]}
\end{frame}

\begin{frame}
\begin{block}{Definición:}
    Un cociente de un marco $A$ es un marco $B$ equipado con un morfismo suprayectivo $f\colon A\to B$.
\end{block}
\onslide<2->{Si $A\in \mathbf{Frm}$ y $j\in NA$, entonces $A_j\in \mathbf{Frm}$.
\[
A_j=\{a\in A\mid j(a)=a\}.
\]}
\begin{itemize}
\item<3-> $A_j$ es un cociente de $A$.
\item<4-> ¿Qué es un cociente compacto?
\end{itemize}
\end{frame}

\begin{frame}{Otros ``tipos'' de cocientes}
    $a\in A\in \mathbf{Frm}$ definimos
    \onslide<2->{\[
    u_a(x)=a\vee x, \quad v_a(x)=a\succ x, \quad w_a(x)=((x\succ a)\succ a)
    \]
    $x\in A$}
    
    \begin{itemize}
    \item<3-> $A_{u_a}$ ``cociente cerrado'' \onslide<7->{$\leftrightarrow$ sublocal cerrado.}
    \item<4-> $A_{v_a}$ ``cociente abierto'' \onslide<7->{$\leftrightarrow$ sublocal abierto.}
    \item<5-> $A_{w_a}$ ``cociente regular'' \onslide<7->{$\leftrightarrow$ sublocal regular.}
    \end{itemize}
    
    \onslide<6->{\[
    \mbox{Núcleos }\leftrightarrow \mbox{ Sublocales}\leftrightarrow \mbox{ Subespacios}
    \]}
    \end{frame}

\begin{frame}{Filtros en $\mathbf{Frm}$}
    Sea $A\in \mathbf{Frm}$. Para $F\subseteq A$, decimos que $F$ es un \emph{filtro} si:
		\begin{enumerate}
			\item $1\in F$.
			\item $a\leq b$, $a\in  F$ $\Rightarrow b \in F$.
			\item $a, b \in F \Rightarrow  a \wedge b \in F$.
		\end{enumerate}
	\uncover<2->{Existen diferentes tipos de filtros:
	\begin{itemize}
	\begin{multicols}{2}
		\item Propio
		\item Primo
		\item Completamente primo
		\item (Scott) abierto
		\item Admisible ($\nabla(j)$)
		\end{multicols}
	\end{itemize} }
\end{frame}

\begin{frame}{Filtros admisibles y núcleos ajustados}
    Sea $j\in NA$. El \emph{filtro de admisibilidad} de $j$ es el conjunto
    \[
    \nabla(j)=\{a\in A\mid j(a)=1\}.
    \]
    \onslide<2->{\begin{block}{Observaciones:}
        \begin{itemize}
            \item<3-> $j, k\in NA$, $j\sim k\Leftrightarrow \nabla(j)=\nabla(k)$.
            \item<4-> $j\in NA$ es \emph{ajustado} si es el menor elemento de su bloque.
            \item<5-> $F\in A^\wedge \Rightarrow F=\nabla(j)$ para algún $j\in NA$.
            \item<6-> $F\in A^\wedge\Rightarrow [v_F, w_F]$.
        \end{itemize}
        \end{block}}
        
    \onslide<7->{\[
        f=\dot{\bigvee}\{v_a\mid a\in F\}, \quad v_F=f^\infty, \quad w_F=w_a.
        \]}

\end{frame}

\begin{frame}{Cocientes compactos}
\begin{block}{Definición:}
Sea $A\in \mathbf{Frm}$. Una \emph{cubierta} $A$ es un subconjunto $X\subseteq A$ tal que $\bigvee X=1$. Una \emph{subcubierta} de $X$ es un subconjunto $Y\subseteq X$ tal que $\bigvee Y=1$
\end{block}
\onslide<2->{\[ 
A \mbox{ es compacto si cada cubierta tiene una subcubierta finita.}
\]}

\begin{itemize}
    \item<3-> $A_j$ es un cociente de $A$.
    \item<4-> $A_j$ es compacto $\Leftrightarrow$ $\nabla(j)\in A^\wedge$.
\end{itemize}
\end{frame}

\section{Marcos arreglados}
\begin{frame}{¿Qué son los marcos arreglados?}
    \begin{itemize}

    \item Una de las aplicaciones de la teoría de marcos es que, en cierto punto, un marco puede llegar a mimetizar el comportamiento de la topología de un espacio.\\

    \item<2-> En este sentido, los marcos arreglados buscan imitar la propiedad de que un espacio sea empaquetado.\\

    \item<3-> Como es habitual, las variantes que proporcionan los marcos son caracterizaciones ``libres de puntos''.
    \end{itemize}
\end{frame}

\begin{frame}{Aspectos sensibles a puntos}
    \begin{definition}
        Un espacio topológico $S$ es \emph{empaquetado} si todo conjunto compacto saturado es cerrado
    \end{definition}

    \onslide<2->{Si $Q\in \mathcal{Q}S$ y $Q\notin \mathcal{C}S \Rightarrow S$ no es empaquetado.

    \[
    \mbox{pbase}=\{U\cap Q'\mid U\in \mathcal{O}S, \; Q\in \mathcal{Q}S\}
    \]}

    \uncover<3->{\begin{definition}
        Para un espacio topológico $S$, sea $^pS$ (\emph{espacio de parches}), el espacio con los mismos puntos que $S$ y la topología $\mathcal{O}^pS$ generada por la pbase.
    \end{definition}}
\end{frame}

\begin{frame}{Marcos arreglados}
    Sea $A\in \mathbf{Frm}$ y $\alpha\in \mathbf{Ord}$. 
    \begin{itemize}
        \item $F\in A^\wedge$ es $\alpha$-arreglado si 
    \[
    x\in F\Rightarrow u_d(x)=d\vee x=1,
    \]
    donde $d=d(\alpha)=f^\alpha(0)$ y $f=\bigvee\{v_a\mid a\in F\}$
    \item $A$ es $\alpha$-arreglado si todo $F\in A^\wedge$ es $\alpha$-arreglado.
    \item $A$ es arreglado si $A$ es $\alpha$-arreglado para algún $\alpha$.
    \end{itemize}
\end{frame}

\begin{frame}{Marcos arreglados}
    \begin{block}{Propiedades:}
        \begin{itemize}
            \item Parche trivial $\Leftrightarrow$ arreglado
            \item Arreglado $\Leftrightarrow $ empaquetado $+$ apilado
            \item Un espacio $S$ tiene topología 1-arreglada $\Leftrightarrow$ $S$ es $T_2$.
            \item Arreglado $\Rightarrow$ $T_1$
            \item Regularidad $\Rightarrow$ arreglado
            \item $(\mathbf{fH})\Rightarrow$ arreglado
        \end{itemize}
    \end{block}
\end{frame}

\begin{frame}{Más información sobre el arreglo}
  Si $A\in \mathbf{Frm}$ y $j\in NA\Rightarrow$ $A_j\in \mathbf{Frm}$.
  \onslide<2->{\begin{block}{Observaciones:}
   \begin{itemize}
       \item<3-> $A_j$ es un cociente de $A$.
       \item<5-> $F\in A^\wedge\Rightarrow F=\nabla(j)$.
       \item<6-> Si $u_d(x)=1$, entonces $u_d=v_F$ 
   \end{itemize}
   \end{block}}
\end{frame}

\section{C. C. vs M. A.}
\begin{frame}{Cocientes compactos vs marcos arreglados}
¿Qué relación existe entre los cocientes compactos y los marcos arreglados?

\onslide<2->{\[
A \mbox{ es arreglado } \Leftrightarrow A  \mbox{ tiene al menos un cociente compacto y cerrado.}
\] 

De manera adicional...}
\begin{itemize}
    \item<3-> $[v_F, w_F]$ produce una familia de cocientes compactos.
    \item<4-> $v_F=w_F$ produce un único cociente compacto.  
\end{itemize}

\end{frame}

\begin{frame}{Espacios-Marcos KC}
\begin{block}{Definición:}
Decimos que un espacio topológico $S$ es un \emph{ espacio KC} si cada conjunto compacto es cerrado. 
\end{block}

\onslide<2->{\begin{block}{Definición:}
Un marco $A$ es un \emph{marco KC} si cada cociente compacto de $A$ es cerrado.
\end{block}}

\onslide<3->{En otras palabras, si cada sublocal compacto es cerrado.}
\end{frame}

\begin{frame}[fragile]{El comportamiento de KC}
En espacios
\[\begin{tikzcd}
	{T_3} & {T_2} & KC & {T_1}
	\arrow[Rightarrow, from=1-1, to=1-2]
	\arrow[Rightarrow, from=1-2, to=1-3]
	\arrow[Rightarrow, from=1-3, to=1-4]
\end{tikzcd}\]

En marcos
\[\begin{tikzcd}
	{\mathbf{(reg)}} & {\mathbf{(H)}} & KC & {\mathbf{Arr}} & {T_1}
	\arrow[Rightarrow, from=1-1, to=1-2]
	\arrow[Rightarrow, from=1-2, to=1-3]
	\arrow[Rightarrow, from=1-3, to=1-4]
	\arrow[Rightarrow, from=1-4, to=1-5]
\end{tikzcd}\]
\end{frame}

\begin{frame}[fragile]
    En espacios
\[\begin{tikzcd}
	{T_2+\mbox{compacto}} & {\mbox{Regular}}
	\arrow[Rightarrow, from=1-1, to=1-2]
\end{tikzcd}\]

En marcos
\[\begin{tikzcd}
	{\mathbf{(H)}+\mbox{compacto}} & {\mathbf{(reg)}}
	\arrow["\shortmid"{marking}, Rightarrow, from=1-1, to=1-2]
\end{tikzcd}\]
Existe un marco que cumple $\mathbf{(H)}$ compacto que no es $\mathbf{(saju)}$
\[\begin{tikzcd}
	{\mathbf{(reg)}} & {\mathbf{(aju)}} & {\mathbf{(saju)}}
	\arrow[Rightarrow, from=1-1, to=1-2]
	\arrow[Rightarrow, from=1-2, to=1-3]
\end{tikzcd}\]
\end{frame}
    
 \begin{frame}[fragile]{Colapso del intervalo de admisibilidad}

    \[\begin{tikzcd}
	{\mathbf{(aju)}} \\
	{\mathbf{(reg)}} && {v_F=w_F} \\
	{\mathbf{(fH)}} \\
	{\mathbf{(H)}}
	\arrow[Rightarrow, from=1-1, to=2-3]
	\arrow[Rightarrow, from=2-1, to=1-1]
	\arrow[Rightarrow, from=2-1, to=2-3]
	\arrow[Rightarrow, from=2-1, to=3-1]
	\arrow[Rightarrow, from=3-1, to=2-3]
	\arrow[Rightarrow, from=3-1, to=4-1]
	\arrow["{?}"', Rightarrow, from=4-1, to=2-3]
\end{tikzcd}\]
    \end{frame}

\begin{frame}{Lo que estamos trabajando}
\begin{itemize}
\item Condiciones que permitan colapsar los intervalos de admisibilidad.
\item<2-> Caracterizar o conocer el comportamiento de los filtros abiertos de un marco.
\item<3-> Conocer la relación de la condición de arreglo con los distintos cocientes que produce el intervalo de admisibilidad.
\item<4-> Definir nociones libres de puntos relacionadas con $KC$.
\item<5-> Dar ejemplos de los diferentes marcos que se vayan definiendo.
\item<-6> $\vdots$
\end{itemize}
\end{frame}

%    \section{Lo que hemos hecho}
%\begin{frame}[fragile]{Resultados probados}
%    \begin{block}{Proposición:}
%        Para $F\in A^\wedge$ y $Q\in \mathcal{Q}S$, si $j\in [V_Q, W_Q]$, entonces 
%    \[
%    \nabla(U_* j U^*)=F.
%    \]
%    \end{block}

%    De esta manera $\mho\colon [V_Q, W_Q]\to [V_F, W_F]$
    
%    \[\begin{tikzcd}
%        NA && {N\mathcal{O}S} \\
%        {[v_F, w_F]} && {[v_Q, w_Q]}
%        \arrow["{N(U)}", shift left=2, from=1-1, to=1-3]
%        \arrow["{N(U)_*}", shift left=2, from=1-3, to=1-1]
%        \arrow[hook, from=2-1, to=1-1]
%        \arrow["\mho^*", shift left=2, dashed, from=2-1, to=2-3]
%        \arrow[hook, from=2-3, to=1-3]
%        \arrow["\mho", shift left=2, from=2-3, to=2-1]
%    \end{tikzcd}\]
%\end{frame}

%\begin{frame}{Resultados probados}
%\begin{block}{Proposición:}
%Para un marco $A$ Hausdorff, el intervalo de admisibilidad asociado a un filtro abierto es trivial. En otras palabras, para $F\in A^\wedge$, $[v_F, w_F]=\{*\}$, es decir, $v_F=w_F$.
%\end{block}

%\begin{block}{Teorema:}
%En un marco Hausdorff todo cociente compacto es isomorfo a un cociente cerrado de la topología de un espacio Hausdorff.
%\end{block}
%\end{frame}

%\begin{frame}[fragile]{EL Q-cuadrado}
%	\begin{block}{Ingredientes:}
%		\begin{multicols}{3}
		
%		 \begin{itemize}
%			\item $U_A\colon \mathcal{O}S$
%			\item $F\in A^\wedge \rightarrow v_F$
%			\item $A_F=A_{v_F}$
%			\item $\nabla\in \mathcal{O}S \rightarrow v_\nabla$
%			\item $\mathcal{O}S_\nabla=\mathcal{O}S_{v_\nabla}$
%			\item $G=\nabla U_A$
%			\item $Q=\pt A_F$
%			\item $g=G_{\mid A_F}$
%			\item $?\colon \mathcal{O}S_\nabla\to A_F$
%		 \end{itemize}
%		
%		\end{multicols}
%	\end{block}
%\[\begin{tikzcd}
%	A && {A_F} && {} \\
%	&&& {\mathcal{O}Q} \\
%	{\mathcal{O}S} && {\mathcal{O}S_\nabla}
%	\arrow["F", shift left=2, color={rgb,255:red,67;green,67;blue,239}, from=1-1, to=1-3]
%	\arrow["{U_A}"', shift right=2, color={rgb,255:red,67;green,67;blue,239}, from=1-1, to=3-1]
%	\arrow["G", from=1-1, to=3-3]
%	\arrow["{F_*}", shift left=2, color={rgb,255:red,245;green,61;blue,61}, from=1-3, to=1-1]
%	\arrow[from=1-3, to=2-4]
%	\arrow["g", shift left=2, color={rgb,255:red,69;green,237;blue,72}, dashed, from=1-3, to=3-3]
%	\arrow["{(U_A)_*}"', shift right=2, color={rgb,255:red,245;green,61;blue,61}, from=3-1, to=1-1]
%	\arrow["\nabla"', shift right=2, color={rgb,255:red,67;green,67;blue,239}, from=3-1, to=3-3]
%	\arrow["{?}", shift left=2, from=3-3, to=1-3]
%	\arrow[from=3-3, to=2-4]
%	\arrow["{\nabla_*}"', shift right=2, color={rgb,255:red,245;green,61;blue,61}, from=3-3, to=3-1]
%\end{tikzcd}\]
%\end{frame}

%\begin{frame}{Resultados probados}
%\begin{block}{Proposición:}
 %   Sea $A$ un marco y $j\in NA$. Si $A$ es arreglado, entonces $A_j$ es arreglado.
%\end{block}
%\end{frame}

%\begin{frame}{Preguntas abiertas}
%    \begin{itemize}
%        \item ¿Se cumple que $PPA=PA$ o $PPPA=PPA$?
%        \item ¿$\mathbf{(H)}\Rightarrow$ arreglado?
%        \item ¿$A$ arreglado $\Rightarrow$ $\prod A$ arreglado?
%        \item ¿Cómo se puede medir el fallo de la condición de arreglo?
%        \item ¿Qué significa que un marco sea apilado?
%        \item ¿Existen ejemplos de marcos (locales) Hausdorff y compactos que sean cerrados?
%        \end{itemize}
%\end{frame}
\End

\section*{\textsc{Referencias}}
\begin{frame}[allowframebreaks]
\frametitle{References}
\begin{thebibliography}{20}\markboth{Bibliografía}{Bibliografía}

\bibitem{P.T.} P. T. Johnstone, \textit{Stone spaces}, Cambridge Studies in Advanced Mathematics, vol. 3, Cambridge University Press, Cambridge, 1982. MR 698074


\bibitem{J.M.} J. Monter; A. Zaldívar, \textit{El enfoque locálico de las reflexiones booleanas: un análisis en la categoría de marcos} [tesis de maestría], 2022. Universidad de Guadalajara.

\bibitem{J.P.} J. Picado and A. Pultr, \textit{Frames and locales: Topology without points}, Frontiers in Mathematics, Springer Basel, 2012.

\bibitem{J.P.2} J. Picado and A. Pultr, \textit{Separation in point-free topology}, Springer, 2021.

\bibitem{R.S.} Rosemary A Sexton, \textit{A point free and point-sensitive analysis of the patch assembly}, The University of Manchester (United Kingdom), 2003.

\bibitem{H.S.3} Harold Simmons, \textit{The assembly of a frame}, University of Manchester (2006).

\bibitem{R.S.3} RA Sexton and H. Simmons, \textit{Point-sensitive and point-free patch constructions}, Journal of Pure and Applied Algebra \textbf{207} (2006), no. 2, 433-468.

\bibitem{A.Z.} A. Zaldívar, \textit{Introducción a la teoría de marcos} [notas curso], 2024. Universidad de Guadalajara.

\end{thebibliography}
\end{frame}

\end{document}
