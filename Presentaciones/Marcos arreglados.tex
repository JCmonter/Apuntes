\documentclass[compress,12pt]{beamer}

\usetheme{Arguelles}
\usepackage{graphicx}
\usepackage{caption}
%\usepackage[spanish,es-noshorthands]{babel}
\usepackage[spanish]{babel} 
\usepackage[pages=some]{background}
\usepackage{tikz-cd}
\usepackage{amsmath,amssymb,latexsym,amscd} 
\usepackage[all,cmtip]{xy}
\usepackage{fancyhdr}
\usepackage{mathalfa}
\usepackage{mathrsfs}
\usetikzlibrary{babel}
\usepackage{hyperref}
\usepackage{ragged2e}
\usepackage{wasysym}
\usepackage{multicol}
%\hypersetup{colorlinks=true,linkcolor=blue,citecolor=brown,linktocpage=true,pagebackref=true,hyperindex=true}
\pagenumbering{arabic}

\DeclareMathOperator{\op}{op}
\DeclareMathOperator{\pt}{pt}
\DeclareMathOperator{\spec}{spec}
\DeclareMathOperator{\Fit}{Fit}
\DeclareMathOperator{\Pth}{P}
\DeclareMathOperator{\Frm}{Frm}
\DeclareMathOperator{\Obj}{Obj}
\DeclareMathOperator{\Hom}{Hom}

\title{Marcos arreglados}
%\event{Seminario de Álgebra, CUCEI}
\date{\today}
\author{Juan Carlos Monter Cortés}
\institute{Universidad de Guadalajara}
\email{juan.monter2902@alumnos.udg.mx}

%\homepage{www.mywebsite.com}
%\github{username}

\begin{document}

\frame[plain]{\titlepage}

%\begin{frame}{Contenido}
%\tableofcontents %Imprime la tabla de contenido
%\end{frame}

\begin{frame}{¿Qué son los marcos arreglados?}
    \begin{itemize}

    \item Una de las aplicaciones de la teoría de marcos es que, en cierto punto, un marco puede llegar a mimetizar el comportamiento de la topología de un espacio.\\

    \item En este sentido, los marcos arreglados buscan imitar la propiedad de que un espacio sea empaquetado.\\

    \item Como es habitual, las variantes que proporcionan los marcos son caracterizaciones ``libres de puntos''.
    \end{itemize}
\end{frame}

\section{Información sobre los parches}
\begin{frame}{Espacio de parches}
    Sea $S\in \mathbf{Top}$. El espacio de parches, denotado por $^pS$, es el espacio cuya topología está dada por 
\[
\mbox{pbase}=\{U\cap Q'\mid U\in \mathcal{O}S, Q\in \mathcal{Q}S\}
\]
Un espacio es \emph{empaquetado} si todo conjunto compacto (saturado) es cerrado.
\begin{block}{Propiedades:}
    \begin{itemize}
        \item $T_2\Rightarrow \mbox{ empaquetado }\Rightarrow T_1$
        \item $S\,=\,^pS\Leftrightarrow S$ es empaquetado
        \item $^{pp}S\,=\,^{ppp}S$.
    \end{itemize}
\end{block}
\end{frame}

\begin{frame}{Filtros en $\mathbf{Frm}$}
    Sea $A\in \mathbf{Frm}$. Para $F\subseteq A$, decimos que $F$ es un \emph{filtro} si:
		\begin{enumerate}
			\item $1\in F$.
			\item $a\leq b$, $a\in  F$ $\Rightarrow b \in F$.
			\item $a, b \in F \Rightarrow  a \wedge b \in F$.
		\end{enumerate}
	\uncover<2->{Existen diferentes tipos de filtros:
	\begin{itemize}
	\begin{multicols}{2}
		\item Propio
		\item Primo
		\item Completamente primo
		\item (Scott) abierto
		\item Admisible ($\nabla(j)$)
		\end{multicols}
	\end{itemize} }
\end{frame}

\begin{frame}{Filtros admisibles y núcleos ajustados}
    Sea $j\in NA$. El \emph{filtro de admisibilidad} de $j$ es el conjunto
    \[
    \nabla(j)=\{a\in A\mid j(a)=1\}.
    \]
    \onslide<2->{\begin{block}{Observaciones:}
        \begin{itemize}
            \item<3-> $j, k\in NA$, $j\sim k\Leftrightarrow \nabla(j)=\nabla(k)$.
            \item<4-> $j\in NA$ es \emph{ajustado} si es el menor elemento de su bloque.
            \item<5-> $F\in A^\wedge \Rightarrow F=\nabla(j)$ para algún $j\in NA$.
            \item<6-> $F\in A^\wedge\Rightarrow [v_F, w_F]$.
        \end{itemize}
        \end{block}}
\end{frame}

\begin{frame}
	\only<-1>{\begin{block}{Teorema (Hoffman-Mislove):}
	Sean $A\in \mathbf{Frm}$ y $S=\pt(A)$, entonces existe una correspondencia biyectiva entre:
	\begin{enumerate}
		\item $\mathcal{Q}S=$ compactos saturados en $S$
		\item $A^\wedge=$ filtros abiertos en $A$
	\end{enumerate}
\end{block}}

 \only<2->{\begin{block}{Teorema (Hoffman-Mislove extendido):}
	Sean $A\in \mathbf{Frm}$ y $S=\pt(A)$, entonces existe una correspondencia biyectiva entre:
	\begin{enumerate}
		\item $\mathcal{Q}S=$ compactos saturados en $S$
		\item $A^\wedge=$ filtros abiertos en $A$
		\item $v_F=$ núcleos ajustados
	\end{enumerate}
\end{block}}
\onslide<3->{El Teorema de H.-M. nos proporciona $(F, Q, \nabla(Q))$
\[
F\in A^\wedge  \leftrightarrow Q\in \mathcal{Q}S\leftrightarrow \nabla(Q)\in \mathcal{O}S^\wedge
\]
\[
	x\in F  \Leftrightarrow Q\subseteq U_A(x)\Leftrightarrow U_A(x)\in \nabla(Q)
\]}
\end{frame}

\begin{frame}{El marco de parches}
    Basados en el Teorema de H.-M. se introduce el \emph{marco de parches}.

	\[\text{Pbase}(A) =\{u_{a}\wedge v_{F}\mid a\in A, F\in A^{\wedge}\}.\]
	
	\uncover<2->{
        \begin{itemize}
            \item $PA$ es el marco generado por la $\mbox{Pbase}$. 
            \item $A\in \mathbf{Frm}$ es \emph{parche trivial} si $A\simeq PA$
            \item $A$ es parche trivial $\Leftrightarrow$ $u_d=v_F$
        \end{itemize}} 

        \uncover<3->{
	 \[\xymatrix{ A\ar[r]^{i}\ar@/^/@<+1.5ex>[rr]^{\eta_{A}} & PA\ar[r]^{\iota} & NA }\]
	}
\end{frame}

\section{La condición de arreglo}
\begin{frame}{Marcos arreglados}
    Sea $A\in \mathbf{Frm}$ y $\alpha\in \mathbf{Ord}$. 
    \begin{itemize}
        \item $F\in A^\wedge$ es $\alpha$-arreglado si 
    \[
    x\in F\Rightarrow u_d(x)=d\vee x=1,
    \]
    donde $d=d(\alpha)=f^\alpha(0)$ y $f=\bigvee\{v_a\mid a\in F\}$
    \item $A$ es $\alpha$-arreglado si todo $F\in A^\wedge$ es $\alpha$-arreglado.
    \item $A$ es arreglado si $A$ es $\alpha$-arreglado para algún $\alpha$.
    \end{itemize}
\end{frame}

\begin{frame}{Marcos arreglados}
    \begin{block}{Propiedades:}
        \begin{itemize}
            \item Parche trivial $\Leftrightarrow$ arreglado
            \item Arreglado $\Leftrightarrow $ empaquetado $+$ apilado
            \item Un espacio $S$ tiene topología 1-arreglada $\Leftrightarrow$ $S$ es $T_2$.
            \item Arreglado $\Rightarrow$ $T_1$
            \item Regularidad $\Rightarrow$ arreglado
            \item $(\mathbf{fH})\Rightarrow$ arreglado
        \end{itemize}
    \end{block}
\end{frame}

\begin{frame}{Otra forma de ver arreglado}
    Si $A\in \mathbf{Frm}$ y $j\in NA\Rightarrow$ $A_j\in \mathbf{Frm}$.
    \onslide<2->{\begin{block}{Observaciones:}
    \begin{itemize}
        \item<3-> $A_j$ es un cociente de $A$.
        \item<4-> $A_j$ es compacto $\Leftrightarrow$ $\nabla(j)\in A^\wedge$.
        \item<5-> $F\in A^\wedge\Rightarrow F=\nabla(j)$.
        \item<6-> $A$ es arreglado si todo cociente compacto es cerrado. 
        \item<7-> $F\in [v_F, w_F]$ produce una familia de cocientes compactos.
        \item<8-> $v_F=w_F$ produce un único cociente compacto.  
    \end{itemize}
    \end{block}}
    
    \end{frame}
    
    \begin{frame}[fragile]{Colapso del intervalo de admisibilidad}
        \[\begin{tikzcd}
        {(\mathbf{aju})} \\
        {(\mathbf{reg})} & {v_F=w_F} \\
        {(\mathbf{fH})} \\
        {(\mathbf{H})}
        \arrow[Rightarrow, from=1-1, to=2-2]
        \arrow[Rightarrow, from=2-1, to=1-1]
        \arrow[Rightarrow, from=2-1, to=2-2]
        \arrow[Rightarrow, from=2-1, to=3-1]
        \arrow[Rightarrow, from=3-1, to=2-2]
        \arrow[Rightarrow, from=3-1, to=4-1]
        \arrow[Rightarrow, from=4-1, to=2-2]
    \end{tikzcd}\]
    \end{frame}

    \section{Lo que hemos hecho}
\begin{frame}[fragile]{Resultados probados}
    \begin{block}{Proposición:}
        Para $F\in A^\wedge$ y $Q\in \mathcal{Q}S$, si $j\in [V_Q, W_Q]$, entonces 
    \[
    \nabla(U_* j U^*)=F.
    \]
    \end{block}

    De esta manera $\mho\colon [V_Q, W_Q]\to [V_F, W_F]$
    
    \[\begin{tikzcd}
        NA && {N\mathcal{O}S} \\
        {[v_F, w_F]} && {[v_Q, w_Q]}
        \arrow["{N(U)}", shift left=2, from=1-1, to=1-3]
        \arrow["{N(U)_*}", shift left=2, from=1-3, to=1-1]
        \arrow[hook, from=2-1, to=1-1]
        \arrow["\mho^*", shift left=2, dashed, from=2-1, to=2-3]
        \arrow[hook, from=2-3, to=1-3]
        \arrow["\mho", shift left=2, from=2-3, to=2-1]
    \end{tikzcd}\]
\end{frame}

\begin{frame}{Resultados probados}
\begin{block}{Proposición:}
Para un marco $A$ Hausdorff, el intervalo de admisibilidad asociado a un filtro abierto es trivial. En otras palabras, para $F\in A^\wedge$, $[v_F, w_F]=\{*\}$, es decir, $v_F=w_F$.
\end{block}

\begin{block}{Teorema:}
En un marco Hausdorff todo cociente compacto es isomorfo a un cociente cerrado de la topología de un espacio Hausdorff.
\end{block}
\end{frame}

\begin{frame}[fragile]{EL Q-cuadrado}
	\begin{block}{Ingredientes:}
		\begin{multicols}{3}
		
		 \begin{itemize}
			\item $U_A\colon \mathcal{O}S$
			\item $F\in A^\wedge \rightarrow v_F$
			\item $A_F=A_{v_F}$
			\item $\nabla\in \mathcal{O}S \rightarrow v_\nabla$
			\item $\mathcal{O}S_\nabla=\mathcal{O}S_{v_\nabla}$
			\item $G=\nabla U_A$
			\item $Q=\pt A_F$
			\item $g=G_{\mid A_F}$
			\item $?\colon \mathcal{O}S_\nabla\to A_F$
		 \end{itemize}
		
		\end{multicols}
	\end{block}
\[\begin{tikzcd}
	A && {A_F} && {} \\
	&&& {\mathcal{O}Q} \\
	{\mathcal{O}S} && {\mathcal{O}S_\nabla}
	\arrow["F", shift left=2, color={rgb,255:red,67;green,67;blue,239}, from=1-1, to=1-3]
	\arrow["{U_A}"', shift right=2, color={rgb,255:red,67;green,67;blue,239}, from=1-1, to=3-1]
	\arrow["G", from=1-1, to=3-3]
	\arrow["{F_*}", shift left=2, color={rgb,255:red,245;green,61;blue,61}, from=1-3, to=1-1]
	\arrow[from=1-3, to=2-4]
	\arrow["g", shift left=2, color={rgb,255:red,69;green,237;blue,72}, dashed, from=1-3, to=3-3]
	\arrow["{(U_A)_*}"', shift right=2, color={rgb,255:red,245;green,61;blue,61}, from=3-1, to=1-1]
	\arrow["\nabla"', shift right=2, color={rgb,255:red,67;green,67;blue,239}, from=3-1, to=3-3]
	\arrow["{?}", shift left=2, from=3-3, to=1-3]
	\arrow[from=3-3, to=2-4]
	\arrow["{\nabla_*}"', shift right=2, color={rgb,255:red,245;green,61;blue,61}, from=3-3, to=3-1]
\end{tikzcd}\]
\end{frame}

\begin{frame}{Resultados probados}
\begin{block}{Proposición:}
    Sea $A$ un marco y $j\in NA$. Si $A$ es arreglado, entonces $A_j$ es arreglado.
\end{block}
\end{frame}

\begin{frame}{Preguntas abiertas}
    \begin{itemize}
        \item ¿Se cumple que $PPA=PA$ o $PPPA=PPA$?
        \item ¿$\mathbf{(H)}\Rightarrow$ arreglado?
        \item ¿$A$ arreglado $\Rightarrow$ $\prod A$ arreglado?
        \item ¿Cómo se puede medir el fallo de la condición de arreglo?
        \item ¿Qué significa que un marco sea apilado?
        \item ¿Existen ejemplos de marcos (locales) Hausdorff y compactos que sean cerrados?
        \end{itemize}
\end{frame}

\end{document}
