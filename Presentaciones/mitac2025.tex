\documentclass[compress,12pt]{beamer}

\usetheme{Arguelles}
\usepackage{graphicx}
\usepackage{caption}
%\usepackage[spanish,es-noshorthands]{babel}
\usepackage[spanish]{babel} 
\usepackage[pages=some]{background}
\usepackage{tikz-cd}
\usepackage{amsmath,amssymb,latexsym,amscd} 
\usepackage[all,cmtip]{xy}
\usepackage{fancyhdr}
\usepackage{mathalfa}
\usepackage{mathrsfs}
\usetikzlibrary{babel}
\usepackage{hyperref}
\usepackage{ragged2e}
\usepackage{wasysym}
\usepackage{multicol}
%\hypersetup{colorlinks=true,linkcolor=blue,citecolor=brown,linktocpage=true,pagebackref=true,hyperindex=true}
\pagenumbering{arabic}

\DeclareMathOperator{\op}{op}
\DeclareMathOperator{\pt}{pt}
\DeclareMathOperator{\spec}{spec}
\DeclareMathOperator{\Fit}{Fit}
\DeclareMathOperator{\Pth}{P}
\DeclareMathOperator{\Frm}{Frm}
\DeclareMathOperator{\Top}{Top}
\DeclareMathOperator{\Obj}{Obj}
\DeclareMathOperator{\Hom}{Hom}

%%Se define el "environment" teorema
\newtheorem{thm}{Teorema}
\newtheorem{dfn}{Definición}
\newtheorem*{dfn*}{Definición}
\newtheorem{lem}{Lema}
\newtheorem{cor}{Corolario}
\newtheorem{prop}{Proposición}
\newtheorem{obs}{Observación}
\newtheorem{ej}{Ejemplo}

\title{Intervalos de admisibilidad y marcos $KC$}
%\date{\today}
\author{Juan Carlos Monter Cortés}
\institute{Universidad de Guadalajara}
\email{juan.monter2902@alumnos.udg.mx}
\event{MITAC, Agosto 2025}

%\homepage{www.mywebsite.com}
\github{JCmonter}

\begin{document}

\frame[plain]{\titlepage}

\begin{frame}{Contenido}
\tableofcontents %Imprime la tabla de contenido
\end{frame}

\section{Información preliminar}
\begin{frame}{Marcos}
    \begin{itemize}
        \begin{multicols}{2}
            \item $A$
            \item $(A, \leq)$
            \item $(A, \leq, \vee, 0)$ o $(A, \leq, \wedge, 1)$
            \item $(A, \leq, \bigvee, \bigwedge, 0, 1)$
        \end{multicols} 
        \end{itemize}
    
        \onslide<2->{Un marco es una retícula completa que cumple cierta ley distributiva (ley distributiva de marcos), es decir,}
        \onslide<3->{\[
        (A, \leq, \bigvee, \wedge, 0, 1), \quad a\wedge\bigvee X=\bigvee \{a\wedge x\mid x\in X\}
        \]}
    
        \onslide<4->\[
        \mathbf{Frm}=\left\{ \begin{array}{ll} A, & \mbox{ marcos}\\ \\  f, & \mbox{ morfismo de marcos} \end{array} \right.
        \]
\end{frame}

\begin{frame}{¿Por qué estudiamos los marcos?}
    \begin{itemize}
        \item Estructuras simples.
        \item<2-> Existen herramientas que facilitan el estudio de los marcos.
        \item<3-> Correspondencias biyectivas.
        \item<4-> Buen comportamiento categórico.
        \item<5-> \alert<7->{La topología de un espacio ($\mathcal{O}S)$ es un marco.}
        \item<6-> $\mathbf{Loc}=\mathbf{Frm}^{\op}$ está en relación con $\mathbf{Top}$.
    \end{itemize}
\end{frame}

\begin{frame}{Cocientes en $\mathbf{Frm}$}
$\mathbf{Frm}$ proporciona correspondencias biyectivas interesantes
\onslide<2->{\[
\mbox{Congruencias}\leftrightarrow \mbox{Conjuntos implicativos}\leftrightarrow \mbox{\alert<3->{Núcleos}}
\]}

\onslide<4->{\begin{block}{Definición:}
Sea $A\in \mathbf{Frm}$ y $j\colon A\to A$, decimos que $j$ es un \emph{núcleo} si:
\begin{enumerate}
    \item $j$ infla.
    \item $j$ es monótona.
    \item $j$ es idempotente.
    \item $j$ respeta ínfimos finitos.
\end{enumerate}
\end{block}
\[
    NA= \mbox{núcleos de } A.
\]}
\end{frame}

\begin{frame}
\begin{block}{Definición:}
    Un cociente de un marco $A$ es un marco $B$ equipado con un morfismo suprayectivo $f\colon A\to B$.
\end{block}
\onslide<2->{Si $A\in \mathbf{Frm}$ y $j\in NA$, entonces $A_j\in \mathbf{Frm}$.
\[
A_j=\{a\in A\mid j(a)=a\}.
\]}
\begin{itemize}
\item<3-> $A_j$ es un cociente de $A$.
\item<4-> ¿Qué es un cociente compacto?
\end{itemize}
\end{frame}

\begin{frame}{Otros ``tipos'' de cocientes}
    $a\in A\in \mathbf{Frm}$ definimos
    \onslide<2->{\[
    u_a(x)=a\vee x, \quad v_a(x)=a\succ x, \quad w_a(x)=((x\succ a)\succ a)
    \]
    $x\in A$}
    
    \begin{itemize}
    \item<3-> $A_{u_a}$ ``cociente cerrado'' \onslide<7->{$\leftrightarrow$ sublocal cerrado.}
    \item<4-> $A_{v_a}$ ``cociente abierto'' \onslide<7->{$\leftrightarrow$ sublocal abierto.}
    \item<5-> $A_{w_a}$ ``cociente regular'' \onslide<7->{$\leftrightarrow$ sublocal regular.}
    \end{itemize}
    
    \onslide<6->{\[
    \mbox{Núcleos }\leftrightarrow \mbox{ Sublocales}\leftrightarrow \mbox{ Subespacios}
    \]}
    \end{frame}


\section{Intervalos de admisibilidad}
\begin{frame}{Teoría de marcos}

\end{frame}

\begin{frame}{Núcleos}

\end{frame}

\begin{frame}{Filtros de admisiblidad}

\end{frame}

\begin{frame}{Intervalos de admisibilidad}

\end{frame}

\section{Marcos $KC$}

\begin{frame}{Cocientes en $\Frm$}

\end{frame}

\begin{frame}{Intervalos de admisibilidad}
Si $F\in A^\wedge$, entonces 
\[
u_d\leq v_F \leq  w_F
\] 
para $d=v_F(0)$, $v_F=f^\infty$.

\begin{block}{Información con los intervalos}
\begin{itemize}
\item<2-> $[v_F, w_F]\subseteq NA$ es el intervalo de admisibilidad.
\item<3-> Si $j\in [v_F,w_F]$, $A_j$ es compacto.
\item<4-> Arreglado $\Leftrightarrow v_F\leq u_d$
\item<5-> $\mathbf{(fH)}\Leftrightarrow\forall\, j\in [v_F,w_F]$, $j=u_\bullet$ y $\bullet\in A$.  
\item<6-> $\mathbf{(aju)}\Leftrightarrow [v_F,w_F]=\{*\}$ y $*=u_\bullet$ para $\bullet\in A$. 
\end{itemize}
\end{block}
\end{frame}

\begin{frame}{Marcos $KC$}
$S\in \Top$ es $\mathrm{KC}$ si todo conjunto compacto es cerrado. $S$ es $\mathrm{US}$ si cada sucesión convergente tiene exactamente un límite al cual converge.
\[
T_2\Rightarrow KC\Rightarrow US\Rightarrow T_1
\]

\onslide<2->{\begin{dfn}
$A\in \Frm$ es $KC$ si todo cociente compacto de $A$ es cerrado. 
\end{dfn}}
\onslide<3->{Equivalentemente
\[
A_F=u_d
\]
para algún $d\in A$ y $F\in A^\wedge$.}
\end{frame}

\begin{frame}[fragile]{Propiedades de los marcos $KC$}
\[
KC\Rightarrow \mbox{Arreglado}
\]

\onslide<2->{\begin{prop}
Si $A$ es $KC$ entonces $A_j$ es $KC$ para todo $j\in NA$.
\end{prop}}

\onslide<3->{\begin{prop}
Si $A$ es $KC$, entonces $A$ es $T_1$.
\end{prop}

De hecho}
\onslide<4->\[\begin{tikzcd}
	{\mathbf{(reg)}} & {\mathbf{(fH)}} & KC & {\mbox{Arreglado}} & {T_1} \\
	& {\mathbf{(aju)}} && {}
	\arrow[Rightarrow, from=1-1, to=1-2]
	\arrow[Rightarrow, from=1-1, to=2-2]
	\arrow[Rightarrow, from=1-2, to=1-3]
	\arrow[Rightarrow, from=1-3, to=1-4]
	\arrow[Rightarrow, from=1-4, to=1-5]
	\arrow[Rightarrow, from=2-2, to=1-3]
\end{tikzcd}\]

\end{frame}

\begin{frame}{La topología máximo compacta}
Consideremos $S=\{x,y\}\cup \mathbb{N}^2$ con $x,y\notin \mathbb{N}^2$ y sea 
\[
R_n=\{(m,n)\mid m\in \mathbb{N}\}
\]
Definimos 
\[
\mathcal{O}S=\mathcal{P}\mathbb{N}^2 \cup \mathcal{U}\cup\mathcal{V}
\]
donde
\[
\mathcal{U}=\{U\subseteq S\mid x\in U\mbox{ y }\forall n\in \mathbb{N}, U\cap R_n\mbox{ es cofinito}\}
\]
\[
\mathcal{V}=\{V\subseteq S\mid y\in V \mbox{ y }\exists F\subseteq \mathbb{N}\mbox{ finito tal que }\forall n\notin F, R_n\subseteq V \}
\]
$\mathcal{O}S$ es una topología..
\end{frame}

\begin{frame}{Propiedades de $\mathcal{O}S$}
	\begin{itemize}
	\item $\mathcal{O}S$ es $T_1$.
	\item $\mathcal{O}S$ no es $\mathbf{(H)}$.
	\item $\mathcal{O}S$ es compacto.
	\item $\mathcal{O}S$ es $\mathbf{(aju)}$.
	\item $\mathcal{O}S$ es $KC$.
	\item $\mathcal{O}S$ es $2$-arreglado.
	\end{itemize}
\end{frame}

\begin{frame}{El ejemplo de Paseka y Smarda}
Consideremos $A\in\Frm$ y $A_r=\{a\in A\mid \neg\neg a=a\}$. Definimos
\[
K(A)=\{(u,v)\mid u\in A,\, v\in A_r,\, u\leq v\}
\]
$K(A)\in \Frm$.
\begin{block}{Propiedades de $K(A)$}
\begin{itemize}
	\item Si $A$ es $\mathbf{(H)}$ y $\neg m=0$ para $m$ máximo, $K(A)$ es $\mathbf{(H)}$.
	\item Si $A$ es compacto, entonces $K(A)$ es compacto.
	\item $K(A)$ no es subajustado
\end{itemize}
\end{block}
\end{frame}

\begin{frame}
De manera adicional, sea $A=[0,1]$ con la topología usual. Entonces
\begin{itemize}
\item $\mathcal{O}I$ es $\mathbf{(H)}$.
\item $\mathcal{O}I$ es compacto.
\item $K(\mathcal{O}I)$ es compacto y $\mathbf{(H)}$.
\item $K(\mathcal{O}I)$ no es subajustado.
\item $K(\mathcal{O}I)$ no es espacial.
\end{itemize}

\[
\mbox{Existe marcos Hausdorr y compactos que no son espaciales}.
\]
\end{frame}

\End

\section*{\textsc{Referencias}}
\begin{frame}[allowframebreaks]
\frametitle{Bibliografía}
\begin{thebibliography}{20}\markboth{Bibliografía}{Bibliografía}

\bibitem{P.T.} P. T. Johnstone, \textit{Stone spaces}, Cambridge Studies in Advanced Mathematics, vol. 3, Cambridge University Press, Cambridge, 1982. MR 698074

\bibitem{J.M.} J. Monter; A. Zaldívar, \textit{El enfoque locálico de las reflexiones booleanas: un análisis en la categoría de marcos} [tesis de maestría], 2022. Universidad de Guadalajara.

\bibitem{P.S.} J. Paseka and B. Smarda, \textit{$ T_2 $-frames and almost compact frames.} Czechoslovak Mathematical Journal (1992), 42(3), 385-402.

\bibitem{J.P.} J. Picado and A. Pultr, \textit{Frames and locales: Topology without points}, Frontiers in Mathematics, Springer Basel, 2012.

\bibitem{J.P.2} J. Picado and A. Pultr, \textit{Separation in point-free topology}, Springer, 2021.

\bibitem{R.S.} RA Sexton, \textit{A point free and point-sensitive analysis of the patch assembly}, The University of Manchester (United Kingdom), 2003.

\bibitem{R.S.2} RA Sexton, \textit{Frame theoretic assembly as a unifying construct}, The University of Manchester (United Kingdom), 2000.

\bibitem{R.S.3} RA Sexton and H. Simmons, \textit{Point-sensitive and point-free patch constructions}, Journal of Pure and Applied Algebra \textbf{207} (2006), no. 2, 433-468.

\bibitem{H.S.} H. Simmons, \textit{An Introduction to Frame Theory}, lecture notes, University of Manchester. Disponible en línea en \url{https://web.archive.org/web/20190714073511/http://staff.cs.manchester.ac.uk/~hsimmons}.

\bibitem{H.S.R} H. Simmons, \textit{Regularity, fitness, and the block structure of frames.} Applied Categorical Structures 14 (2006): 1-34.

\bibitem{H.S.4} H. Simmons, \textit{The lattice theoretic part of topological separation properties}, Proceedings of the Edinburgh Mathematical Society, vol.~21, pp.~41--48, 1978.

\bibitem{H.S.V} H. Simmons, \textit{The Vietoris modifications of a frame}. Unpublished manuscript (2004), 79pp., available online at http://www. cs. man. ac. uk/hsimmons.

\bibitem{A.W.} A. Wilansky, \textit{Between T1 and T2}, MONTHLY (1967): 261-266.

\bibitem{A.Z.} A. Zaldívar, \textit{Introducción a la teoría de marcos} [notas curso], 2025. Universidad de Guadalajara.
\end{thebibliography}
\end{frame}

\end{document}