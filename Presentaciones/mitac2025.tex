\documentclass[compress,12pt]{beamer}

\usetheme{Arguelles}
\usepackage{graphicx}
\usepackage{caption}
%\usepackage[spanish,es-noshorthands]{babel}
\usepackage[spanish]{babel} 
\usepackage[pages=some]{background}
\usepackage{tikz-cd}
\usepackage{amsmath,amssymb,latexsym,amscd} 
\usepackage[all,cmtip]{xy}
\usepackage{fancyhdr}
\usepackage{mathalfa}
\usepackage{mathrsfs}
\usetikzlibrary{babel}
\usepackage{hyperref}
\usepackage{ragged2e}
\usepackage{wasysym}
\usepackage{multicol}
%\hypersetup{colorlinks=true,linkcolor=blue,citecolor=brown,linktocpage=true,pagebackref=true,hyperindex=true}
\pagenumbering{arabic}

\DeclareMathOperator{\op}{op}
\DeclareMathOperator{\pt}{pt}
\DeclareMathOperator{\spec}{spec}
\DeclareMathOperator{\Fit}{Fit}
\DeclareMathOperator{\Pth}{P}
\DeclareMathOperator{\Frm}{Frm}
\DeclareMathOperator{\Top}{Top}
\DeclareMathOperator{\Obj}{Obj}
\DeclareMathOperator{\Hom}{Hom}

%%Se define el "environment" teorema
\newtheorem{thm}{Teorema}
\newtheorem{dfn}{Definición}
\newtheorem*{dfn*}{Definición}
\newtheorem{lem}{Lema}
\newtheorem{cor}{Corolario}
\newtheorem{prop}{Proposición}
\newtheorem{obs}{Observación}
\newtheorem{ej}{Ejemplo}

\title{Intervalos de admisibilidad y marcos $KC$}
%\date{\today}
\author{Juan Carlos Monter Cortés}
\institute{Universidad de Guadalajara}
\email{juan.monter2902@alumnos.udg.mx}
\event{MITAC, Agosto 2025}

%\homepage{www.mywebsite.com}
\github{JCmonter}

\begin{document}

\frame[plain]{\titlepage}

\begin{frame}{Contenido}
\tableofcontents %Imprime la tabla de contenido
\end{frame}

\section{Información preliminar}
\begin{frame}{Marcos}
    \begin{itemize}
        \begin{multicols}{2}
            \item $A$
            \item $(A, \leq)$
            \item $(A, \leq, \vee, 0)$ o $(A, \leq, \wedge, 1)$
            \item $(A, \leq, \bigvee, \bigwedge, 0, 1)$
        \end{multicols} 
        \end{itemize}
    
        \onslide<2->{Un marco es una retícula completa que cumple cierta ley distributiva (ley distributiva de marcos), es decir,}
        \onslide<3->{\[
        (A, \leq, \bigvee, \wedge, 0, 1), \quad a\wedge\bigvee X=\bigvee \{a\wedge x\mid x\in X\}
        \]}
    
        \onslide<4->\[
        \mathbf{Frm}=\left\{ \begin{array}{ll} A, & \mbox{ marcos}\\ \\  f, & \mbox{ morfismo de marcos} \end{array} \right.
        \]
\end{frame}

\begin{frame}{¿Por qué estudiamos los marcos?}
    \begin{itemize}
        \item Estructuras simples.
        \item<2-> Existen herramientas que facilitan el estudio de los marcos.
        \item<3-> Correspondencias biyectivas.
        \item<4-> Buen comportamiento categórico.
        \item<5-> \alert<8->{La topología de un espacio ($\mathcal{O}S)$ es un marco.}
        \item<6-> $\mathbf{Loc}=\mathbf{Frm}^{\op}$ está en relación con $\mathbf{Top}$.
    \end{itemize}        
    
    \onslide<7->\[
    U_A(x)=\{x\in \pt A\mid x\nleq p\}\qquad A\simeq \mathcal{O}S, A \mbox{ es espacial}.
    \]
\end{frame}

\begin{frame}{Cocientes en $\mathbf{Frm}$}
$\mathbf{Frm}$ proporciona correspondencias biyectivas interesantes
\onslide<2->{\[
\mbox{Congruencias}\leftrightarrow \mbox{Conjuntos implicativos}\leftrightarrow \mbox{\alert<3->{Núcleos}}
\]}

\onslide<4->{\begin{block}{Definición:}
Sea $A\in \mathbf{Frm}$ y $j\colon A\to A$, decimos que $j$ es un \emph{núcleo} si:
\begin{enumerate}
    \item $j$ infla.
    \item $j$ es monótona.
    \item $j$ es idempotente.
    \item $j$ respeta ínfimos finitos.
\end{enumerate}
\end{block}
\[
    NA= \mbox{núcleos de } A.
\]}
\end{frame}

\begin{frame}
\begin{block}{Definición:}
    Un cociente de un marco $A$ es un marco $B$ equipado con un morfismo suprayectivo $f\colon A\to B$.
\end{block}
\onslide<2->{Si $A\in \mathbf{Frm}$ y $j\in NA$, entonces $A_j\in \mathbf{Frm}$.
\[
A_j=\{a\in A\mid j(a)=a\}.
\]}
\begin{itemize}
\item<3-> $A_j$ es un cociente de $A$.
\item<4-> Existen cocientes interesantes que estudiar
\end{itemize}
\end{frame}

\begin{frame}{Algunos tipos de cocientes}
    $a\in A\in \mathbf{Frm}$ definimos
    \onslide<2->{\[
    u_a(x)=a\vee x, \quad v_a(x)=a\succ x, \quad w_a(x)=((x\succ a)\succ a)
    \]
    $x\in A$}
    
    \begin{itemize}
    \item<3-> $A_{u_a}$ ``cociente cerrado'' 
    \item<4-> $A_{v_a}$ ``cociente abierto'' 
    \item<5-> $A_{w_a}$ ``cociente regular'' 
    \end{itemize}
    \end{frame}

\section{Intervalos de admisibilidad}
\begin{frame}{Filtros en $\mathbf{Frm}$}
    Sea $A\in \mathbf{Frm}$. Para $F\subseteq A$, decimos que $F$ es un \emph{filtro} si:
		\begin{enumerate}
			\item $1\in F$.
			\item $a\leq b$, $a\in  F$ $\Rightarrow b \in F$.
			\item $a, b \in F \Rightarrow  a \wedge b \in F$.
		\end{enumerate}
	\uncover<2->{Existen diferentes tipos de filtros:
	\begin{itemize}
	\begin{multicols}{2}
		\item Propio
		\item Primo
		\item Completamente primo
		\item \alert<3->{(Scott) abierto ($A^\wedge$)}
		\item \alert<3->{Admisible ($\nabla(j)$)}
		\end{multicols}
	\end{itemize} }
\end{frame}

\begin{frame}{Filtros de admisiblidad}
    Sea $j\in NA$. El \emph{filtro de admisibilidad} de $j$ es el conjunto
    \[
    \nabla(j)=\{a\in A\mid j(a)=1\}.
    \]
    \onslide<2->{\begin{block}{Observaciones:}
        \begin{itemize}
            \item<3-> $j, k\in NA$, $j\sim k\Leftrightarrow \nabla(j)=\nabla(k)$.
            \item<4-> $j\in NA$ es \emph{ajustado} si es el menor elemento de su bloque.
            \item<5-> $F\in A^\wedge \Rightarrow F=\nabla(j)$ para algún $j\in NA$.
            \item<6-> $F\in A^\wedge\Rightarrow [v_F, w_F]$.
        \end{itemize}
        \end{block}}
        
    \onslide<7->{\[
        f=\dot{\bigvee}\{v_a\mid a\in F\}, \quad v_F(a)=f^\infty(a), \quad w_F(a)=\bigwedge\{p\in M\mid a\leq p\}.
        \]}
    \onslide<8->{$M=\{m\in A\setminus F\mid m \mbox{ es máximo}\}$ y $M\subseteq S=\pt A$.}

\end{frame}

\begin{frame}{Intervalos de admisibilidad}
Si $F\in A^\wedge$, entonces 
\[
u_d\leq v_F \leq  w_F
\] 
para $d=v_F(0)$, $v_F=f^\infty$.

\begin{block}{Información con los intervalos}
\begin{itemize}
\item<2-> $[v_F, w_F]\subseteq NA$ es el intervalo de admisibilidad.
\item<3-> Bajo ciertas condiciones, los intervalos se comportan de manera particular:
\begin{itemize}
    \item<4-> Se puede colapsar el intervalo ($[v_F, w_F]=\{*\}$).
    \item<5-> Si $j\in [v_F, w_F]$, $j$ tiene una forma peculiar ($j=u_\bullet$, $\bullet\in A$).
\end{itemize}
\item<6-> ¿Qué significa que ocurra alguno de los casos anteriores?
\end{itemize}
\end{block}
\end{frame}

\begin{frame}{Relación con otras propiedades}
Sean $A\in \Frm$ y $F\in A^\wedge$.
\begin{itemize}
\item<2-> Si $A$ es \emph{arreglado}, $v_F=u_d$, para $d=v_F(0)$.
\item<3-> Si $A$ es \emph{ajustado}, $v_F=w_F=u_\bullet$ para algún $\bullet\in A$.
\item<4-> Si $A$ es \emph{fuertemente Hausdorff}, $u_\bullet=j\in [v_F,w_F]$ para algún $\bullet\in A$.
\end{itemize}
\onslide<5->\[
\mbox{¿Existen otras propiedades que se relacionen con los intervalos?}
\]
\end{frame}

\begin{frame}{Algo más técnico}
\begin{itemize}
\item $A\in\Frm, F\in A^\wedge$, entonces $[v_F, w_F]\subseteq NA$
\item<2-> $\mathcal{O}S\in \Frm, F\in \mathcal{O}S^\wedge$, entonces $[v_F, w_F]\subseteq N\mathcal{O}S$
\onslide<3->\[F=\nabla(Q), \mbox{ donde } Q\in \mathcal{Q}S\]

\item<4-> $\phi\colon [v_{\nabla(Q)},w_{\nabla(Q)}]\to [v_F, w_F]$
\onslide<5->\[
\mbox{¿Qué propiedades cumple $\phi$?}
\]
\end{itemize}
\end{frame}

\section{Marcos $KC$}
\begin{frame}{Cocientes compactos}
\begin{block}{Definición:}
Sea $A\in \mathbf{Frm}$. 
\begin{itemize}
\item<2-> Una \emph{cubierta} de $A$ es un subconjunto $X\subseteq A$ tal que $\bigvee X=1$. 
\item<3-> Una \emph{subcubierta} de $X$ es un subconjunto $Y\subseteq X$ tal que $\bigvee Y=1$.
\item <4-> $A$ es \emph{compacto} si toda cubierta tiene una subcubierta finita.
\end{itemize}
\end{block}
\onslide<5->{\[ 
A \mbox{ es compacto si }1 \mbox{ es compacto.}
\]}

\begin{itemize}
    \item<6-> $A_j$ es un cociente de $A$.
    \item<7-> $A_j$ es compacto $\Leftrightarrow$ $\nabla(j)\in A^\wedge$.
\end{itemize}
\end{frame}

\begin{frame}{Marcos $KC$}
$S\in \Top$ es $\mathrm{KC}$ si todo conjunto compacto es cerrado. $S$ es $\mathrm{US}$ si cada sucesión convergente tiene exactamente un límite al cual converge.
\[
T_2\Rightarrow KC\Rightarrow US\Rightarrow T_1
\]

\onslide<2->{\begin{dfn}
$A\in \Frm$ es $KC$ si todo cociente compacto de $A$ es cerrado. 
\end{dfn}}
\onslide<3->{Equivalentemente
\[
A_j=A_{u_\bullet}
\]
para algún $\bullet\in A$, $F\in A^\wedge$ y $j\in [v_F, w_F]$.}
\end{frame}

\begin{frame}[fragile]{Propiedades de los marcos $KC$}
\begin{itemize}
\item $KC\Rightarrow$ Arreglado

\item<2-> Si $A$ es $KC$ entonces $A_j$ es $KC$ para todo $j\in NA$.


\item<3-> Si $A$ es $KC$, entonces $A$ es $T_1$.
\end{itemize}
\onslide<4->{De hecho}
\onslide<5->\[\begin{tikzcd}
	{\mathbf{(reg)}} & {\mathbf{(fH)}} & KC & {\mbox{Arreglado}} & {T_1} \\
	& {\mathbf{(aju)}} && {}
	\arrow[Rightarrow, from=1-1, to=1-2]
	\arrow[Rightarrow, from=1-1, to=2-2]
	\arrow[Rightarrow, from=1-2, to=1-3]
	\arrow[Rightarrow, from=1-3, to=1-4]
	\arrow[Rightarrow, from=1-4, to=1-5]
	\arrow[Rightarrow, from=2-2, to=1-3]
\end{tikzcd}\]
\onslide<6->\[
\mbox{¿Como se relaciona $KC$ con otras propiedades en }\Frm?
\]
\end{frame}

\begin{frame}{$KC$ y su relación con los intervalos}
\begin{itemize}
    \item $A\in \Frm$, $F\in A^\wedge$, entonces $[v_F, w_F]\subseteq NA$.
    \item $F\in A^\wedge$, entonces para todo $j\in [v_F, w_F]$, $A_j$ es compacto. 
    \item<2-> Si $A$ es $KC$, entonces $A_j$ es compacto y cerrado
    \item<3-> Si $A$ es $KC$, entonces para todo $j\in [v_F, w_F]$, $A_j$ es compacto y cerrado.
\end{itemize}
\onslide<4->\[
\mbox{Bajo }KC\mbox{ los intervalos de admisibilidad están conformados por }u_\bullet
\]
\end{frame}

\begin{frame}{¿Para qué usamos los marcos $KC$}
Si $S\in \Top$ y es $T_2$, entonces todo conjunto compacto (saturado) es cerrado.

\onslide<2->\[
T_2\Rightarrow \mbox{Empaquetado}\Rightarrow T_1
\]

\begin{itemize}
    \item<3-> En $\Top:$ $KC\Leftrightarrow$ Empaquetado
    \item<4-> En $\Frm:$ $KC\Rightarrow$ Arreglado
\end{itemize}

\onslide<5->\[
\mathcal{O}S\mbox{ es }KC\Leftrightarrow S \mbox{ es empaquetado}
\]
\onslide<6->\[
\mbox{¿Qué pasa en el caso no espacial?}
\]
\end{frame}

\section{Ejemplos}
\begin{frame}{Complejidad del intervalo}
Si $A\in \Frm$ y $F\in A^\wedge$, entonces $[v_F, w_F]\subseteq NA$.
\onslide<2->{\[
I_F=[v_F(0), w_F(0)]\subseteq A\quad \mbox{ y }\quad j_a=v_F\vee u_a
\]
para $a\in I_F$}
\onslide<3->{
    \[
    \begin{split}
I_F& \to [v_F, w_F]\\
a &\mapsto j_a
    \end{split}
    \]
}
\onslide<4->{$I_F$ indica la complejidad del intervalo de admisibilidad $[v_F, w_F]$.}
\onslide<5->\[
\mbox{¿Qué pasa si } A \mbox{ es espacial}?
\]
\end{frame}

\begin{frame}
\begin{dfn}
Sea $S\in \Top$ y $E\in \mathcal{P}S$. Definimos $[E]\colon \mathcal{O}S\to \mathcal{O}S$ por
\[
[E](U)=(E\cup U)^\circ.
\]
\end{dfn}
\begin{itemize}
    \item<2-> $[E]\in N\mathcal{O}S$.
    \item<3-> Si $j\in N\mathcal{O}S$ tiene la forma de $[E]$, se dice que es un \emph{núcleo espacialmente inducido}. 
\end{itemize}
\onslide<4->{Si $A=\mathcal{O}S$ y $Q\in \mathcal{Q}S$ 
\[
v_F\leq [Q']\leq [M']=w_F
\]}
\onslide<5->{Si $S$ es $T_1$, $Q=M$.}
\end{frame}

\begin{frame}{La topología máximo compacta}
Consideremos $S=\{x,y\}\cup \mathbb{N}^2$ con $x,y\notin \mathbb{N}^2$ y sea 
\[
R_n=\{(m,n)\mid m\in \mathbb{N}\}
\]
\onslide<2->{Definimos 
\[
\mathcal{O}S=\mathcal{P}\mathbb{N}^2 \cup \mathcal{U}\cup\mathcal{V}
\]
donde
\[
\mathcal{U}=\{U\subseteq S\mid x\in U\mbox{ y }\forall n\in \mathbb{N}, U\cap R_n\mbox{ es cofinito}\}
\]
\[
\mathcal{V}=\{V\subseteq S\mid y\in V \mbox{ y }\exists F\subseteq \mathbb{N}\mbox{ finito tal que }\forall n\notin F, R_n\subseteq V \}
\]}
\onslide<3->{$\mathcal{O}S$ es una topología y es un marco $KC$ que no es $\mathbf{(H)}$.}
\onslide<4->\[
¿\mathbf{(H)}\Rightarrow KC?
\]
\end{frame}

\End

\section*{\textsc{Referencias}}
\begin{frame}[allowframebreaks]
\frametitle{Bibliografía}
\begin{thebibliography}{20}\markboth{Bibliografía}{Bibliografía}

\bibitem{P.T.} P. T. Johnstone, \textit{Stone spaces}, Cambridge Studies in Advanced Mathematics, vol. 3, Cambridge University Press, Cambridge, 1982. MR 698074

%\bibitem{J.M.} J. Monter; A. Zaldívar, \textit{El enfoque locálico de las reflexiones booleanas: un análisis en la categoría de marcos} [tesis de maestría], 2022. Universidad de Guadalajara.

%\bibitem{P.S.} J. Paseka and B. Smarda, \textit{$ T_2 $-frames and almost compact frames.} Czechoslovak Mathematical Journal (1992), 42(3), 385-402.

\bibitem{J.P.} J. Picado and A. Pultr, \textit{Frames and locales: Topology without points}, Frontiers in Mathematics, Springer Basel, 2012.

\bibitem{J.P.2} J. Picado and A. Pultr, \textit{Separation in point-free topology}, Springer, 2021.

\bibitem{R.S.} RA Sexton, \textit{A point free and point-sensitive analysis of the patch assembly}, The University of Manchester (United Kingdom), 2003.

\bibitem{R.S.2} RA Sexton, \textit{Frame theoretic assembly as a unifying construct}, The University of Manchester (United Kingdom), 2000.

\bibitem{R.S.3} RA Sexton and H. Simmons, \textit{Point-sensitive and point-free patch constructions}, Journal of Pure and Applied Algebra \textbf{207} (2006), no. 2, 433-468.

\bibitem{H.S.} H. Simmons, \textit{An Introduction to Frame Theory}, lecture notes, University of Manchester. Disponible en línea en \url{https://web.archive.org/web/20190714073511/http://staff.cs.manchester.ac.uk/~hsimmons}.

\bibitem{H.S.R} H. Simmons, \textit{Regularity, fitness, and the block structure of frames.} Applied Categorical Structures 14 (2006): 1-34.

%\bibitem{H.S.4} H. Simmons, \textit{The lattice theoretic part of topological separation properties}, Proceedings of the Edinburgh Mathematical Society, vol.~21, pp.~41--48, 1978.

%\bibitem{H.S.V} H. Simmons, \textit{The Vietoris modifications of a frame}. Unpublished manuscript (2004), 79pp., available online at http://www. cs. man. ac. uk/hsimmons.

\bibitem{A.W.} A. Wilansky, \textit{Between T1 and T2}, MONTHLY (1967): 261-266.

\bibitem{A.Z.} A. Zaldívar, \textit{Introducción a la teoría de marcos} [notas curso], 2025. Universidad de Guadalajara.
\end{thebibliography}
\end{frame}

\begin{frame}[plain,standout]
      \centering
      \Huge{\smiley Gracias por su atención\smiley}
      %In combination with \textit{plain}, \\
      %it makes a nice thank-you slide!
      \vfill
      \scalebox{4}{\faGithub} \par\bigskip
      \url{https://github.com/JCmonter/Apuntes/tree/main/Presentaciones} \\
      %\url{https://ctan.org/pkg/beamertheme-arguelles}
\end{frame}

\end{document}