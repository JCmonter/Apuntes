\documentclass[compress,12pt]{beamer}

\usetheme{Arguelles}
\usepackage{graphicx}
\usepackage{caption}
\usepackage[spanish,es-noshorthands]{babel}
%\usepackage{babel} 
\usepackage[pages=some]{background}
\usepackage{tikz}
\usepackage{tikz-cd}
\usepackage{amsmath,amssymb,latexsym,amscd} 
\usepackage[all,cmtip]{xy}
\usepackage{fancyhdr}
\usepackage{mathalfa}
\usepackage{mathrsfs}
\usetikzlibrary{babel}
\usepackage{hyperref}
\usepackage{ragged2e}
\usepackage{wasysym}
\usepackage{tikz}
\usetikzlibrary{arrows.meta,calc,positioning,overlay-beamer-styles,shadows.blur}
%\hypersetup{colorlinks=true,linkcolor=blue,citecolor=brown,linktocpage=true,pagebackref=true,hyperindex=true}
\makeatletter
\tikzset{
  invisible/.style={opacity=0},
  visible on/.style={alt=#1{}{invisible}},
  alt/.code args={<#1>#2#3}{\alt<#1>{\pgfkeysalso{#2}}{\pgfkeysalso{#3}}}
}
\makeatother
\pagenumbering{arabic}

\DeclareMathOperator{\op}{op}
\DeclareMathOperator{\pt}{pt}
\DeclareMathOperator{\id}{id}
\DeclareMathOperator{\spec}{spec}
\DeclareMathOperator{\Fit}{Fit}
\DeclareMathOperator{\Pth}{P}
\DeclareMathOperator{\Frm}{Frm}
\DeclareMathOperator{\Loc}{Loc}
\DeclareMathOperator{\Top}{Top}
\DeclareMathOperator{\Ord}{Ord}
\DeclareMathOperator{\Obj}{Obj}
\DeclareMathOperator{\Hom}{Hom}

\title{Modificaciones de parches}
\subtitle{y algunos axiomas de separación en la topología sin puntos}
%\event{58° Congreso Nacional de la SMM}
\date{\today}
\author{Juan Carlos Monter Cortés \\ Director: Dr. Luis Ángel Zaldívar Corichi}
\institute{Universidad de Guadalajara}
%\email{juan.monter2902@alumnos.udg.mx}

%\homepage{www.mywebsite.com}
%\github{username}

\begin{document}

\frame[plain]{\titlepage}

\section{Preliminares}
\begin{frame}{La construcción de parches}
\centering
\resizebox{0.98\textwidth}{!}{%
\begin{tikzpicture}[
  x=0.90cm, y=1cm,
  >=Stealth,
  year/.style={font=\normalsize\bfseries},
  base/.style={
    draw, rounded corners=2pt, fill=white, align=left,
    inner sep=4pt, font=\footnotesize, text width=3.9cm
  },
  E1968/.style={base, fill=blue!3},
  E1982/.style={base, fill=green!5},
  E1996/.style={base, fill=violet!6},
  E2000/.style={base, fill=orange!6},
  E2013/.style={base, fill=red!5},
  EPres/.style={
    base,
    fill=blue!8,
    draw=blue!60!black,
    line width=0.9pt
  },
  tick/.style={circle, fill, minimum size=3.5pt, inner sep=0pt},
  axis/.style={line width=0.6pt, gray!60}
]

% --- Coordenadas (espaciado uniforme)
\coordinate (p68)   at (1,0);
\coordinate (p82)   at (4,0);
\coordinate (p96)   at (7,0);
\coordinate (p00)   at (10,0);
\coordinate (p13)   at (13,0);
\coordinate (pPres) at (16,0);

% --- Eje base
\draw[axis] (0,0) -- (17.2,0);

% --- Años / Presente
\node[tick, visible on=<1->] at (p68) {};
\node[year, visible on=<1->] at ($(p68)+(0,-0.6)$) {1968};

\node[tick, visible on=<2->] at (p82) {};
\node[year, visible on=<2->] at ($(p82)+(0,0.7)$) {1982};

\node[tick, visible on=<3->] at (p96) {};
\node[year, visible on=<3->] at ($(p96)+(0,-0.6)$) {1996};

\node[tick, visible on=<4->] at (p00) {};
\node[year, visible on=<4->] at ($(p00)+(0,0.7)$) {2000};

\node[tick, visible on=<5->] at (p13) {};
\node[year, visible on=<5->] at ($(p13)+(0,-0.6)$) {2013};

\node[tick, visible on=<6->] at (pPres) {};
\node[year, visible on=<6->] at ($(pPres)+(0,0.7)$) {Presente};

% --- Eventos
\node[E1968, anchor=south, visible on=<1->] (e1968) at ($(p68)+(0,1.25)$)
  {\textbf{Hochster} define la \emph{topología de parches}};

\node[E1982, anchor=north, visible on=<2->] (e1982) at ($(p82)+(0,-1.25)$)
  {\textbf{Priestley, Cornish, Joyal y Stralka} usan el parche de manera más general
   (ver Johnstone, \emph{Stone Spaces}, \cite{P.T.})};

\node[E1996, anchor=south, visible on=<3->] (e1996) at ($(p96)+(0,1.35)$)
  {\textbf{Escard\'o} desarrolla el \emph{parche continuo} para retículas};

\node[E2000, anchor=north, visible on=<4->] (e2000) at ($(p00)+(0,-1.25)$)
  {\textbf{Sexton} introduce el \emph{marco de parches}};

\node[E2013, anchor=south, visible on=<5->] (e2013) at ($(p13)+(0,1.35)$)
  {\textbf{Klinke} trabaja con una clase particular de marcos
   y extiende el parche};

% --- Presente (destacado)
\node[EPres, anchor=north, visible on=<6->] (ePres) at ($(pPres)+(0,-1.25)$)
  {Estudiamos el parche definido por \textbf{Sexton}
   y su relación con los axiomas de separación en \textbf{Frm}.};

% --- Flechas
\draw[->, thick, visible on=<1->] (e1968.south) -- ($(p68)+(0,0.08)$);
\draw[->, thick, visible on=<2->] (e1982.north) -- ($(p82)+(0,-0.08)$);
\draw[->, thick, visible on=<3->] (e1996.south) -- ($(p96)+(0,0.08)$);
\draw[->, thick, visible on=<4->] (e2000.north) -- ($(p00)+(0,-0.08)$);
\draw[->, thick, visible on=<5->] (e2013.south) -- ($(p13)+(0,0.08)$);
\draw[->, thick, visible on=<6->] (ePres.north) -- ($(pPres)+(0,-0.08)$);

\end{tikzpicture}%
}
\end{frame}

\begin{frame}[fragile]{Teoría de marcos}
\[
\Frm= \left\{ \begin{array}{lc} \Obj: & (A, \leq, \wedge, \bigvee, 1, 0) \\ \\ \mbox{Flechas:} & f\colon A\to B  \end{array} \right.
\]

Para $S\in \Top$,
\[
(\mathcal{O}S, \subseteq, \cap, \bigcup, S, \emptyset)\in \Frm
\]
Además,
\[
\begin{tikzcd}
\Top \arrow[rr, "\mathcal{O}( \_ )", bend left] & \perp & \Frm \arrow[ll, "\pt( \_ )", bend left]
\end{tikzcd}
\]
es una adjunción.
\end{frame}

\begin{frame}{La relación entre Top, Loc y Frm}
\[
\Loc=\Frm^{\op}
\qquad\text{y}\qquad
f\in\Frm\ \Longrightarrow\ f_* \in\Loc.
\]

\vspace{1.0ex}

\centering
\resizebox{0.92\textwidth}{!}{%
\begin{tikzpicture}[
  >=Stealth,
  every node/.style={font=\normalsize},
  col/.style={draw, rounded corners=3pt, fill=gray!6, minimum width=4.4cm},
  head/.style={font=\Large\bfseries, fill=gray!15},
  item/.style={align=left, text width=4.3cm},
  arr/.style={->, thick, gray!60}
]

% --- Encabezados
\node[col, head] (top) at (0,0) {Top};
\node[col, head] (loc) at (5,0) {Loc};
\node[col, head] (frm) at (10,0) {Frm};

% --- Fila 1: Subespacio / Sublocal / Cociente
\node[item] at (0,-1.4)
  {\textbf{Subespacio}\\
   $X \subseteq S$};

\node[item] at (5,-1.4)
  {\textbf{Sublocal}\\
   $M \hookrightarrow L$};

\node[item] at (10,-1.4)
  {\textbf{Cociente}\\
   $A \rightarrow B$};

% --- Flechas (relación conceptual)
\draw[arr] (top.east) -- (loc.west);
\draw[arr] (loc.east) -- (frm.west);

\end{tikzpicture}%
}

\vspace{1.2ex}

El estudio de los \textbf{cocientes} en $\Frm$ se realiza a través de los \textbf{núcleos}.

\end{frame}

\begin{frame}{Núcleos y cocientes}
\begin{block}{Definición}
Sea $j\colon A\to A$. Decimos que $j$ es un \textbf{núcleo} si:
\begin{columns}[T,onlytextwidth]

\begin{column}{0.48\textwidth}
\begin{itemize}
  \item[\textbf{(N1)}] $a \le j(a)$.
  \item[\textbf{(N2)}] si $a \le b\Rightarrow j(a) \le j(b)$.
\end{itemize}
\end{column}

\begin{column}{0.48\textwidth}
\begin{itemize}
  \item[\textbf{(N3)}] $j(j(a)) = j(a)$.
  \item[\textbf{(N4)}] $j(a \wedge b) = j(a) \wedge j(b)$.
\end{itemize}
\end{column}

\end{columns}
\end{block}

$NA=\{\mbox{núcleos en }A\}$. Para $a\in A$, definimos 
\[
u_a(x)=a\vee x\quad v_a(x)=(a\succ x)\quad w_a(x)=((x\succ a)\succ a)
\]
y $u_a, v_a, w_a\in NA$.\footnotetext[0]{Recordemos que $a\succ x=\bigvee\{c\in A\mid c\wedge a \le x\}$.}
\end{frame}

\begin{frame}
\begin{block}{Definición}
Sea $A\in \Frm$. Un \textbf{cociente} de $A$ es un marco $B$ y un morfismo 
\[
f\colon A\to B
\]
suprayectivo.
\end{block}
\onslide<2->{Si $j\in NA$ y $A_j=\{a\in A\mid j(a)=a\}$, entonces
\[
j^*\colon A\to A_j, \mbox{ dado por } j^*(a)=j(a), \mbox{ es suprayectivo}\; \mbox{ y }\; A_j\in \Frm.
\]
$A_j$ es el \emph{marco cociente}.} \onslide<3->{En particular,
\[
A_{u_a}=\mbox{c. cerrado}, \quad A_{v_a}=\mbox{c. abierto},\quad A_{w_a}=\mbox{c. regular}.
\]}
\end{frame}

\begin{frame}{Filtros}
  \begin{block}{Definición}
    Sea $A\in \Frm$. Decimos que $F\subseteq A$ es un \textbf{filtro} si:
    \begin{enumerate}
      \item $1\in F$.
      \item Si $a\leq b$ y $a\in F$, entonces $b\in F$.
      \item Si $a, b\in F$, entonces $a\wedge b\in F$.
    \end{enumerate}
  \end{block}
  \onslide<2->{En particular, decimos que $F$ es un \textbf{filtro abierto} si: 
  \[
    X\subseteq A \mbox{ dirigido tal que }\bigvee X\in F\Rightarrow F\cap X\neq \emptyset.
  \]
  $A^\wedge=$Filtros abiertos en $A$.}
\end{frame}

\begin{frame}{Filtros de admisibilidad}
\begin{block}{Definición}
Sean $A\in \Frm$ y $j\in NA$. 
\begin{enumerate}
  \item Un filtro es \textbf{admisible} si tiene la forma
  \[
    \nabla(j)=\{a\in A\mid j(a)=1\}.
  \]
  \item<2-> Para $j, k\in NA$ definimos 
  \[
    j\sim k\iff \nabla(j)=\nabla(k)
  \]
  \item<3-> Decimos que $f$ es un \textbf{núcleo ajustado} si es el menor elemento de su bloque. Equivalentemente
  \[
    f \mbox{ es ajustado} \iff f=\bigvee\{v_a\mid a\in A\}.
  \]
\end{enumerate}
\end{block}
\end{frame}

\begin{frame}{El Teorema de Hofmann-Mislove}
\begin{block}{Proposición}
\begin{enumerate}
  \item Si $F\in A^\wedge$, entonces $F=\nabla(j)$.
  \item $F\in A^\wedge$ si y solo si $A_j$ es compacto\footnotemark[1]. 
  \item $F\in A^\wedge$ si y solo $v_F=f_F^\infty$ es el menor núcleo que admite a $F$
\end{enumerate}
\end{block}

  \only<1>{
    \begin{block}{Teorema [Hofmann-Mislove]}
      Sea $A\in \Frm$ y $S=\pt A\in \Top$. Existe una correspondencia biyectiva entre $Q\in \mathcal{Q}S$ y $F\in A^\wedge$.
    \end{block}
  }

  % Segunda versión (extendida)
  \only<2->{
    \begin{block}{Teorema [Hofmann-Mislove-Johnstone]}
      Sea $A\in \Frm$ y $S=\pt A\in \Top$. Existe una correspondencia biyectiva entre $Q\in \mathcal{Q}S$ y núcleos ajustados.
    \end{block}}
\onslide<3->\[
Q\longleftrightarrow F\longleftrightarrow v_F
\]
  \footnotetext[0]{$f_F=\dot{\bigvee}=\{v_a\mid a\in F\}$.  $f_F^0=id, f_F^{\alpha +1}=f_F\circ (f_F^\alpha)$ y $f_F^\lambda=\bigvee\{f_F^\alpha\mid \alpha< \lambda\}$.}
  \footnotetext[1]{$A\in \Frm$ es compacto si y solo si para $X\subseteq A$, $1=\bigvee X$.}
\end{frame}

\begin{frame}[fragile]{Espacio de parches}
    Consideremos $S\in \Top$. Denotamos por $^pS=(S, \mathcal{O}^pS)$ al \textbf{espacio de parches} de $S$, donde $\mathcal{O}^pS$ está generado por
    \[
    \mbox{pbase}=\{U\cap Q'\mid U\in \mathcal{O}S, Q\in \mathcal{Q}S\}
    \]
   \[\begin{tikzcd}
	{\mathcal{O}S} & {\mathcal{O}^pS}
	\arrow[from=1-1, to=1-2]
\end{tikzcd}\]
\begin{block}{Definición}
$S$ es \textbf{empaquetado} si todo subconjunto compacto (saturado\footnotemark[2]) es cerrado
\[
S \mbox{ es empaquetado}\iff\, ^pS=S\quad \mbox{ y }\quad T_2\Rightarrow \mbox{empaquetado}\Rightarrow T_1
\]
\end{block}
\footnotetext[2]{$E\subseteq S$ es \emph{saturado} si $E=\bigcap \{U\in \mathcal{O}S\mid E\subseteq U\}$.}
\end{frame}

\begin{frame}[fragile]{Marco de parches}
Consideremos $A\in \Frm$. Denotamos por $PA$ al \textbf{marco de parches} de $A$, donde $PA$ está generado por
\[
\mbox{Pbase}=\{u_a\wedge v_F\mid a\in A, F\in A^\wedge\}
\]
\begin{tikzcd}
A \arrow[r] \arrow[rr, "\eta_A", bend left] & PA \arrow[r] & NA &  & \eta_A(a)=u_a
\end{tikzcd}
\begin{block}{Definición}
    $A$ es \textbf{parche trivial} si y solo si $A\cong PA$.
      \[
        ?\Rightarrow \mbox{ parche trivial }\Rightarrow T_1
      \]
\end{block}
\alert{\textbf{¿Cuándo ocurre que $A\cong PA$?}}
\end{frame}

\begin{frame}{El diccionario}

\begin{columns}[T,onlytextwidth]
  \column{0.48\textwidth}
  \begin{block}{\centering $\Top$}
    \begin{itemize}
      \item<2-> Espacio de parches ($^pS$)
      \item<4-> pbase$=\{U\cap Q'\}$ 
      \item<6-> Empaquetado ($^pS=S$)
      \item<8-> $Q\in \mathcal{Q}S\Rightarrow Q\in \mathcal{C}S$
      \item<10-> $^{pp}S=\,^{ppp}S$
      \item<12-> $T_2\Rightarrow $Empaquetado
      \item<14-> Subespacios compactos cerrados
    \end{itemize}
  \end{block}

  \column{0.48\textwidth}
  \begin{block}{\centering $\Frm$}
    \begin{itemize}
      \item<3-> Marco de parches ($PA$)
      \item<5-> Pbase$=\{u_a\wedge v_F\}$
      \item<7-> Parche trivial ($PA\cong A$)
      \item<9-> $u_d=v_F$
      \item<11-> ¿$PPA=PPPA$?
      \item<13-> ¿$\mathbf{(H)}\Rightarrow $Parche trivial?
      \item<15-> \alert<16->{Cocientes compactos cerrados.}
    \end{itemize}
  \end{block}
\end{columns}
\end{frame}

\begin{frame}{Marcos eficientes}
\begin{block}{Definición [\cite{R.S.3}, Def. 8.2.1]}
Sean $A\in \Frm$, $F\in A^\wedge$ y $\alpha\in \Ord$. Decimos que:
\begin{enumerate}
    \item<2-> $F$ es $\alpha$\textbf{-eficiente} si para $x\in F$, $d\vee x=1$, donde
    \[
    d:=d(\alpha)=f^\alpha(0).
    \]
    \item<3-> $A$ es $\alpha$\textbf{-eficiente} si cada $F\in A^\wedge$ es $\alpha$-eficiente.
    \item<4-> $A$ es \textbf{eficiente} si es $\alpha$-eficiente para algún $\alpha\in \Ord$.
\end{enumerate}
\end{block}

\onslide<5->{\begin{block}{Proposición [\cite{R.S.3}, Lema 8.2.2]}
\[
    A \mbox{ es eficiente}\quad \iff\quad A \mbox{ es parche trivial}.
\]
\end{block}}
\end{frame}

\begin{frame}{Axiomas de separación en Frm}
Para cualesquiera $a\nleq b\in A$ tenemos que $A$ es:
\begin{itemize}
    \item \textbf{(reg):} si $\exists \, x,y\in A$ tales que $a\vee x=1, y\nleq b$ y $x\wedge y=0$.
    \item \textbf{(H):} si $\exists\, c\in A$ tal que $c\nleq a$ y $\neg c\leq b$.
    \item \textbf{(aju):} si $\exists\, x,y\in A$ tales que $x\vee a=1, y\nleq b$ y $x\wedge y\leq b$.
    \item \textbf{(saju):} si $\exists\, c\in A$ tal que $c\vee a=1\neq c\vee b$.
    \item \textbf{(fH):} si el sublocal diagonal es cerrado.
    \item $T_1:$ si $p\in \pt A$, $p$ es máximo.
  \end{itemize}
\end{frame}

\begin{frame}[fragile]{Axiomas de separación en Frm}
\[\begin{tikzcd}
	& {\mathbf{(H)}+\mathbf{(saju)}} \\
	{\mathbf{(reg)}} & {\mathbf{(fH)}} & {\mathbf{(H)}} & {T_1} \\
	& {\mathbf{(aju)}} & {\mathbf{(saju)}}
	\arrow[Rightarrow, from=1-2, to=2-3]
	\arrow[Rightarrow, from=2-1, to=1-2]
	\arrow[Rightarrow, from=2-1, to=2-2]
	\arrow[Rightarrow, from=2-1, to=3-2]
	\arrow[Rightarrow, from=2-2, to=2-3]
	\arrow[Rightarrow, from=2-3, to=2-4]
	\arrow[Rightarrow, from=3-2, to=3-3]
\end{tikzcd}\]

\alert{\textbf{¿Qué relación existe entre los axiomas de separación y los marcos eficientes?}}
\end{frame}

\begin{frame}{Propiedades de los marcos eficientes}
Este es in resumen de las propiedades que Sexton menciona en \cite{R.S.3}
\begin{itemize}
\item En el caso espacial ($A=\mathcal{O}S$),
\[
\begin{split}
\mathcal{O}S \mbox{ es 0-eficiente } & \iff S=\emptyset \\
\mathcal{O}S \mbox{ es 1-eficiente } & \iff S \mbox{ is }T_2 \\
\mathcal{O}S \mbox{ es eficiente } & \iff S \mbox{ es empaquetado y apilado}.
\end{split}
\]

\item Para $A\in \Frm$ arbitrario
\[
\begin{split}
A\mbox{ es }\mathbf{(reg)} & \Rightarrow A \mbox{ es eficiente}\\
A\mbox{ es eficiente } & \Rightarrow A\mbox{ es } T_1
\end{split}
\]
\[
\boxed{\alert{A\mbox{ es }\mathbf{(aju)} \Rightarrow A \mbox{ es eficiente}}}
\]
\end{itemize}
\end{frame}

\begin{frame}{Algunos ejemplos}
\begin{itemize}
\item Con la topología cofinita vemos que $PA=NA$.
\item<2-> Con la topología conumerable vemos que $\pt NA\subseteq \pt PA$.
\[
 \alert{\mbox{En ambos casos }A \mbox{ no es } 1\mbox{-eficiente y } PA \mbox{ es espacial.}}
\] 
\item<3-> La topología subregular de los reales proporciona un marco 1-eficiente que no es regular. 
\[
\alert{1\mbox{-eficiente}\nRightarrow \mathbf{(reg)}}
\]
\item<4-> Con la topología máximo compacta tenemos un marco 2-eficiente que no es 1-eficiente.
\[
\alert{2\mbox{-eficiente}\nRightarrow \mathbf{(H)}}
\]
\item<5-> La topología guía sobre un árbol muestra que existen marcos $\omega$-eficientes.
\[
\alert{\omega\mbox{-eficiente}\nRightarrow n\mbox{-eficiente para algún } n\in \mathbb{N}}
\]
\end{itemize}
\end{frame}

\begin{frame}[plain,standout]
  \vfill
  \centering
  \Huge Objetivos
  \vfill
\end{frame}

\begin{frame}[plain,standout]
\textbf{Objetivo principal:} Establecer la eficiencia como un axioma de separación libre de puntos.\\
\vspace{2ex}
\textbf{Objetivos específicos:}
\small{
\begin{enumerate}
        \item Comprender la noción de eficiencia con mayor detalle.
        \item Analizar el comportamiento del parche libre de puntos
        \item Explorar su relación con algunos axiomas de separación en $\Frm$.
        \item Investigar nociones libres de puntos y sensibles de puntos relacionadas con la eficiencia.
        \item Desarrollar herramientas que permitan el estudio de los marcos eficientes.
        \item Construir ejemplos que ilustren el comportamiento de los marcos eficientes y del marco de parches.
      \end{enumerate}}
\end{frame}

\section{Nuestro trabajo}

\begin{frame}{¿Qué significa $u_d=v_F$?}
Para $f=\dot{\bigvee}\{v_a\mid a\in F\}\notin NA$ y tomamos
\[
f^0=\id, \quad f^{\alpha +1}=f(f^\alpha),\quad f^\lambda=\bigvee\{f^\alpha\mid \alpha< \lambda\}.
\]
Consideramos $v_F=f^\infty\in NA$. Entonces
\begin{itemize}
  \item Si $d=v_F(0)\Rightarrow u_d\leq v_F$
  \item Si $x\in F$ tal que $u_d(x)=x\vee d=1\Rightarrow v_F\leq u_d$
  \item Bajo eficiencia obtenemos $F=\nabla(u_d)$.
  
\end{itemize}
\end{frame}

\begin{frame}{¿Qué significa ser $\alpha$-eficiente?}
\[
1-\mbox{Eficiente}\Rightarrow 2-\mbox{Eficiente}\Rightarrow \cdots \Rightarrow \alpha-\mbox{Eficiente}\Rightarrow \cdots \Rightarrow T_1
\]

\onslide<2->{
\begin{block}{Proposición}
Sean $A\in \Frm$ y $j, k\in NA$. 
\begin{enumerate}
  \item<3-> Si $j\leq k$, entonces $\nabla(j)\subseteq \nabla(k)$.
  \item<4-> Si $j$ es ajustado, se cumple que 
\[
j\leq k\iff \nabla(j)\subseteq \nabla(k).
\]
\end{enumerate}
\end{block}}
\end{frame}

\begin{frame}
Sean $A\in \Frm$ y $F\in A^\wedge$.
\begin{itemize}
\item<2-> Si $A$ es 1-eficiente, para $x\in F$ se cumple que $x\vee d(1)=1$, es decir,
\onslide<3->{\[
\begin{split}
u_{d(1)}(x)=1& \Rightarrow x\in \nabla(u_{d(1)})\Rightarrow \nabla(v_F)=F\subseteq \nabla(u_{d(1)})\\
& \Rightarrow v_F\leq u_{d(1)}\\
& \Rightarrow v_F(0)=d(\infty)\leq u_{d(1)}=d(1).\\
& \Rightarrow d(\infty)= d(1).
\end{split}
\]}
\onslide<4->{Además, 
\[
u_{d(1)}\leq u_d\leq v_F\Rightarrow \nabla(u_{d(1)})\subseteq \nabla(u_d)\subseteq \nabla(v_F)=F.
\]}
\end{itemize}
\onslide<5->{Por lo tanto $\nabla(u_{d(1)})=F$}.
\end{frame}

\begin{frame}
\begin{block}{Observaciones:}
\begin{itemize}
\item<2-> Bajo eficiencia, la sucesión $d(\alpha)$ se estabiliza en el grado de eficiencia.
\item<3-> Proporciona el menor núcleo cerrado que admite a un filtro abierto.
\item<4-> Para cada filtro abierto $F$, proporciona un cociente compacto cerrado.  
\item<5-> Entre más grande es el grado de eficiencia, más lejos está el marco de ser Hausdorff.
\end{itemize}
\end{block}
\end{frame}

\begin{frame}[fragile]{Algunos resultados}
Si $(f\colon A\to B)\in \Frm$, $G\subseteq A$ y $F\subseteq B$ son filtros, entonces
\[
b\in f[G]\iff f_*(b)\in G\quad\mbox{ y }\quad a\in f_*[F]\iff f(a)\in F.
\]
También, si $F\in B^\wedge$, entonces $f_*(F)\in A^\wedge$.
\begin{block}{Proposición}
Para $f^\infty$ y $f_j^\infty$ los núcleos asociados a $F\in A_j^\wedge$ y $j_*F\in A^\wedge$, respectivamente, tenemos que
\[
j\circ f_j^\infty\leq f^\infty \circ j
\]
\end{block}
\begin{proof}
Por inducción transfinita.
\end{proof}
\end{frame}

\begin{frame}{Más propiedades de los marcos eficientes}
\begin{block}{Proposición}
Si $A\in \Frm$ es eficiente y $j\in NA$, entonces $A_j$ es eficiente.
\end{block}
\begin{proof}
\begin{itemize}
\item Tomamos $x\in F\in A_j^\wedge$ y $F\subseteq j_*[F]\in A^\wedge$.
\item Ya que $A$ es eficiente, entonces $d\vee x=1$, para todo $x\in j_*[F]$. En particular, para todo $x\in F$.
\item Para $f^\infty$ y $f_j^\infty$ como antes, tenemos
\[
d=d(\alpha)\leq d_j(\alpha)=d_j
\]
\item Por lo tanto, $d_j\vee x=1$.
\end{itemize}
\end{proof}

%\begin{block}{Proposition}
%If $A, B\in \Frm$ are tidy, then $A\oplus B$ is tidy.
%\end{block}
\end{frame}

\begin{frame}
\begin{block}{Proposición}
Si $\{A_i\}_{i\in I}$ es una familia de marcos eficientes, entonces $\bigoplus_{i\in I}A_i$ es un marco eficiente.
\end{block}

\begin{proof}
Sabemos que $\bigoplus A_i\in \Frm$ y $(\iota_i\colon A_i\to \bigoplus A_i)\in Frm$. Entonces
\begin{itemize}
\item Existe $(\iota_i)_*$ y para $F\in (\bigoplus A_i)^\wedge$, $(\iota_i)_*[F]\in A_i^\wedge$.
\item Consideramos $\alpha=\sup\{\alpha_i\}$ como el grado de eficiencia de cada $A_i$.
\item Por la eficiencia, si $x_i\in (\iota_i)_*[F]$, entonces $x_i\vee d_i=1$.
\item Consideramos $\langle x_i \rangle\in F$.
\item Verificamos que $\langle d_i\rangle\leq f_F^\alpha(\iota_i(0))=d(\alpha)=d$.
\item Por lo tanto $\langle x_i\rangle \vee d=1$.
\end{itemize}
\end{proof}
\end{frame}

\begin{frame}[fragile]
%\begin{block}{Corolario}
%La subcategoría de marcos eficientes es correflexiva.
%\end{block}

\begin{block}{Corolario}
Si $A$ es $\mathbf{(fH)}$, $A$ es eficiente.
\end{block}

\[\begin{tikzcd}
	& {\mathbf{(H)}} \\
	{\mathbf{(reg)}} & {\mathbf{(fH)}} & {\text{Eficiente}} \\
	& {\mathbf{(aju)}}
	\arrow[Rightarrow, dotted, from=1-2, to=2-3]
	\arrow[Rightarrow, from=2-1, to=1-2]
	\arrow[Rightarrow, from=2-1, to=2-2]
	\arrow[Rightarrow, from=2-1, to=3-2]
	\arrow[Rightarrow, from=2-2, to=1-2]
	\arrow[Rightarrow, from=2-2, to=2-3]
	\arrow[Rightarrow, from=3-2, to=2-3]
\end{tikzcd}\]
\end{frame}

\begin{frame}{Cocientes compactos}
  \[
  \mbox{Eficiente} \iff \mbox{ P. trivial }\iff u_d=v_F
  \]

  Notemos que
  \[
  A_{u_d} \mbox{ es un cociente cerrado}\quad \mbox{ y }\quad A_{v_F} \mbox{ es un cociente compacto.}
  \]
  Si $A$ es eficiente, tenemos un cociente compacto y cerrado.
\vspace{1.2ex}

  En \cite{A.W.}, Wilansky menciona que $S\in \Top$ es \textbf{KC} si cada subconjunto compacto es cerrado. 
\end{frame}

\begin{frame}{Marcos KC}
\begin{block}{Definición}
$A\in \Frm$ es un \textbf{marco KC} si cada cociente compacto es cerrado.
\end{block}
\[
\mathbf{KC}\Rightarrow \mbox{Eficiente}
\]
\alert{\textbf{¿Existen marcos que son eficientes y no son KC?}}

\begin{block}{Proposición:} 
Para $j\in NA$ y $k\in NA_j$. Si $\nabla(k)\in A_j^\wedge$, $\nabla(j_*kj)\in A^\wedge$.
\end{block}
\end{frame}

\begin{frame}
\begin{block}{Proposición}
Si $A\in \Frm$ es $\mathbf{KC}$ y $j\in NA$, entonces $A_j$ es $\mathbf{KC}$.
\end{block}
\begin{proof}
\begin{itemize}
  \item Consideramos $k\in NA_j$ tal que $\nabla(k)\in A_j^\wedge$.
  \item Si $\nabla(k)\in A_j^\wedge \Rightarrow j_*[\nabla(k)]\in A^\wedge$.
  \item Tomamos $l=j_*\circ k\circ j\in NA$ y $\nabla(l)\in A^\wedge\Rightarrow l=u_a$ para algún $a\in A$.
  \item Además $a=k(j(a))$.
  \item Para $x, b\in A_j$ con $b=j(a)$ tenemos $u_b(x)=k(x)$. 
\end{itemize}
\end{proof}
\end{frame}
\begin{frame}
\begin{block}{Definición:}
  \begin{enumerate}
    \item Para $\mathcal{P}$ una propiedad que satisface $A\in Frm$ decimos que $\mathcal{P}$ es conservativa si $S=\pt A$ satisface la propiedad $\mathcal{P}_s$. 
    \item Para $\mathcal{P}_s$ una propiedad que satisface $S\in \Top$ decimos que $\mathcal{P}_s$ es conservativa si $A=\mathcal{O}S$ satisface la propiedad $\mathcal{P}$.
  \end{enumerate}
\end{block}

\begin{itemize}
\item $\mathbf{(reg)}$, $\mathbf{(H)}$, y $T_1$ son conservativas.
\item $\mathbf{(fH)}$ no es conservativa ($T_2\nRightarrow \mathbf{(fH)}$).
\item Eficiente no es conservativa (empaquetado $\nRightarrow$ eficiente).
\item En el caso espacial, $\mathbf{KC}=$empaquetado.
\end{itemize}
\end{frame}

\begin{frame}[fragile]
Podemos construir el diagrama (ver \cite{H.S.V})
\[
\begin{tikzcd}
  |[visible on=<2->]| A
    && |[visible on=<5->]| A_F \\
  &&& |[visible on=<10->]| \mathcal{O}Q \\
  |[visible on=<3->]| \mathcal{O}S
    && |[visible on=<7->]| \mathcal{O}S_\nabla
%
  \arrow["{U_A}"',      from=1-1, to=3-1, visible on=<4->]
  \arrow["{v_F}",       from=1-1, to=1-3, visible on=<6->]
  \arrow["{v_\nabla}"', from=3-1, to=3-3, visible on=<8->]
  \arrow["g",           from=1-3, to=3-3, visible on=<9->]
  \arrow["{\simeq }"',  from=3-3, to=2-4, visible on=<11->]
  \arrow["{\simeq }",   from=1-3, to=2-4, visible on=<12->]
\end{tikzcd}
\]
\onslide<13->{donde $g=(U_A)_*\circ (v_\nabla)_{\mid A_F}$.
\[
\mbox{¿Qué pasa si } A \mbox{ tiene la propiedad }\mathbf{(H)}?
\]}
\end{frame}

\begin{frame}
\begin{block}{Teorema}
Sea $A$ un marco con la propiedad $\mathbf{(H)}$, entonces para cada $F\in A^\wedge$ y su correspondiente $Q\in \mathcal{Q}S$, tenemos
\[
\mathcal{O}Q\simeq \uparrow{Q'},
\]
es decir, el marco de abiertos del espacio de puntos de $A_F$ es isomorfo a un cociente compacto y cerrado de un espacio Hausdorff.
\end{block}
\end{frame}

\section{Plan de trabajo}
\begin{frame}[standout]{Cronograma de actividades (Al inicio)}

\begin{tiny}
\begin{center}
\begin{tabular}{|c|c|c|c|c|c|c|c|c|}
\hline
\multicolumn{8}{c}
{\textbf{Cronograma de actividades}}\\
\hline
\hline

\large{\textbf{Actividades}} & \rotatebox{90}{SEMESTRE 1 } & \rotatebox{90}{SEMESTRE 2 } & \rotatebox{90}{SEMESTRE 3 } & \rotatebox{90}{SEMESTRE 4 } & \rotatebox{90}{SEMESTRE 5 } & \rotatebox{90}{SEMESTRE 6 } & \rotatebox{90}{SEMESTRE 7 } & \rotatebox{90}{SEMESTRE 8 }\\\hline
\scriptsize{Revisión de bibliografía}  & \checkmark & \checkmark & \checkmark & \checkmark & \checkmark & \checkmark &  &  \\\hline
\scriptsize{Lectura de artículos}  & \checkmark & \checkmark & \checkmark & \checkmark & \checkmark & \checkmark &  &  \\\hline
\scriptsize{Generar de conjeturas} &  & \checkmark & \checkmark & \checkmark & \checkmark & \checkmark & & \\\hline
\scriptsize{Probar resultados} &  &  &  \checkmark & \checkmark & \checkmark & \checkmark & \checkmark & \\\hline
\scriptsize{Validación y rechazo de conjeturas} &  &  &  \checkmark & \checkmark & \checkmark & \checkmark & \checkmark & \\\hline
\scriptsize{Redacción de artículos y otros documentos} &  & \checkmark & \checkmark  & \checkmark  & \checkmark & \checkmark  & \checkmark & \checkmark \\\hline
\scriptsize{Desarrollar conclusiones} &  &  &  & & & \checkmark & \checkmark & \checkmark \\\hline
\scriptsize{Sustentación} &  &  &  & & & & & \checkmark \\\hline
\end{tabular}
\end{center}
\end{tiny}
\end{frame}

\begin{frame}[standout]{Cronograma de actividades (Particular)}
\begin{tiny}
\begin{center}
\begin{tabular}{|c|c|c|c|c|c|c|c|c|}
\hline
\multicolumn{8}{c}
{\textbf{Cronograma de actividades}}\\
\hline
\hline

\large{\textbf{Actividades}} & \rotatebox{90}{SEMESTRE 1 } & \rotatebox{90}{SEMESTRE 2 } & \rotatebox{90}{SEMESTRE 3 } & \rotatebox{90}{SEMESTRE 4 } & \rotatebox{90}{SEMESTRE 5 } & \rotatebox{90}{SEMESTRE 6 } & \rotatebox{90}{SEMESTRE 7 } & \rotatebox{90}{SEMESTRE 8 }\\\hline
\scriptsize{Aprender sobre construciones de parches}  & \checkmark & \checkmark & \checkmark & &  &  &  &  \\\hline
\scriptsize{Aprender sobre axiomas de separación}  & \checkmark & \checkmark & \checkmark & \checkmark & \checkmark & \checkmark &  &  \\\hline
\scriptsize{Entender ejemplos} &  & \checkmark &  \checkmark & \checkmark & \checkmark & \checkmark & \checkmark & \\\hline
\scriptsize{Consultar fuentes alternativas} &  & \checkmark & \checkmark & \checkmark & \checkmark & \checkmark & \checkmark & \\\hline
\scriptsize{Desarrollar teoría} &  &  &  \checkmark & \checkmark & \checkmark & \checkmark & \checkmark & \\\hline
\scriptsize{Probar resultados} &  &  &  \checkmark & \checkmark & \checkmark & \checkmark & \checkmark & \\\hline
\scriptsize{Validación y rechazo de conjeturas} &  &  &  \checkmark & \checkmark & \checkmark & \checkmark & \checkmark & \\\hline
\scriptsize{Redacción de artículos y otros documentos} &  & \checkmark & \checkmark  & \checkmark  & \checkmark & \checkmark  & \checkmark & \checkmark \\\hline
\scriptsize{Desarrollar conclusiones} &  &  &  & & & \checkmark & \checkmark & \checkmark \\\hline
\scriptsize{Sustentación} &  &  &  & & & & & \checkmark \\\hline
\end{tabular}
\end{center}
\end{tiny}
\end{frame}

\begin{frame}[standout]{Cronograma de actividades (2da mitad)}
\begin{tiny}
\begin{center}
\begin{tabular}{|c|c|c|c|c|}
\hline
\multicolumn{4}{c}
{\textbf{Cronograma de actividades}}\\
\hline
\hline

\large{\textbf{Actividades}} & \rotatebox{90}{SEMESTRE 5 } & \rotatebox{90}{SEMESTRE 6 } & \rotatebox{90}{SEMESTRE 7 } & \rotatebox{90}{SEMESTRE 8 }\\\hline
\scriptsize{Consultar fuentes bibliográficas}  & \checkmark & \checkmark & \checkmark & \\\hline
\scriptsize{Proponer conjeturas} & \checkmark & \checkmark & \checkmark & \checkmark \\\hline
\scriptsize{Validación y rechazo de conjeturas} & \checkmark & \checkmark & \checkmark & \checkmark \\\hline
\scriptsize{Redacción y presentación de artículos} & \checkmark & \checkmark & \checkmark  & \checkmark \\\hline
\scriptsize{Escritura de la tesis} & \checkmark & \checkmark & \checkmark  & \checkmark \\\hline
\scriptsize{Desarrollar conclusiones} & \checkmark & \checkmark & \checkmark & \checkmark \\\hline
\scriptsize{Sustentación} &  &  &  & \checkmark \\\hline
\end{tabular}
\end{center}
\end{tiny}
\end{frame}

\begin{frame}{Cosas por hacer}
\begin{itemize}
\item Obtener un ejemplo de marco eficiente que no sea KC.
\item Probar que los marcos KC son cerrados bajo coproductos.
\item Dar la noción libre de puntos de apilado y fuertemente apilado.
\item Establecer las condicines necesarias y suficientes para relacionar eficiencia con $\mathbf{(H)}$.
%\item Comprender la relación que existen entre las construcciones dadas por Escardo, Klinke e Igor con la de Sexton. 
%\item Explorar la relación de los $V$-puntos con la definición de eficiencia.
\item $\vdots$
\end{itemize}
\end{frame}

\End

\section*{\textsc{Referencias}}
\begin{frame}[allowframebreaks]
\frametitle{Bibliografía}
\begin{thebibliography}{20}\markboth{Bibliografía}{Bibliografía}

\bibitem{P.T.} P. T. Johnstone, \textit{Stone spaces}, Cambridge Studies in Advanced Mathematics, vol. 3, Cambridge University Press, Cambridge, 1982. MR 698074

\bibitem{J.M.} J. Monter; A. Zaldívar, \textit{El enfoque locálico de las reflexiones booleanas: un análisis en la categoría de marcos} [tesis de maestría], 2022. Universidad de Guadalajara.

\bibitem{P.S.} J. Paseka and B. Smarda, \textit{$ T_2 $-frames and almost compact frames.} Czechoslovak Mathematical Journal (1992), 42(3), 385-402.

\bibitem{J.P.} J. Picado and A. Pultr, \textit{Frames and locales: Topology without points}, Frontiers in Mathematics, Springer Basel, 2012.

\bibitem{J.P.2} J. Picado and A. Pultr, \textit{Separation in point-free topology}, Springer, 2021.

\bibitem{R.S.} RA Sexton, \textit{A point free and point-sensitive analysis of the patch assembly}, The University of Manchester (United Kingdom), 2003.

\bibitem{R.S.2} RA Sexton, \textit{Frame theoretic assembly as a unifying construct}, The University of Manchester (United Kingdom), 2000.

\bibitem{R.S.3} RA Sexton and H. Simmons, \textit{Point-sensitive and point-free patch constructions}, Journal of Pure and Applied Algebra \textbf{207} (2006), no. 2, 433-468.

\bibitem{H.S.} H. Simmons, \textit{An Introduction to Frame Theory}, lecture notes, University of Manchester. Disponible en línea en \url{https://web.archive.org/web/20190714073511/http://staff.cs.manchester.ac.uk/~hsimmons}.

\bibitem{H.S.R} H. Simmons, \textit{Regularity, fitness, and the block structure of frames.} Applied Categorical Structures 14 (2006): 1-34.

\bibitem{H.S.4} H. Simmons, \textit{The lattice theoretic part of topological separation properties}, Proceedings of the Edinburgh Mathematical Society, vol.~21, pp.~41--48, 1978.

\bibitem{H.S.V} H. Simmons, \textit{The Vietoris modifications of a frame}. Unpublished manuscript (2004), 79pp., available online at http://www. cs. man. ac. uk/hsimmons.

\bibitem{A.W.} A. Wilansky, \textit{Between T1 and T2}, MONTHLY (1967): 261-266.

\bibitem{A.Z.} A. Zaldívar, \textit{Introducción a la teoría de marcos} [notas curso], 2025. Universidad de Guadalajara.
\end{thebibliography}
\end{frame}


\end{document}