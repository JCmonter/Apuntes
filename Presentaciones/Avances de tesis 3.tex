\documentclass[compress,12pt]{beamer}

\usetheme{Arguelles}
\usepackage{graphicx}
\usepackage{caption}
%\usepackage[spanish,es-noshorthands]{babel}
\usepackage[spanish]{babel} 
\usepackage[pages=some]{background}
\usepackage{tikz-cd}
\usepackage{amsmath,amssymb,latexsym,amscd} 
\usepackage[all,cmtip]{xy}
\usepackage{fancyhdr}
\usepackage{mathalfa}
\usepackage{mathrsfs}
\usetikzlibrary{babel}
\usepackage{hyperref}
\usepackage{ragged2e}
\usepackage{wasysym}
\usepackage{multicol}
%\hypersetup{colorlinks=true,linkcolor=blue,citecolor=brown,linktocpage=true,pagebackref=true,hyperindex=true}
\pagenumbering{arabic}

\DeclareMathOperator{\op}{op}
\DeclareMathOperator{\pt}{pt}
\DeclareMathOperator{\spec}{spec}
\DeclareMathOperator{\Fit}{Fit}
\DeclareMathOperator{\Pth}{P}
\DeclareMathOperator{\Frm}{Frm}
\DeclareMathOperator{\Obj}{Obj}
\DeclareMathOperator{\Hom}{Hom}

\title{Patch modifications and separation axioms in point-free topology}
\event{Seminario de avance de tesis III}
\date{\today}
\author{Juan Carlos Monter Cortés\\ Luis Ángel Zaldívar Corichi} 
\institute{Universidad de Guadalajara}
\email{juan.monter2902@alumnos.udg.mx}

%\homepage{www.mywebsite.com}
%\github{username}

\begin{document}

\frame[plain]{\titlepage}

\begin{frame}{Lo que veremos hoy \smiley}
\tableofcontents %Imprime la tabla de contenido
\end{frame}

\section{Lo que si sabemos}
\subsection{Espacio de parches vs ensamble de parches}
\begin{frame}{Espacio de parches}
		
		\begin{block}{Definición:}
			Un espacio topológico $S$ es \emph{empaquetado} si todo conjunto compacto saturado es cerrado.
		\end{block}

		\onslide<2->{\[
   \mbox{pbase}=\{U\cap Q'\mid U\in \mathcal{O}S, \; Q\in \mathcal{Q}S\}.
   \]}

   
   \uncover<3->{\begin{block}{Definición:}
	Para un espacio topológico $S$, sea $^pS$ (\emph{espacio de parches}), el espacio con los mismos puntos que $S$ y la topología $\mathcal{O}^pS$ generada por la pbase.
\end{block}}
		
\end{frame}

\begin{frame}
\begin{block}{Observaciones:}
\begin{itemize}
\item $T_2\Rightarrow T_0+\mbox{empaqueta}\Rightarrow T_1$.
\item<2-> Si $S$ es $T_0$ $\Rightarrow$ $^pS$ es $T_1$ y si $S$ es $T_1$ $\Rightarrow$ $^pS=\,^{pp}S$.
\item<3-> Si $S$ es $T_0$ $\Rightarrow$ $^{pp}S=\,^{ppp}S$.
\item<4-> Si $S$ es $T_2$ $\Rightarrow$ $S=\,^pS$.
\item<5-> $S$ es empaquetado $\Leftrightarrow$ $S=\,^pS$. 
\end{itemize}
\end{block}

\onslide<6->{\alert{¿Cuál es el análogo de empaquetado en $\mathbf{Frm}$?}}
\end{frame}

\begin{frame}{Filtros, núcleos y el Teorema de H.-M.}
	\begin{definition}
		Sea $A\in \mathbf{Frm}$. Para $F\subseteq A$, decimos que $F$ es un \emph{filtro} si:
		\begin{enumerate}
			\item $1\in F$.
			\item $a\leq b$, $a\in  F$ $\Rightarrow b \in F$.
			\item $a, b \in F \Rightarrow  a \wedge b \in F$.
		\end{enumerate}
	\end{definition}
	\uncover<2->{Existen diferentes tipos de filtros:
	\begin{itemize}
	\begin{multicols}{2}
		\item Propio
		\item Primo
		\item Completamente primo
		\item (Scott) abierto
		\item Admisible ($\nabla(j)$)
		\end{multicols}
	\end{itemize} }
\end{frame}

\begin{frame}
Sean $A\in \mathbf{Frm}$ y $a, b\in A$. Decimos que $j\colon A\to A$ es un \emph{núcleo} si:
\begin{enumerate}
\item<2-> $a\leq b$ $\Rightarrow$ $j(a)\leq j(b)$
\item<3-> $a\leq j(a)$
\item<4-> $j^2(a)=j(a)$
\item<5-> $j(a\wedge b)=j(a)\wedge j(b)$
\end{enumerate}

\onslide<6->{\begin{block}{Observaciones:}
\begin{itemize}
\item<7-> $NA=$ conjunto de todos los núcleos en $A$ y $NA\in \mathbf{Frm}$.
\item<8-> $u_a(x)=a\vee x$, $v_a(x)=(a\succ x)$, $w_a=((x\succ a)\succ a)\in NA$.
\item<9-> $\eta_A\colon A\to NA$, $a\mapsto u_a$.
\item<10-> Si $f=f^*\in \mathbf{Frm}$ $\Rightarrow$ $k=f_*f^*\in NA$.
\item<11-> Para $U_A^*\colon A\to \mathcal{O}S$, $sp=(U_A)_*U_A^*\in NA$.
\end{itemize}
\end{block}}
\end{frame}

\begin{frame}{Filtros admisibles y núcleos ajustados}
\begin{block}{Definición:}
Sea $j\in NA$. El \emph{filtro de admisibilidad} de $j$ es el conjunto
\[
\nabla(j)=\{a\in A\mid j(a)=1\}.
\]
\end{block}

\onslide<2->{\begin{block}{Observaciones:}
\begin{itemize}
	\item<3-> $j, k\in NA$, $j\sim k\Leftrightarrow \nabla(j)=\nabla(k)$.
	\item<4-> $j\in NA$ es \emph{ajustado} si es el menor elemento de su bloque.
	\item<5-> $F\in A^\wedge \Rightarrow F=\nabla(j)$ para algún $j\in NA$.
	\item<6-> $F\in A^\wedge\Rightarrow [v_F, w_F]$.
\end{itemize}
\end{block}}
\end{frame}

\begin{frame}
	\only<-1>{\begin{block}{Teorema (Hoffman-Mislove):}
	Sean $A\in \mathbf{Frm}$ y $S=\pt(A)$, entonces existe una correspondencia biyectiva entre:
	\begin{enumerate}
		\item $\mathcal{Q}S=$ compactos saturados en $S$
		\item $A^\wedge=$ filtros abiertos en $A$
	\end{enumerate}
\end{block}}

 \only<2->{\begin{block}{Teorema (Hoffman-Mislove extendido):}
	Sean $A\in \mathbf{Frm}$ y $S=\pt(A)$, entonces existe una correspondencia biyectiva entre:
	\begin{enumerate}
		\item $\mathcal{Q}S=$ compactos saturados en $S$
		\item $A^\wedge=$ filtros abiertos en $A$
		\item $v_F=$ núcleos ajustados
	\end{enumerate}
\end{block}}
\onslide<3->{El Teorema de H.-M. nos proporciona $(F, Q, \nabla(Q))$
\[
F\in A^\wedge  \leftrightarrow Q\in \mathcal{Q}S\leftrightarrow \nabla(Q)\in \mathcal{O}S^\wedge
\]
\[
	x\in F  \Leftrightarrow Q\subseteq U_A(x)\Leftrightarrow U_A(x)\in \nabla(Q)
\]}
\end{frame}

\begin{frame}{El ensamble parches}
	Basados en el Teorema de H.-M. se introduce el \emph{ensamble de parches}.

	\[\text{p-base}(A) =\{u_{a}\wedge v_{F}\mid a\in A, F\in A^{\wedge}\}.\]
	
	\uncover<2->{Sea $PA=\langle\text{p-base}(A)\rangle$, es decir, tomamos supremos arbitrarios de elementos en $\text{p-base}(A)$
	
	 \[\xymatrix{ A\ar[r]^{i}\ar@/^/@<+1.5ex>[rr]^{\eta_{A}} & PA\ar[r]^{\iota} & NA }\]
	}
	
	\uncover<3->{
	\[
	¿\mbox{Cuándo } A\cong PA?
	\]} 

\end{frame}

\begin{frame}{Parche trivial}
	\begin{block}{Definición:}
		$A\in \mathbf{Frm}$ es \emph{parche trivial} si $i\colon A\to P(A)$ es un isomorfismo.
	\end{block}

\onslide<2->{¿Bajo qué circunstancias es el marco $A$ parche trivial?}
	\onslide<3->{\begin{block}{Observaciones:}
    \begin{itemize}
		\item<4-> $S=\,^pS \leftrightarrow A\cong PA$.
		\item<5-> Si $j\in NA$, $j=\bigvee\{v_x\wedge u_{j(x)}\mid x\in A\}$.
		\item<6-> Si $j\in PA$, $j=\bigvee\{v_F\wedge u_d\mid F\in A^\wedge \mbox{ y }d\in A\}$.
		\item<7-> $A$ es parche trivial $\Leftrightarrow$ $v_F=u_d$.
		\item<8-> $\forall\, a\in A$, $u_a\leq v_a$ y $v_a\leq j \Leftrightarrow j(a)=1$.
    \end{itemize}
	\end{block}}
\end{frame}

\begin{frame}{Regularidad implica parche trival}
	\begin{block}{Teorema:}
		Supongamos que $A$ es un marco regular y sea $j\in NA$ tal que $\bigtriangledown (j)$ es abierto. Entonces $j=u_d$, donde $d=j(0)$
	\end{block}

	\onslide<2->{\begin{proof}
		\begin{enumerate}
			\item<3-> $A$ es regular $\Rightarrow$ $A$ es ajustado.
			\item<4-> $x=\bigvee\{y\in A\mid (\exists \,z)[z\wedge y=0 \mbox{ y }z\vee x=1]\}$ es dirigido.
			\item<5-> $\nabla(u_d)=\nabla(j)\Rightarrow j\sim u_d$.
			\item<6-> $A$ es ajustado $\Rightarrow$ $u_d$ es el único en su bloque.
			\item<7-> $j=u_d$.
		\end{enumerate}
	\end{proof}}
\end{frame}    

\begin{frame}{Marcos arreglados}
	\begin{block}{Definición:}
		Sea $A\in \mathbf{Frm}$ y $\alpha$ un ordinal, un filtro abierto $F$ en $A$ es $\alpha-$\emph{arreglado} si \[x\in F\Rightarrow u_{d(\alpha)}(x)=d(\alpha)\vee x=1,\] donde $d(\alpha)=f^{\alpha}(0)$ y $f=\dot\bigvee\{v_a\mid a\in F\}$.
	\end{block}

	\begin{itemize}
	\item<2-> Un marco $A$ es $\alpha-$arreglado si $\forall\, F\in A^\wedge \Rightarrow F$ es $\alpha-$arreglado.
    \item<3-> Parche trivial $\Leftrightarrow$ Arreglado	
    \end{itemize}
\end{frame}

\begin{frame}
	\begin{block}{Teorema:}
		Sea $S$ un espacio $T_{0}$, éste tiene \alert<2->{marco de abiertos 1-arreglado si y solo si $S$ es $T_2$}.
	\end{block}
\onslide<3->{Arreglado es una noción que caracteriza marcos por medio de propiedades espaciales.}

\onslide<4->{\begin{block}{Observaciones:}
\begin{itemize}
	\item<5-> Arreglado $\Rightarrow T_1$.
	\item<6-> Arreglado $\Rightarrow$ empaquetado y Empaquetado $\nRightarrow$ Arreglado.
	\item<7-> Arreglado $\Leftrightarrow$ Empaquetado + \alert<8->{Apilado}.
\end{itemize}
\end{block}}
\end{frame}

\begin{frame}{Propiedad conservativa, de 1° orden y de 2° orden}

	\begin{block}{Definición:}
		\begin{enumerate}
			\item Para un espacio $S$ decimos que una propiedad $P$ es \emph{conservativa} si y solo si $\mathcal{O}S$ tiene la propiedad $P_S$.
	
			\item<2-> Decimos que una propiedad en marcos $P$ es \emph{suficientemente Hausdorff} si y solo si $P$ implica la propiedad Hausdorff espacial.
			
			\item<3-> Decimos que una propiedad en marcos $P$ es de \emph{1° orden} si y solo si $P$ es enunciada como una fórmula para elementos del marco.
	
			\item<4-> Decimos que una propiedad en marcos $P$ es de \emph{2° orden} si y solo si $P$ es enunciada como una caracterización de sublocales.
		\end{enumerate}
	\end{block}
	
	\end{frame}

\begin{frame}{Marcos Hausdorff}
	\begin{description}
        \item[$(\mathbf{dH})$] $\quad a\vee b=1$ y $a, b\neq 1$, $\exists$ $u, v$ tales que $u\nleq a$, $v\nleq b$ y $u\wedge v=0$. 
        \item<2->[$(\mathbf{H})$] $\quad 1\neq a\nleq b\in L$, $\exists$ $u, v\in L$ tales que $u\nleq a$, $v\nleq b$ y $u\wedge v=0$. 
        \item<3->[$(\mathbf{Hp})$] $\quad$Cada elemento semiprimo en $L$ es máximo.
        \item<4->{[$(\mathbf{fH})$] $\quad$El sublocal diagonal $\Delta[L]$ es cerrado en $L\oplus L$.
        \[
        \Leftrightarrow \Delta[L]=\uparrow d_L
        \]
        donde $d_L$ es el menor elemento de $\Delta[L]$, es decir,
\[
d_L=\Delta(0)=\{(x, y)\mid x\wedge y\leq 0\}=\downarrow\{(x, x^*)\mid x\in L\}.
\]}
    \end{description}
\end{frame}

\section{Los problemas que tenemos}
\subsection{La conjetura}
\begin{frame}{El razonamiento}
	\begin{block}{Definición:}
		Decimos que un marco $A$ es espacial si $A=\mathcal{O}S$, para $S$ un espacio topológico.
	\end{block}
	
	\uncover<2->{\begin{block}{Teorema:}
		Si $A$ es un marco espacial entonces 
		\[
		A \text{ es } 1\text{-arreglado si y solo si } S \text{ es } T_2
		\] 
	\end{block}
	 }


\end{frame}

\begin{frame}{La conjetura}

		\begin{itemize}
			\item Si $S$ es $T_2$ $\Rightarrow$ $\mathcal{O}S$ es parche trivial
			\item<2-> Un marco $A$ es arreglado $\Leftrightarrow$ A parche trivial.
			\item<3-> $\mathbf{(H)}$ es conservativa
		\end{itemize}
	
		\uncover<4->{\[
	\mathcal{O}S \mbox{ es }\mathbf{(H)}\Leftrightarrow S\mbox{ es }T_2\Rightarrow \mathcal{O}S \mbox{ parche trivial }\Leftrightarrow \mathcal{O}S \mbox{ arreglado}
	\]}
	
	\uncover<5->{\begin{block}{Conjetura}
		Todo marco Hausdorff es 1-arreglado.
	\end{block}}
\end{frame}

\begin{frame}[plain, fragile]
\begin{block}{Teorema:}
        Para $A$ un marco espacial, $\mathcal{O}S$ es un marco Hausdorff si y solo si $A$ es $1-$arreglado.
    \end{block}

\begin{block}{Teorema:}
    Todo marco fuertemente Hausdorff es arreglado.
\end{block}

\[\begin{tikzcd}
	{\mathbf{(reg)}} && {\mathbf{(fH)}} && {\mathbf{(H)}} \\
	\\
	&& {\mathbf{Arreglado}}
	\arrow[Rightarrow, from=1-1, to=1-3]
	\arrow[Rightarrow, from=1-1, to=3-3]
	\arrow[Rightarrow, from=1-3, to=1-5]
	\arrow[Rightarrow, from=1-3, to=3-3]
	\arrow[Rightarrow, dashed, from=1-5, to=3-3]
\end{tikzcd}\]
\end{frame}

\begin{frame}{Otra forma de ver arreglado}
Si $A\in \mathbf{Frm}$ y $j\in NA\Rightarrow$ $A_j\in \mathbf{Frm}$.
\onslide<2->{\begin{block}{Observaciones:}
\begin{itemize}
	\item<3-> $A_j$ es un cociente de $A$.
	\item<4-> $A_j$ es compacto $\Leftrightarrow$ $\nabla(j)\in A^\wedge$.
	\item<5-> $F\in A^\wedge\Rightarrow F=\nabla(j)$.
	\item<6-> $A$ es arreglado si todo cociente compacto es cerrado. 
\end{itemize}
\end{block}}

\onslide<7->{¿Existen ejemplos de marcos (locales) Hausdorff y compactos que sean cerrados?}
\end{frame}

\begin{frame}{¿Qué significa apilado en marcos?}
	\[
		\begin{split}
			\mathcal{O}S\; 0-\mbox{arreglado }& \Leftrightarrow S=\emptyset\\
			\mathcal{O}S\; 1-\mbox{arreglado }& \Leftrightarrow S \mbox{ es } T_2\\
			\mathcal{O}S\, \mbox{ arreglado }& \Leftrightarrow S \mbox{ empaquetado }+ \mbox{ apilado}
		\end{split}
	\]

	\begin{block}{Definición:}
		Sea $S\in \mathbf{Top}$ y $Q\in \mathcal{Q}S$. Decimos que $X\in \mathcal{C}S$ es \emph{Q-irreducible} (denotado por ``$Q\ltimes X$''), si
		\[
		Q\subseteq U\Rightarrow X\subseteq \overline{(X\cap U)}
		\]
		Equivalentemente $Q\subseteq U\Rightarrow X=\overline{(X\cap U)}$, para cada $U\in \mathcal{O}S$.
	\end{block}
\end{frame}

\begin{frame}
\begin{block}{Definición:}
	\begin{itemize}
		\item $S\in \mathbf{Top}$ es \emph{apilado} si
		\[
		Q\ltimes X\Rightarrow X\subseteq \overline{Q}
		\] 
		se cumple para cada $Q\in \mathcal{Q}S$ y $X\in \mathcal{C}S$.
		\item<2-> $S\in \mathbf{Top}$ es \emph{fuertemente apilado} si 
		\[
		Q\ltimes X\Rightarrow X\subseteq \overline{X\cap Q}
		\]
		se cumple para cada $Q\in \mathcal{Q}S$ y $X\in \mathcal{C}S$.
	\end{itemize}
\end{block}

\onslide<3->{¿Qué relación tiene esta noción espacial con las nociones en marcos?}
\end{frame}

\section{Otras alternativas}
\begin{frame}[fragile]{EL Q-cuadrado}
	\begin{block}{Ingredientes:}
		\begin{multicols}{3}
		
		 \begin{itemize}
			\item $U_A\colon \mathcal{O}S$
			\item $F\in A^\wedge \rightarrow v_F$
			\item $A_F=A_{v_F}$
			\item $\nabla\in \mathcal{O}S \rightarrow v_\nabla$
			\item $\mathcal{O}S_\nabla=\mathcal{O}S_{v_\nabla}$
			\item $G=\nabla U_A$
			\item $Q=\pt A_F$
			\item $g=G_{\mid A_F}$
			\item $?\colon \mathcal{O}S_\nabla\to A_F$
		 \end{itemize}
		
		\end{multicols}
	\end{block}
\[\begin{tikzcd}
	A && {A_F} && {} \\
	&&& {\mathcal{O}Q} \\
	{\mathcal{O}S} && {\mathcal{O}S_\nabla}
	\arrow["F", shift left=2, color={rgb,255:red,67;green,67;blue,239}, from=1-1, to=1-3]
	\arrow["{U_A}"', shift right=2, color={rgb,255:red,67;green,67;blue,239}, from=1-1, to=3-1]
	\arrow["G", from=1-1, to=3-3]
	\arrow["{F_*}", shift left=2, color={rgb,255:red,245;green,61;blue,61}, from=1-3, to=1-1]
	\arrow[from=1-3, to=2-4]
	\arrow["g", shift left=2, color={rgb,255:red,69;green,237;blue,72}, dashed, from=1-3, to=3-3]
	\arrow["{(U_A)_*}"', shift right=2, color={rgb,255:red,245;green,61;blue,61}, from=3-1, to=1-1]
	\arrow["\nabla"', shift right=2, color={rgb,255:red,67;green,67;blue,239}, from=3-1, to=3-3]
	\arrow["{?}", shift left=2, from=3-3, to=1-3]
	\arrow[from=3-3, to=2-4]
	\arrow["{\nabla_*}"', shift right=2, color={rgb,255:red,245;green,61;blue,61}, from=3-3, to=3-1]
\end{tikzcd}\]
\end{frame}

\begin{frame}
\begin{block}{Observaciones:}
\begin{itemize}
	\item<2-> El $Q-$cuadrado está definido para cada $F\in A^\wedge$.
	\item<3-> Si $A$ es espacial $\Rightarrow$ $A\to A_F \to \mathcal{O}Q$.
	\item<4-> Si $?=g_*$ y $g_*g=sp$ $\Rightarrow$ $A\to A_F \to \mathcal{O}Q$.
	\item<5-> En general, ¿quién es ``?''?
	\item<6-> Si conocemos todos los $F\in A^\wedge$, conocemos todos los $Q-$cuadrados.
	\item<7-> Existen ejemplos donde se conocen todos los $F\in A^\wedge$

\end{itemize}
\end{block}
\end{frame}

\begin{frame}[fragile]{Las propiedades del funtor $N$}
	\[\begin{tikzcd}
		A & NA \\
		\\
		{\mathcal{O}S} & {N\mathcal{O}S}
		\arrow[""{name=0, anchor=center, inner sep=0}, "U"', from=1-1, to=3-1]
		\arrow[""{name=1, anchor=center, inner sep=0}, "{N(U)}", from=1-2, to=3-2]
		\arrow["{N(\_)}", shorten <=7pt, shorten >=7pt, maps to, from=0, to=1]
	\end{tikzcd}\]

	\begin{block}{Observaciones:}
\onslide<2->{\begin{itemize}
    \item<3-> $N(U) \dashv N(U)_*$
    \item<4-> $N(U)(j)\leq k\Leftrightarrow j\leq N(U)_*(k)$
    \item<5-> $N(U)(j)\leq k\Leftrightarrow Uj\leq kU$
    \item<6-> $N(U)_*(j)=U_*jU^*$ y $UN(U)_*(j)=jU$
\end{itemize}
\end{block}}
	\end{frame}

	\begin{frame}[fragile]{Los intervalos $[v_{\_}, w_{\_}]$}
		Para $F\in A^\wedge$, $Q\in \mathcal{Q}S$ y $j\in N\mathcal{O}S$ 
		\[\begin{tikzcd}
			x\in A & {\mathcal{O}S} & {\mathcal{O}S} & A
			\arrow["{U^*}", from=1-1, to=1-2]
			\arrow["j", from=1-2, to=1-3]
			\arrow["{U_*}", from=1-3, to=1-4]
		\end{tikzcd}\]

		\begin{itemize}
		\item<2-> $U_*(j(U(x)))=\bigwedge(S\setminus j(U(x)))$
		\item<3-> $U_*(j(U^*(x)))\subseteq \pt A$. Así,
		\item<4-> $x\in F \Leftrightarrow  x\in \nabla(U_*jU^*)$
        \item<5-> $F=\nabla(U_*jU^*)$
	\end{itemize}

	\onslide<6->{Por lo tanto 
	\[
	j\in [v_Q, w_Q]\Rightarrow F\in [v_F, w_F].
	\]}
	\end{frame}

	\begin{frame}[fragile]
		De esta manera
\[
\mho\colon [V_Q, W_Q]\to [V_F, W_F]
\]

\[\begin{tikzcd}
	NA && {N\mathcal{O}S} \\
	{[v_F, w_F]} && {[v_Q, w_Q]}
	\arrow["{N(U)}", shift left=2, from=1-1, to=1-3]
	\arrow["{N(U)_*}", shift left=2, from=1-3, to=1-1]
	\arrow[hook, from=2-1, to=1-1]
	\arrow["\mho^*", shift left=2, dashed, from=2-1, to=2-3]
	\arrow[hook, from=2-3, to=1-3]
	\arrow["\mho", shift left=2, from=2-3, to=2-1]
\end{tikzcd}\]

\begin{block}{Observaciones:}
\begin{itemize}
\item<2-> ¿Qué información aporta el morfismo $\mho*$?
\item<3-> ¿Cuál es el comportamiento de los diagramas anteriores cuando $A$ cumple alguna propiedad Hausdorff?
\end{itemize}
\end{block}
	\end{frame}
\End

\section*{\textsc{Referencias}}
\begin{frame}[allowframebreaks]
\frametitle{References}
\begin{thebibliography}{20}\markboth{Bibliografía}{Bibliografía}

\bibitem{P.T.} P. T. Johnstone, \textit{Stone spaces}, Cambridge Studies in Advanced Mathematics, vol. 3, Cambridge University Press, Cambridge, 1982. MR 698074


\bibitem{J.M.} J. Monter; A. Zaldívar, \textit{El enfoque locálico de las reflexiones booleanas: un análisis en la categoría de marcos} [tesis de maestría], 2022. Universidad de Guadalajara.

\bibitem{J.P.} J. Picado and A. Pultr, \textit{Frames and locales: Topology without points}, Frontiers in Mathematics, Springer Basel, 2012.

\bibitem{J.P.2} J. Picado and A. Pultr, \textit{Separation in point-free topology}, Springer, 2021.

\bibitem{R.S.} Rosemary A Sexton, \textit{A point free and point-sensitive analysis of the patch assembly}, The University of Manchester (United Kingdom), 2003.

\bibitem{H.S.3} Harold Simmons, \textit{The assembly of a frame}, University of Manchester (2006).

\bibitem{R.S.3} RA Sexton and H. Simmons, \textit{Point-sensitive and point-free patch constructions}, Journal of Pure and Applied Algebra \textbf{207} (2006), no. 2, 433-468.

\bibitem{A.Z.} A. Zaldívar, \textit{Introducción a la teoría de marcos} [notas curso], 2024. Universidad de Guadalajara.

\end{thebibliography}
\end{frame}
\end{document}
