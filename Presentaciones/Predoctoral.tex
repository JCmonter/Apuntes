\documentclass[compress,12pt]{beamer}

\usetheme{Arguelles}
\usepackage{graphicx}
\usepackage{caption}
%\usepackage[spanish,es-noshorthands]{babel}
\usepackage[spanish]{babel} 
\usepackage[pages=some]{background}
\usepackage{tikz-cd}
\usepackage{amsmath,amssymb,latexsym,amscd} 
\usepackage[all,cmtip]{xy}
\usepackage{fancyhdr}
\usepackage{mathalfa}
\usepackage{mathrsfs}
\usetikzlibrary{babel}
\usepackage{hyperref}
\usepackage{ragged2e}
\usepackage{wasysym}
\usepackage{multicol}
%\hypersetup{colorlinks=true,linkcolor=blue,citecolor=brown,linktocpage=true,pagebackref=true,hyperindex=true}
\pagenumbering{arabic}

\DeclareMathOperator{\op}{op}
\DeclareMathOperator{\pt}{pt}
\DeclareMathOperator{\spec}{spec}
\DeclareMathOperator{\Fit}{Fit}
\DeclareMathOperator{\Pth}{P}
\DeclareMathOperator{\Frm}{Frm}
\DeclareMathOperator{\Top}{Top}
\DeclareMathOperator{\Obj}{Obj}
\DeclareMathOperator{\Hom}{Hom}

%%Se define el "environment" teorema
\newtheorem{thm}{Teorema}
\newtheorem{dfn}{Definición}
\newtheorem*{dfn*}{Definición}
\newtheorem{lem}{Lema}
\newtheorem{cor}{Corolario}
\newtheorem{prop}{Proposición}
\newtheorem{obs}{Observación}
\newtheorem{ej}{Ejemplo}

\title{Modificaciones de parches y axiomas de separación en topología sin puntos}
%\event{Seminario de Álgebra, CUCEI}
\date{\today}
\author{Juan Carlos Monter Cortés \\ Director: Dr. Luis Ángel Zaldívar Corichi}
\institute{Universidad de Guadalajara}
%\email{juan.monter2902@alumnos.udg.mx}

%\homepage{www.mywebsite.com}
%\github{username}

\begin{document}

\frame[plain]{\titlepage}

%\begin{frame}{Contenido}
%\tableofcontents %Imprime la tabla de contenido
%\end{frame}

%\section{Lo que sabemos}
\begin{frame}{El axioma $T_2$}
\begin{description}
    \item[$T_2$:] Si $x, y\in S$ con $x\neq y$, $\exists\, U, V\in \mathcal{O}S$ tal que $x\in U$, $y\in V$ y $U\cap V=\emptyset$.
\end{description}

Si $S$ es $T_2$,

\begin{itemize}
    \item Todo conjunto compacto (saturado) es cerrado.
    \item El subespacio diagonal es cerrado.
    \item $T_2+$ compacto $\Rightarrow$ Regular.
    \item $\vdots$
\end{itemize}    
\end{frame}

\begin{frame}[fragile]{Construcciones de parches}
La construcción en $\Top$:
\begin{itemize}
    \item $S\in \Top$ es empaquetado si y solo si $Q\in \mathcal{Q}S$ entonces $Q\in \mathcal{C}S$. 
    \item $^pS\in \Top$ es el espacio de parches con la topología generada por 
    \[
    \mbox{pbase} = \{U\cap Q'\mid U\in \mathcal{O}S,\ Q\in \mathcal{Q}S\}.
    \]
    \item $S$ es empaquetado si y solo si $^pS = S$.
    \item $T_2 \Rightarrow \mbox{Empaquetado} \Rightarrow T_1$.
    \item Si $S$ es $T_1$, entonces $^pS = \,^{pp}S$.
    \item Si $S$ es $T_0$, entonces $^{pp}S = \,^{ppp}S$
\end{itemize}
\[
\begin{tikzcd}
	{\mathcal{O}S} & {\mathcal{O}^pS} & {\mathcal{O}^fS}
	\arrow[hook, from=1-1, to=1-2]
	\arrow[hook, from=1-2, to=1-3]
\end{tikzcd}
\]
\end{frame}

\begin{frame}{Para la ``traducción''}
Lo que necesitamos:
\begin{enumerate}
\item Núcleos
\begin{itemize}
	\item Núcleos abiertos
	\item Núcleos cerrados
\end{itemize}
\item Filtros 
\begin{itemize}
	\item Filtros abiertos
	\item Filtros admisibles
\end{itemize}
\item El Teorema de Hoffman-Mislove
\end{enumerate}
\end{frame}

\begin{frame}[fragile]{Construcciones de parches}
	La construcción en $\Frm$:
	\begin{itemize}
		\item Para $A\in \Frm$, $PA$ es el marco de parches y es generado por 
		\[
		\mbox{Pbase}=\{u_a\wedge v_F\mid a\in A, F\in A^\wedge\}.
		\]
		\item $A$ es parche trivial si y solo si $A\simeq PA$.
		\item ¿Cuándo ocurre que $A\simeq PA$?
		\item ¿Cómo se comparta $PA$?
	\end{itemize}
	\[\begin{tikzcd}
	{A} & {PA} & {NA}
	\arrow[from=1-1, to=1-2]
	\arrow[hook, from=1-2, to=1-3]
\end{tikzcd}\]
\[
\mbox{¿Qué tanto se parece empaquetado y parche trivial?}
\]
\end{frame}

\begin{frame}[fragile]{Ejemplo}
Sea $S=\mathbb{R}$ y consideremos las topologías generadas por:
\[
\mathcal{O}_lS=\{(-\infty, a)\},\quad  \mathcal{O}_mS=\{(a,b)\}, \quad \mathcal{O}_nS=\{[a,b)\},
\]
donde $a,b\in S$. Entonces
\[
\mathcal{O}_lS \hookrightarrow \mathcal{O}_mS \hookrightarrow \mathcal{O}_nS
\]

Se puede verificar que 
\[
\mathcal{O}_l^pS=\mathcal{O}_mS\simeq P\mathcal{O}_lS\quad\mbox{ y }\quad\mathcal{O}_l^fS=\mathcal{O}_nS\simeq N\mathcal{O}_lS,
\]
es decir, 
\[
\mathcal{O}_lS=A\rightarrow PA\hookrightarrow NA
\]
\end{frame}

\begin{frame}[fragile]{El diagrama de parches}
\begin{equation}\label{Diagrama de parches}
\begin{tikzcd}
	A & PA & NA \\
	{\mathcal{O}S} & {P\mathcal{O}S} & {N\mathcal{O}S} \\
	& {\mathcal{O}^pS} & {\mathcal{O}^fS}
	\arrow[from=1-1, to=1-2]
	\arrow["{U_A}"', from=1-1, to=2-1]
	\arrow[hook, from=1-2, to=1-3]
	\arrow["{PU_A}", from=1-2, to=2-2]
	\arrow["{NU_A}", from=1-3, to=2-3]
	\arrow[from=2-1, to=2-2]
	\arrow[from=2-1, to=3-2]
	\arrow[hook, from=2-2, to=2-3]
	\arrow["{\pi_S}", from=2-2, to=3-2]
	\arrow["{\sigma_S}", from=2-3, to=3-3]
	\arrow[hook, from=3-2, to=3-3]
\end{tikzcd}
\end{equation}
\begin{itemize}
	\item La construcción de parches es funtorial si para
\[
f\colon A\to B
\]
convierte filtros abiertos.
\item $U_A$ convierte filtros abiertos.
\end{itemize}
\end{frame}

\begin{frame}{Marcos arreglados}
Sea $A\in \Frm$ y $\alpha\in \mathbf{Ord}$. 
    \begin{itemize}
        \item $F\in A^\wedge$ es $\alpha$-arreglado si 
    \[
    x\in F\Rightarrow u_d(x)=d\vee x=1,
    \]
    donde $d=d(\alpha)=f^\alpha(0)$ y $f=\dot{\bigvee}\{v_a\mid a\in F\}$
    \item $A$ es $\alpha$-arreglado si todo $F\in A^\wedge$ es $\alpha$-arreglado.
    \item $A$ es arreglado si $A$ es $\alpha$-arreglado para algún $\alpha$.
    \end{itemize}
\end{frame}

\begin{frame}{Porpiedades de los marcos arreglados}
\begin{itemize}
	\item Arreglado $\Leftrightarrow$ Parche trivial.
	\item Regular $\Rightarrow$ Arreglado.
	\item Si $A$ es arreglado $\Rightarrow S=\pt A$ es $T_1$.
	\item Si $A$ es $1$-arreglado $\Rightarrow S$ es $T_2$.
	\item $\mathcal{O}S$ es $1$-arreglado $\Leftrightarrow S$ es $T_2$.
	\item $\mathcal{O}S$ es arreglado $\Leftrightarrow S$ es empaquetado y apilado.  
\end{itemize}
\end{frame}

\begin{frame}[plain, standout]{Objetivo}
    \Huge{¿Qué relación existe entre la propiedad arreglado y los distintas propiedades de separación que existe en $\Frm$?}

\end{frame}

\begin{frame}[fragile, plain]{Axiomas de separación en $\Frm$}
    \[\begin{tikzcd}
	&& {\mathbf{(reg)}} \\
	{\mathbf{(aju)}} && {\mathbf{(fH)}} && {\mathbf{(H)}+\mathbf{(saju)}} \\
	{\mathbf{(saju)}} && {T_U} && {\mathbf{(H)}} \\
	&& {T_1}
	\arrow[Rightarrow, from=1-3, to=2-1]
	\arrow[Rightarrow, from=1-3, to=2-3]
	\arrow[Rightarrow, from=1-3, to=2-5]
	\arrow[dotted, no head, from=2-1, to=2-3]
	\arrow[Rightarrow, from=2-1, to=3-1]
	\arrow[Rightarrow, from=2-1, to=3-3]
	\arrow[dotted, no head, from=2-1, to=3-5]
	\arrow[dotted, no head, from=2-3, to=2-5]
	\arrow[dotted, no head, from=2-3, to=3-1]
	\arrow[Rightarrow, from=2-3, to=3-3]
	\arrow[Rightarrow, from=2-3, to=3-5]
	\arrow[Rightarrow, from=2-5, to=3-1]
	\arrow[dotted, no head, from=2-5, to=3-3]
	\arrow[Rightarrow, from=2-5, to=3-5]
	\arrow[dotted, no head, from=3-1, to=3-3]
	\arrow[dotted, no head, from=3-1, to=4-3]
	\arrow[dotted, no head, from=3-3, to=3-5]
	\arrow[Rightarrow, from=3-3, to=4-3]
	\arrow[Rightarrow, from=3-5, to=4-3]
\end{tikzcd}\]
\end{frame}

\begin{frame}{Axiomas tipo Hausdorff}
    \begin{description}
        \item[$\mathbf{(dH):}$]$\quad$ Si $a\vee b=1$, con $a,b\neq 1$, $\exists\, u,v$ tales que $u\nleq a$, $v\nleq b$ y $u\wedge v=0$.
        \item[$\mathbf{(H):}$]$\quad$ Si $1\neq a\nleq b$ $\exists\, u, v$ tales que $u\nleq a$, $v\nleq b$ y $u\wedge v=0$.
        \item[$\mathbf{(Hp):}$]$\quad$ Todo elemento semiprimo es máximo.
        \item[$\mathbf{(fH):}$]$\quad$ El sublocal diagonal es cerrado.
    \end{description}
\end{frame}

\begin{frame}{En resumen...}
\begin{center}
\begin{tabular}{| c | c | c | c | c | c |}
\hline
 Propiedad/Comportamiento & \textbf{C.} & 1° & 2° & \textbf{S. H.} & \textbf{C. S. E.}\\ \hline
$\mathbf{(dH)}$ & x & $\checkmark$ & x & x & x \\ \hline
$\mathbf{(H)}$ & $\checkmark$ & $\checkmark$ & x & $\checkmark$ & x \\ \hline
$\mathbf{(Hp)}$ & $\checkmark$ & $\checkmark$ & x & $\checkmark$ & ? \\ \hline
$\mathbf{(fH)}$ & x & x &  $\checkmark$ & $\checkmark$ & $\checkmark$ \\ \hline
\end{tabular}
\end{center}
\begin{tiny}
$$\mbox{ }$$
\end{tiny}

\textbf{S. H.}= Suficientemente Huasdorff ($P\Rightarrow T_2$)\\
\textbf{C.}= Propiedad conservativa ($P\Leftrightarrow T_2$)\\
\textbf{C. S. E.}= Comportamiento similar al espacial
\[
(\mathbf{(H)}+\mbox{Compacto}\nRightarrow \mathbf{(reg)})
\]
\end{frame}

\begin{frame}[plain, standout]{Otros objetivos}
    \begin{itemize}
        \item Conocer la relación entre parche trivial y empaquetado.
        \item Ver el comportamiento de arreglado con respecto a los axiomas tipo Hausdorff.
        \item Desarrollar teoría que permita comprender la interacción entre las propiedades.
        \item Ver ejemplos.
    \end{itemize}
\end{frame}

\section{Lo que hemos hecho}
\begin{frame}{Analizando $PA$}
¿El marco de parches se comporta como el espacio de parches?
\begin{itemize}
	\item $PA$, en general, no es  $T_1$.
	\item Al no ser $T_1$ no puede cumplir algún otro axioma de separación más fuerte que $T_1$.
	\item ¿$PA$ cumple $\mathbf{(saju)}$?
	\item ¿Se cumple que $PA\simeq P^2A$?
	\item Se cumple que $P^2A\simeq P^3A$.
\end{itemize}
¿Qué pasa si el marco $A$ cumple $\mathbf{(H)}$?
\end{frame}

\begin{frame}[fragile]
\begin{itemize}
\item Si $A$ es $\mathbf{(H)}\Rightarrow S=\pt A$ es $T_2$
\item Si $S$ es $T_2\Rightarrow \mathcal{O}S$ es $1$-arreglado.
\item Si $\mathcal{O}S$ es $1$-arreglado $\Rightarrow \mathcal{O}S\simeq P\mathcal{O}S$.
\item Si $S$ es $T_2\Rightarrow ^pS=S$.
\item Si $^pS=S\Rightarrow \mathcal{O}S=\mathcal{O}^pS$.
\end{itemize}
\[\begin{tikzcd}
	A & PA & NA \\
	& {\mathcal{O}S} & {N\mathcal{O}S} \\
	&& {\mathcal{O}^fS}
	\arrow[from=1-1, to=1-2]
	\arrow["{U_A}"', from=1-1, to=2-2]
	\arrow[from=1-2, to=1-3]
	\arrow["{PU_A}", from=1-2, to=2-2]
	\arrow["{NU_A}", from=1-3, to=2-3]
	\arrow[from=2-2, to=2-3]
	\arrow[from=2-2, to=3-3]
	\arrow["{\sigma_S}", from=2-3, to=3-3]
\end{tikzcd}\]
\end{frame}

\begin{frame}{Marcos arreglados vs propiedades en $\Frm$}
\begin{cor}
Si $A\in \Frm$ es espacial, entonces $\mathcal{O}S$ cumple $\mathbf{(H)}\Leftrightarrow A$ es $1$-arreglado.
\end{cor}

\begin{cor}
	Todo marco ajustado es arreglado.
\end{cor}

\begin{prop}
Todo marco fuertemente Hausdorff es arreglado.
\end{prop}

\begin{prop}
Si $A$ es arreglado, $A_j$ es arreglado para $j\in NA$.
\end{prop}

\end{frame}

\begin{frame}[fragile]
\[\begin{tikzcd}
	{\mathbf{(aju)}} \\
	{\mathbf{(reg)}} && {\mbox{Arreglado}} \\
	{\mathbf{(fH)}} \\
	{\mathbf{(H)}}
	\arrow[Rightarrow, from=1-1, to=2-3]
	\arrow[Rightarrow, from=2-1, to=1-1]
	\arrow[Rightarrow, from=2-1, to=2-3]
	\arrow[Rightarrow, from=2-1, to=3-1]
	\arrow[Rightarrow, from=3-1, to=2-3]
	\arrow[Rightarrow, from=3-1, to=4-1]
	\arrow["{?}"', Rightarrow, from=4-1, to=2-3]
\end{tikzcd}\]
\end{frame}

\begin{frame}{Intervalos de admisibilidad}
Si $F\in A^\wedge$, entonces 
\[
u_d\leq v_F \leq  w_F
\] 
para $d=v_F(0)$, $v_F=f^\infty$.

\begin{block}{Información con los intervalos}
\begin{itemize}
\item $[v_F, w_F]\subseteq NA$ es el intervalo de admisibilidad.
\item Si $j\in [v_F,w_F]$, $A_j$ es compacto.
\item Arreglado $\Leftrightarrow v_F\leq u_d$
\item $\mathbf{(fH)}\Leftrightarrow\forall\, j\in [v_F,w_F]$, $j=u_\bullet$ y $\bullet\in A$.  
\item $\mathbf{(aju)}\Leftrightarrow [v_F,w_F]=\{*\}$ y $*=u_\bullet$ para $\bullet\in A$. 
\end{itemize}
\end{block}
\end{frame}

\begin{frame}
Por el Teorema de Hoffman-Mislove 
\[
Q\in \mathcal{Q}S\leftrightarrow F\in A^\wedge\quad \mbox{ y } \quad Q\in \mathcal{Q}S\leftrightarrow \nabla\in \mathcal{O}S^\wedge.
\]
Además, $[v_Q, w_Q]\subseteq N\mathcal{O}S$ es un intervalo de admisibilidad.

\begin{prop}
Con $F$ y $Q$ como antes, si $j\in [v_Q, w_Q]$, entonces 
\[
\nabla(U_*jU^*)=F
\]
donde $U^*$ es la reflexión espacial y $U_*$ es su adjunto derecho.
\end{prop}
Equivalentemente
\[
\mho\colon [v_Q, w_Q]\to [v_F, w_F].
\]
\end{frame}

\begin{frame}[fragile]{El $Q$-cuadrado}
En \cite{H.S.V} construyen el diagrama
\[\begin{tikzcd}
	A & A \\
	{\mathcal{O}S} & {\mathcal{O}S}
	\arrow["{f^\infty}", from=1-1, to=1-2]
	\arrow["{U_A}"', from=1-1, to=2-1]
	\arrow["{U_A}", from=1-2, to=2-2]
	\arrow["{F^\infty}"', from=2-1, to=2-2]
\end{tikzcd}\]
y prueban que $U_A\circ f^\infty\leq F^\infty \circ U_A$.\\
	
Si $j\in NA$, $\hat{f}^\infty$ es el núcleo asociado al filtro $j_*F\in A^\wedge$.
\[\begin{tikzcd}
	A & A \\
	{A_j} & {A_j}
	\arrow["{\hat{f}^\infty}", from=1-1, to=1-2]
	\arrow["{j}"', from=1-1, to=2-1]
	\arrow["{j}", from=1-2, to=2-2]
	\arrow["{f^\infty}"', from=2-1, to=2-2]
\end{tikzcd}\]
\end{frame}

\begin{frame}[fragile]
\begin{prop}
Para $j, f$ y $\hat{f}$ como antes, se cumple que
\begin{enumerate}
\item $j\circ \hat{f}\leq f\circ j$.
\item $j\circ \hat{f}^\infty \leq f^\infty\circ j$.
\end{enumerate}
\end{prop}

Definiendo $H=f^\infty\circ j$ y $h=H_{\mid A_{j_*F}}$, obtenemos el diagrama
\begin{equation}\label{diagrama h}
	\begin{tikzcd}
	A & {A_{\hat{f}^\infty}} \\
	{A_j} & {A_{f^\infty}}
	\arrow["{\hat{f}^\infty}", from=1-1, to=1-2]
	\arrow["j"', from=1-1, to=2-1]
	\arrow["H"', from=1-1, to=2-2]
	\arrow["h", from=1-2, to=2-2]
	\arrow["{f^\infty}"', from=2-1, to=2-2]
\end{tikzcd}
\end{equation}

\begin{prop}
	El diagrama \eqref{diagrama h} es conmutativo.
\end{prop}
\end{frame}

\begin{frame}[fragile]
Consideremos 
\[\begin{tikzcd}
	A && {A_F} \\
	&&& {\mathcal{O}Q} \\
	{\mathcal{O}S} && {\mathcal{O}S_\nabla}
	\arrow["{v_F}", from=1-1, to=1-3]
	\arrow["{U_A}"', from=1-1, to=3-1]
	\arrow[from=1-3, to=2-4]
	\arrow[from=1-3, to=3-3]
	\arrow["{v_\nabla}"', from=3-1, to=3-3]
	\arrow[from=3-3, to=2-4]
\end{tikzcd}\]
¿Qué pasa si $A$ cumple $\mathbf{(H)}$?
\end{frame}

\begin{frame}{Marcos $KC$}
$S\in \Top$ es $\mathrm{KC}$ si todo conjunto compacto es cerrado. $S$ es $\mathrm{US}$ si cada sucesión convergente tiene exactamente un límite al cual converge.
\[
T_2\Rightarrow KC\Rightarrow US\Rightarrow T_1
\]

\begin{dfn}
$A\in \Frm$ es $KC$ si todo cociente compacto de $A$ es cerrado. 
\end{dfn}
Equivalentemente
\[
A_F=u_d
\]
para algún $d\in A$ y $F\in A^\wedge$.
\end{frame}

\begin{frame}[fragile]{Propiedades de los marcos $KC$}
\[
KC\Rightarrow \mbox{Arreglado}
\]

\begin{prop}
Si $A$ es $KC$ entonces $A_j$ es $KC$ para todo $j\in NA$.
\end{prop}

\begin{prop}
Si $A$ es $KC$, entonces $A$ es $T_1$.
\end{prop}

De hecho
\[\begin{tikzcd}
	{\mathbf{(reg)}} & {\mathbf{(fH)}} & KC & {\mbox{Arreglado}} & {T_1} \\
	& {\mathbf{(aju)}} && {}
	\arrow[Rightarrow, from=1-1, to=1-2]
	\arrow[Rightarrow, from=1-1, to=2-2]
	\arrow[Rightarrow, from=1-2, to=1-3]
	\arrow[Rightarrow, from=1-3, to=1-4]
	\arrow[Rightarrow, from=1-4, to=1-5]
	\arrow[Rightarrow, from=2-2, to=1-3]
\end{tikzcd}\]

\end{frame}

\begin{frame}{La topología máximo compacta}
Consideremos $S=\{x,y\}\cup \mathbb{N}^2$ con $x,y\notin \mathbb{N}^2$ y sea 
\[
R_n=\{(m,n)\mid m\in \mathbb{N}\}
\]
Definimos 
\[
\mathcal{O}S=\mathcal{P}\mathbb{N}^2 \cup \mathcal{U}\cup\mathcal{V}
\]
donde
\[
\mathcal{U}=\{U\subseteq S\mid x\in U\mbox{ y }\forall n\in \mathbb{N}, U\cap R_n\mbox{ es cofinito}\}
\]
\[
\mathcal{V}=\{V\subseteq S\mid y\in V \mbox{ y }\exists F\subseteq \mathbb{N}\mbox{ finito tal que }\forall n\notin F, R_n\subseteq V \}
\]
$\mathcal{O}S$ es una topología..
\end{frame}

\begin{frame}{Propiedades de $\mathcal{O}S$}
	\begin{itemize}
	\item $\mathcal{O}S$ es $T_1$.
	\item $\mathcal{O}S$ no es $\mathbf{(H)}$.
	\item $\mathcal{O}S$ es compacto.
	\item $\mathcal{O}S$ es $\mathbf{(aju)}$.
	\item $\mathcal{O}S$ es $KC$.
	\item $\mathcal{O}S$ es $2$-arreglado.
	\end{itemize}
\end{frame}

\begin{frame}{El ejemplo de Paseka y Smarda}
Consideremos $A\in\Frm$ y $A_r=\{a\in A\mid \neg\neg a=a\}$. Definimos
\[
K(A)=\{(u,v)\mid u\in A,\, v\in A_r,\, u\leq v\}
\]
$K(A)\in \Frm$.
\begin{block}{Propiedades de $K(A)$}
\begin{itemize}
	\item Si $A$ es $\mathbf{(H)}$ y $\neg m=0$ para $m$ máximo, $K(A)$ es $\mathbf{(H)}$.
	\item Si $A$ es compacto, entonces $K(A)$ es compacto.
	\item $K(A)$ no es subajustado
\end{itemize}
\end{block}
\end{frame}

\begin{frame}
De manera adicional, sea $A=[0,1]$ con la topología usual. Entonces
\begin{itemize}
\item $\mathcal{O}I$ es $\mathbf{(H)}$.
\item $\mathcal{O}I$ es compacto.
\item $K(\mathcal{O}I)$ es compacto y $\mathbf{(H)}$.
\item $K(\mathcal{O}I)$ no es subajustado.
\item $K(\mathcal{O}I)$ no es espacial.
\end{itemize}

\[
\mbox{Existe marcos Hausdorr y compactos que no son espaciales}.
\]
\end{frame}

\section{Conclusiones}
\begin{frame}{Conclusiones}
\begin{itemize}
\item Requerimos identificar cual es significado de que un marco sea $US$.
\item Conocer más sobre el comportamiento del marco de parches para niveles superiores.
\item Verificar que $\mathbf{(H)}$ implica (o no) $KC$ (o arreglado).
\item Desarrollar ejemplos donde aparezcan las propiedades que están involucradas en la investigación.
\item Explorar las posibilidades que brindan los intervalos de admisibilidad y el $Q$-cuadrado.
\end{itemize}
\end{frame}


\End

\section*{\textsc{Referencias}}
\begin{frame}[allowframebreaks]
\frametitle{Bibliografía}
\begin{thebibliography}{20}\markboth{Bibliografía}{Bibliografía}

\bibitem{P.T.} P. T. Johnstone, \textit{Stone spaces}, Cambridge Studies in Advanced Mathematics, vol. 3, Cambridge University Press, Cambridge, 1982. MR 698074

\bibitem{J.M.} J. Monter; A. Zaldívar, \textit{El enfoque locálico de las reflexiones booleanas: un análisis en la categoría de marcos} [tesis de maestría], 2022. Universidad de Guadalajara.

\bibitem{P.S.} J. Paseka and B. Smarda, \textit{$ T_2 $-frames and almost compact frames.} Czechoslovak Mathematical Journal (1992), 42(3), 385-402.

\bibitem{J.P.} J. Picado and A. Pultr, \textit{Frames and locales: Topology without points}, Frontiers in Mathematics, Springer Basel, 2012.

\bibitem{J.P.2} J. Picado and A. Pultr, \textit{Separation in point-free topology}, Springer, 2021.

\bibitem{R.S.} RA Sexton, \textit{A point free and point-sensitive analysis of the patch assembly}, The University of Manchester (United Kingdom), 2003.

\bibitem{R.S.2} RA Sexton, \textit{Frame theoretic assembly as a unifying construct}, The University of Manchester (United Kingdom), 2000.

\bibitem{R.S.3} RA Sexton and H. Simmons, \textit{Point-sensitive and point-free patch constructions}, Journal of Pure and Applied Algebra \textbf{207} (2006), no. 2, 433-468.

\bibitem{H.S.} H. Simmons, \textit{An Introduction to Frame Theory}, lecture notes, University of Manchester. Disponible en línea en \url{https://web.archive.org/web/20190714073511/http://staff.cs.manchester.ac.uk/~hsimmons}.

\bibitem{H.S.R} H. Simmons, \textit{Regularity, fitness, and the block structure of frames.} Applied Categorical Structures 14 (2006): 1-34.

\bibitem{H.S.4} H. Simmons, \textit{The lattice theoretic part of topological separation properties}, Proceedings of the Edinburgh Mathematical Society, vol.~21, pp.~41--48, 1978.

\bibitem{H.S.V} H. Simmons, \textit{The Vietoris modifications of a frame}. Unpublished manuscript (2004), 79pp., available online at http://www. cs. man. ac. uk/hsimmons.

\bibitem{A.W.} A. Wilansky, \textit{Between T1 and T2}, MONTHLY (1967): 261-266.

\bibitem{A.Z.} A. Zaldívar, \textit{Introducción a la teoría de marcos} [notas curso], 2025. Universidad de Guadalajara.
\end{thebibliography}
\end{frame}

\end{document}